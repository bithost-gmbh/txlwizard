\section{Introduction}
    This document describes the usage and technical reference of the python program ``TXLWizard''
    written by Esteban Marin (\href{mailto:estebanmarin@gmx.ch}{estebanmarin@gmx.ch}).\\

    \subsection{What does it do?}
        The ``TXLWizard'' provides routines for generating TXL files (.txl) for
        the preparation of E-Beam lithography masks using python code. The TXL files can be processed with BEAMER.
        See the following links:
        \begin{itemize}
            \item \url{http://genisys-gmbh.com/web/products/beamer.html}
            \item \url{http://cad035.psi.ch/LB_index.html}
            \item \url{http://cad035.psi.ch/LBDoc/BEAMER_Manual.pdf}
        \end{itemize}
        The generated TXL files are also converted to HTML / SVG for presentation in any modern browser or
        vector graphics application.\\
        Moreover, a command line interface ``TXLConverter'' provides conversion of existing TXL files to HTML / SVG
        (See Section \ref{sec:TXLConverter}).


    \subsection{Technical Information}
        The ``TXLWizard'' is written in python and will run in Python version 2.7+ and 3.1+.\\
        In order to use the ``TXLWizard'', package must be available as
        a python package, i.e. either it must be copied to \myPath{Path\_to\_my\_python\_installation/site-packages/} or
        to the path where your script is located. \\
        Alternatively, you can also prepend the following command to your python script:\\
        \commandline{sys.path.append('path to the folder containing TXLWizard')}\\




