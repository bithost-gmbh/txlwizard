% Generated by Sphinx.
\def\sphinxdocclass{report}
\documentclass[letterpaper,10pt,english]{sphinxmanual}

\usepackage[utf8]{inputenc}
\ifdefined\DeclareUnicodeCharacter
  \DeclareUnicodeCharacter{00A0}{\nobreakspace}
\else\fi
\usepackage{cmap}
\usepackage[T1]{fontenc}
\usepackage{amsmath,amssymb}
\usepackage{babel}
\usepackage{times}
\usepackage[Bjarne]{fncychap}
\usepackage{longtable}
\usepackage{sphinx}
\usepackage{multirow}
\usepackage{eqparbox}


\addto\captionsenglish{\renewcommand{\figurename}{Fig. }}
\addto\captionsenglish{\renewcommand{\tablename}{Table }}
\SetupFloatingEnvironment{literal-block}{name=Listing }

\addto\extrasenglish{\def\pageautorefname{page}}

\setcounter{tocdepth}{1}


\title{TXLWizard Documentation}
\date{May 19, 2016}
\release{1.0.0}
\author{Esteban Marin}
\newcommand{\sphinxlogo}{}
\renewcommand{\releasename}{Release}
\makeindex

\makeatletter
\def\PYG@reset{\let\PYG@it=\relax \let\PYG@bf=\relax%
    \let\PYG@ul=\relax \let\PYG@tc=\relax%
    \let\PYG@bc=\relax \let\PYG@ff=\relax}
\def\PYG@tok#1{\csname PYG@tok@#1\endcsname}
\def\PYG@toks#1+{\ifx\relax#1\empty\else%
    \PYG@tok{#1}\expandafter\PYG@toks\fi}
\def\PYG@do#1{\PYG@bc{\PYG@tc{\PYG@ul{%
    \PYG@it{\PYG@bf{\PYG@ff{#1}}}}}}}
\def\PYG#1#2{\PYG@reset\PYG@toks#1+\relax+\PYG@do{#2}}

\expandafter\def\csname PYG@tok@gd\endcsname{\def\PYG@tc##1{\textcolor[rgb]{0.63,0.00,0.00}{##1}}}
\expandafter\def\csname PYG@tok@gu\endcsname{\let\PYG@bf=\textbf\def\PYG@tc##1{\textcolor[rgb]{0.50,0.00,0.50}{##1}}}
\expandafter\def\csname PYG@tok@gt\endcsname{\def\PYG@tc##1{\textcolor[rgb]{0.00,0.27,0.87}{##1}}}
\expandafter\def\csname PYG@tok@gs\endcsname{\let\PYG@bf=\textbf}
\expandafter\def\csname PYG@tok@gr\endcsname{\def\PYG@tc##1{\textcolor[rgb]{1.00,0.00,0.00}{##1}}}
\expandafter\def\csname PYG@tok@cm\endcsname{\let\PYG@it=\textit\def\PYG@tc##1{\textcolor[rgb]{0.25,0.50,0.56}{##1}}}
\expandafter\def\csname PYG@tok@vg\endcsname{\def\PYG@tc##1{\textcolor[rgb]{0.73,0.38,0.84}{##1}}}
\expandafter\def\csname PYG@tok@vi\endcsname{\def\PYG@tc##1{\textcolor[rgb]{0.73,0.38,0.84}{##1}}}
\expandafter\def\csname PYG@tok@mh\endcsname{\def\PYG@tc##1{\textcolor[rgb]{0.13,0.50,0.31}{##1}}}
\expandafter\def\csname PYG@tok@cs\endcsname{\def\PYG@tc##1{\textcolor[rgb]{0.25,0.50,0.56}{##1}}\def\PYG@bc##1{\setlength{\fboxsep}{0pt}\colorbox[rgb]{1.00,0.94,0.94}{\strut ##1}}}
\expandafter\def\csname PYG@tok@ge\endcsname{\let\PYG@it=\textit}
\expandafter\def\csname PYG@tok@vc\endcsname{\def\PYG@tc##1{\textcolor[rgb]{0.73,0.38,0.84}{##1}}}
\expandafter\def\csname PYG@tok@il\endcsname{\def\PYG@tc##1{\textcolor[rgb]{0.13,0.50,0.31}{##1}}}
\expandafter\def\csname PYG@tok@go\endcsname{\def\PYG@tc##1{\textcolor[rgb]{0.20,0.20,0.20}{##1}}}
\expandafter\def\csname PYG@tok@cp\endcsname{\def\PYG@tc##1{\textcolor[rgb]{0.00,0.44,0.13}{##1}}}
\expandafter\def\csname PYG@tok@gi\endcsname{\def\PYG@tc##1{\textcolor[rgb]{0.00,0.63,0.00}{##1}}}
\expandafter\def\csname PYG@tok@gh\endcsname{\let\PYG@bf=\textbf\def\PYG@tc##1{\textcolor[rgb]{0.00,0.00,0.50}{##1}}}
\expandafter\def\csname PYG@tok@ni\endcsname{\let\PYG@bf=\textbf\def\PYG@tc##1{\textcolor[rgb]{0.84,0.33,0.22}{##1}}}
\expandafter\def\csname PYG@tok@nl\endcsname{\let\PYG@bf=\textbf\def\PYG@tc##1{\textcolor[rgb]{0.00,0.13,0.44}{##1}}}
\expandafter\def\csname PYG@tok@nn\endcsname{\let\PYG@bf=\textbf\def\PYG@tc##1{\textcolor[rgb]{0.05,0.52,0.71}{##1}}}
\expandafter\def\csname PYG@tok@no\endcsname{\def\PYG@tc##1{\textcolor[rgb]{0.38,0.68,0.84}{##1}}}
\expandafter\def\csname PYG@tok@na\endcsname{\def\PYG@tc##1{\textcolor[rgb]{0.25,0.44,0.63}{##1}}}
\expandafter\def\csname PYG@tok@nb\endcsname{\def\PYG@tc##1{\textcolor[rgb]{0.00,0.44,0.13}{##1}}}
\expandafter\def\csname PYG@tok@nc\endcsname{\let\PYG@bf=\textbf\def\PYG@tc##1{\textcolor[rgb]{0.05,0.52,0.71}{##1}}}
\expandafter\def\csname PYG@tok@nd\endcsname{\let\PYG@bf=\textbf\def\PYG@tc##1{\textcolor[rgb]{0.33,0.33,0.33}{##1}}}
\expandafter\def\csname PYG@tok@ne\endcsname{\def\PYG@tc##1{\textcolor[rgb]{0.00,0.44,0.13}{##1}}}
\expandafter\def\csname PYG@tok@nf\endcsname{\def\PYG@tc##1{\textcolor[rgb]{0.02,0.16,0.49}{##1}}}
\expandafter\def\csname PYG@tok@si\endcsname{\let\PYG@it=\textit\def\PYG@tc##1{\textcolor[rgb]{0.44,0.63,0.82}{##1}}}
\expandafter\def\csname PYG@tok@s2\endcsname{\def\PYG@tc##1{\textcolor[rgb]{0.25,0.44,0.63}{##1}}}
\expandafter\def\csname PYG@tok@nt\endcsname{\let\PYG@bf=\textbf\def\PYG@tc##1{\textcolor[rgb]{0.02,0.16,0.45}{##1}}}
\expandafter\def\csname PYG@tok@nv\endcsname{\def\PYG@tc##1{\textcolor[rgb]{0.73,0.38,0.84}{##1}}}
\expandafter\def\csname PYG@tok@s1\endcsname{\def\PYG@tc##1{\textcolor[rgb]{0.25,0.44,0.63}{##1}}}
\expandafter\def\csname PYG@tok@ch\endcsname{\let\PYG@it=\textit\def\PYG@tc##1{\textcolor[rgb]{0.25,0.50,0.56}{##1}}}
\expandafter\def\csname PYG@tok@m\endcsname{\def\PYG@tc##1{\textcolor[rgb]{0.13,0.50,0.31}{##1}}}
\expandafter\def\csname PYG@tok@gp\endcsname{\let\PYG@bf=\textbf\def\PYG@tc##1{\textcolor[rgb]{0.78,0.36,0.04}{##1}}}
\expandafter\def\csname PYG@tok@sh\endcsname{\def\PYG@tc##1{\textcolor[rgb]{0.25,0.44,0.63}{##1}}}
\expandafter\def\csname PYG@tok@ow\endcsname{\let\PYG@bf=\textbf\def\PYG@tc##1{\textcolor[rgb]{0.00,0.44,0.13}{##1}}}
\expandafter\def\csname PYG@tok@sx\endcsname{\def\PYG@tc##1{\textcolor[rgb]{0.78,0.36,0.04}{##1}}}
\expandafter\def\csname PYG@tok@bp\endcsname{\def\PYG@tc##1{\textcolor[rgb]{0.00,0.44,0.13}{##1}}}
\expandafter\def\csname PYG@tok@c1\endcsname{\let\PYG@it=\textit\def\PYG@tc##1{\textcolor[rgb]{0.25,0.50,0.56}{##1}}}
\expandafter\def\csname PYG@tok@o\endcsname{\def\PYG@tc##1{\textcolor[rgb]{0.40,0.40,0.40}{##1}}}
\expandafter\def\csname PYG@tok@kc\endcsname{\let\PYG@bf=\textbf\def\PYG@tc##1{\textcolor[rgb]{0.00,0.44,0.13}{##1}}}
\expandafter\def\csname PYG@tok@c\endcsname{\let\PYG@it=\textit\def\PYG@tc##1{\textcolor[rgb]{0.25,0.50,0.56}{##1}}}
\expandafter\def\csname PYG@tok@mf\endcsname{\def\PYG@tc##1{\textcolor[rgb]{0.13,0.50,0.31}{##1}}}
\expandafter\def\csname PYG@tok@err\endcsname{\def\PYG@bc##1{\setlength{\fboxsep}{0pt}\fcolorbox[rgb]{1.00,0.00,0.00}{1,1,1}{\strut ##1}}}
\expandafter\def\csname PYG@tok@mb\endcsname{\def\PYG@tc##1{\textcolor[rgb]{0.13,0.50,0.31}{##1}}}
\expandafter\def\csname PYG@tok@ss\endcsname{\def\PYG@tc##1{\textcolor[rgb]{0.32,0.47,0.09}{##1}}}
\expandafter\def\csname PYG@tok@sr\endcsname{\def\PYG@tc##1{\textcolor[rgb]{0.14,0.33,0.53}{##1}}}
\expandafter\def\csname PYG@tok@mo\endcsname{\def\PYG@tc##1{\textcolor[rgb]{0.13,0.50,0.31}{##1}}}
\expandafter\def\csname PYG@tok@kd\endcsname{\let\PYG@bf=\textbf\def\PYG@tc##1{\textcolor[rgb]{0.00,0.44,0.13}{##1}}}
\expandafter\def\csname PYG@tok@mi\endcsname{\def\PYG@tc##1{\textcolor[rgb]{0.13,0.50,0.31}{##1}}}
\expandafter\def\csname PYG@tok@kn\endcsname{\let\PYG@bf=\textbf\def\PYG@tc##1{\textcolor[rgb]{0.00,0.44,0.13}{##1}}}
\expandafter\def\csname PYG@tok@cpf\endcsname{\let\PYG@it=\textit\def\PYG@tc##1{\textcolor[rgb]{0.25,0.50,0.56}{##1}}}
\expandafter\def\csname PYG@tok@kr\endcsname{\let\PYG@bf=\textbf\def\PYG@tc##1{\textcolor[rgb]{0.00,0.44,0.13}{##1}}}
\expandafter\def\csname PYG@tok@s\endcsname{\def\PYG@tc##1{\textcolor[rgb]{0.25,0.44,0.63}{##1}}}
\expandafter\def\csname PYG@tok@kp\endcsname{\def\PYG@tc##1{\textcolor[rgb]{0.00,0.44,0.13}{##1}}}
\expandafter\def\csname PYG@tok@w\endcsname{\def\PYG@tc##1{\textcolor[rgb]{0.73,0.73,0.73}{##1}}}
\expandafter\def\csname PYG@tok@kt\endcsname{\def\PYG@tc##1{\textcolor[rgb]{0.56,0.13,0.00}{##1}}}
\expandafter\def\csname PYG@tok@sc\endcsname{\def\PYG@tc##1{\textcolor[rgb]{0.25,0.44,0.63}{##1}}}
\expandafter\def\csname PYG@tok@sb\endcsname{\def\PYG@tc##1{\textcolor[rgb]{0.25,0.44,0.63}{##1}}}
\expandafter\def\csname PYG@tok@k\endcsname{\let\PYG@bf=\textbf\def\PYG@tc##1{\textcolor[rgb]{0.00,0.44,0.13}{##1}}}
\expandafter\def\csname PYG@tok@se\endcsname{\let\PYG@bf=\textbf\def\PYG@tc##1{\textcolor[rgb]{0.25,0.44,0.63}{##1}}}
\expandafter\def\csname PYG@tok@sd\endcsname{\let\PYG@it=\textit\def\PYG@tc##1{\textcolor[rgb]{0.25,0.44,0.63}{##1}}}

\def\PYGZbs{\char`\\}
\def\PYGZus{\char`\_}
\def\PYGZob{\char`\{}
\def\PYGZcb{\char`\}}
\def\PYGZca{\char`\^}
\def\PYGZam{\char`\&}
\def\PYGZlt{\char`\<}
\def\PYGZgt{\char`\>}
\def\PYGZsh{\char`\#}
\def\PYGZpc{\char`\%}
\def\PYGZdl{\char`\$}
\def\PYGZhy{\char`\-}
\def\PYGZsq{\char`\'}
\def\PYGZdq{\char`\"}
\def\PYGZti{\char`\~}
% for compatibility with earlier versions
\def\PYGZat{@}
\def\PYGZlb{[}
\def\PYGZrb{]}
\makeatother

\renewcommand\PYGZsq{\textquotesingle}

\begin{document}

\maketitle
\tableofcontents
\phantomsection\label{index::doc}



\chapter{Table of Contents}
\label{index:table-of-contents}\label{index:welcome-to-txlwriter-s-documentation}

\section{Introduction}
\label{Chapters/10_Introduction:introduction}\label{Chapters/10_Introduction::doc}
This document describes the usage and technical reference of the python program \titleref{TXLWizard}
written by Esteban Marin (\href{mailto:estebanmarin@gmx.ch}{estebanmarin@gmx.ch}).


\subsection{What does it do?}
\label{Chapters/10_Introduction:what-does-it-do}
The \titleref{TXLWizard} provides routines for generating TXL files (.txl) for
the preparation of E-Beam lithography masks using python code. The TXL files can be processed with BEAMER.
See the following links:
\begin{itemize}
\item {} 
\url{http://genisys-gmbh.com/web/products/beamer.html}

\item {} 
\url{http://cad035.psi.ch/LB\_index.html}

\item {} 
\url{http://cad035.psi.ch/LBDoc/BEAMER\_Manual.pdf}

\end{itemize}

The \titleref{TXLWizard} currently implements version \titleref{4.8} of the TextLIB (TXL) standard.

The generated TXL files are also converted to HTML / SVG for presentation in any modern browser or
vector graphics application and allow rapid mask development.

Moreover, a command line interface \titleref{TXLConverter} provides conversion of existing TXL files to HTML / SVG
(See Section {\hyperref[Chapters/30_TXLConverter:sec\string-txlconverter]{\crossref{\DUrole{std,std-ref}{TXLConverter}}}}).


\subsection{Installation}
\label{Chapters/10_Introduction:installation}
The ``TXLWizard'' is written in python and will run in Python version 2.7+ and 3.1+.

In order to use it, the \titleref{TXLWizard} package must be available as
a python package, i.e. either it must be copied to
\code{Path\_to\_my\_python\_installation/site-packages/}
or to the path where your script is located.
Alternatively, you can also prepend the following command to your python script:
\code{sys.path.append('path to the folder containing TXLWizard')}


\subsection{Structure / Pattern / Attribute}
\label{Chapters/10_Introduction:structure-pattern-attribute}
The following terms are used throughout this manual:


\subsubsection{\titleref{Structure}}
\label{Chapters/10_Introduction:structure}
Refers to an object containing one or more \titleref{Pattern} objects.
A \titleref{Structure} corresponds to the \titleref{STRUCT} command in TXL files.


\subsubsection{\titleref{Pattern}}
\label{Chapters/10_Introduction:pattern}
Refers to a pattern such as a circle, a polygon, an ellipse, a path, etc.
The following patterns with the corresponding TXL command in brackets are supported:
\begin{itemize}
\item {} 
\titleref{Circle} (\titleref{C})

\item {} 
\titleref{Ellipse} (\titleref{ELP})

\item {} 
\titleref{Polygon} (\titleref{B})

\item {} 
\titleref{Polyline} (\titleref{P})

\item {} 
\titleref{Reference} (\titleref{SREF})

\item {} 
\titleref{Array} (\titleref{AREF})

\end{itemize}

For more information, supported parameters, etc., see Section {\hyperref[Chapters/40_PythonModuleReference:pythonmodulereferencepatterns]{\crossref{\DUrole{std,std-ref}{Patterns}}}}.


\subsubsection{\titleref{Attribute}}
\label{Chapters/10_Introduction:attribute}
Refers to an property of a \titleref{Pattern} determining the visual appearance of the \titleref{Pattern}.
The following attributes with the corresponding TXL command in brackets are supported:
\begin{itemize}
\item {} 
\titleref{Layer} (\titleref{LAYER})

\item {} 
\titleref{DataType} (\titleref{DATATYPE})

\item {} 
\titleref{RotationAngle} (\titleref{ANGLE})

\item {} 
\titleref{StrokeWidth} (\titleref{WIDTH})

\item {} 
\titleref{ScaleFactor} (\titleref{MAG})

\end{itemize}

Please note that the \titleref{TXLWizard} strictly implements the specification of the TXL format.
This implies some peculiarities, such as
\begin{itemize}
\item {} 
\titleref{Attribute} commands preceed the corresponding \titleref{Pattern} in a \titleref{Structure} and are valid for all patterns that follow
unless the attribute value is changed. Therefore, when adding a \titleref{Pattern} to a \titleref{Structure} with certain attributes,
the attributes are valid for any subsequently added pattern, unless a different attribute value is specified.

\item {} 
\titleref{Attribute} commands are valid for all patterns, except for \titleref{Reference} (\titleref{SREF}) and \titleref{Array} (\titleref{AREF}).
Therefore the attributes of a pattern can only be specified in the structure where the pattern is added / defined.

\item {} 
The \titleref{RotationAngle} attribute applies to each \titleref{Pattern} individually and rotates about each \titleref{Pattern}`s individual origin.

\end{itemize}


\subsection{Example SVG Output}
\label{Chapters/10_Introduction:example-svg-output}
An example output can be seen here:
\begin{figure}[htbp]
\centering
\capstart

\includegraphics{{Mask_Example}.png}
\caption{Example SVG output for a mask}\label{Chapters/10_Introduction:id1}\end{figure}


\subsection{How to start?}
\label{Chapters/10_Introduction:how-to-start}
Have a look at the examples in Section {\hyperref[Chapters/20_Examples:examples]{\crossref{\DUrole{std,std-ref}{TXLWizard Examples}}}} and consult the {\hyperref[Chapters/40_PythonModuleReference:pythonmodulereference]{\crossref{\DUrole{std,std-ref}{Python Module Reference}}}}.
Happy scripting!


\section{TXLWizard Examples}
\label{Chapters/20_Examples:txlwizard-examples}\label{Chapters/20_Examples::doc}\label{Chapters/20_Examples:examples}

\subsection{Introductory Example}
\label{Chapters/20_Examples:introductory-example}

\subsubsection{Introduction}
\label{Chapters/20_Examples:introduction}
The following code demonstrates an introductory example usage of the \titleref{TXLWizard} for
generating TXL files with python code.

The code can be found in the file \code{Content/Example\_Introduction.py}.
The resulting SVG image is shown in Figure {\hyperref[Chapters/20_Examples:exampleintroductionsvg]{\crossref{\DUrole{std,std-ref}{Generated SVG Image}}}}.

Have a look at more advanced examples in Sections {\hyperref[Chapters/20_Examples:txlwizardexampleadvanced]{\crossref{\DUrole{std,std-ref}{Advanced Example}}}} and {\hyperref[Chapters/20_Examples:txlwizardexampleadvanced]{\crossref{\DUrole{std,std-ref}{Advanced Example}}}}
and at the {\hyperref[Chapters/40_PythonModuleReference:pythonmodulereference]{\crossref{\DUrole{std,std-ref}{Python Module Reference}}}}.


\subsubsection{Code}
\label{Chapters/20_Examples:code}
\begin{Verbatim}[commandchars=\\\{\},numbers=left,firstnumber=1,stepnumber=1]
\PYG{c+c1}{\PYGZsh{}\PYGZsh{}\PYGZsh{}\PYGZsh{}\PYGZsh{}\PYGZsh{}\PYGZsh{}\PYGZsh{}\PYGZsh{}\PYGZsh{}\PYGZsh{}\PYGZsh{}\PYGZsh{}\PYGZsh{}\PYGZsh{}\PYGZsh{}\PYGZsh{}\PYGZsh{}\PYGZsh{}\PYGZsh{}\PYGZsh{}\PYGZsh{}\PYGZsh{}\PYGZsh{}\PYGZsh{}\PYGZsh{}\PYGZsh{}\PYGZsh{}\PYGZsh{}\PYGZsh{}\PYGZsh{}\PYGZsh{}\PYGZsh{}\PYGZsh{}\PYGZsh{}\PYGZsh{}\PYGZsh{}\PYGZsh{}\PYGZsh{}\PYGZsh{}\PYGZsh{}\PYGZsh{}\PYGZsh{}}
\PYG{c+c1}{\PYGZsh{} Import Libraries / Initialize TXLWriter \PYGZsh{}}
\PYG{c+c1}{\PYGZsh{}\PYGZsh{}\PYGZsh{}\PYGZsh{}\PYGZsh{}\PYGZsh{}\PYGZsh{}\PYGZsh{}\PYGZsh{}\PYGZsh{}\PYGZsh{}\PYGZsh{}\PYGZsh{}\PYGZsh{}\PYGZsh{}\PYGZsh{}\PYGZsh{}\PYGZsh{}\PYGZsh{}\PYGZsh{}\PYGZsh{}\PYGZsh{}\PYGZsh{}\PYGZsh{}\PYGZsh{}\PYGZsh{}\PYGZsh{}\PYGZsh{}\PYGZsh{}\PYGZsh{}\PYGZsh{}\PYGZsh{}\PYGZsh{}\PYGZsh{}\PYGZsh{}\PYGZsh{}\PYGZsh{}\PYGZsh{}\PYGZsh{}\PYGZsh{}\PYGZsh{}\PYGZsh{}\PYGZsh{}}

\PYG{c+c1}{\PYGZsh{} Import TXLWriter, the main class for generating TXL Output}
\PYG{k+kn}{import} \PYG{n+nn}{TXLWizard}\PYG{n+nn}{.}\PYG{n+nn}{TXLWriter}

\PYG{c+c1}{\PYGZsh{} Import Pre\PYGZhy{}Defined Shapes / Structures wrapped in functions}
\PYG{k+kn}{import} \PYG{n+nn}{TXLWizard}\PYG{n+nn}{.}\PYG{n+nn}{ShapeLibrary}\PYG{n+nn}{.}\PYG{n+nn}{Label}

\PYG{c+c1}{\PYGZsh{} Initialize TXLWriter}
\PYG{n}{TXLWriter} \PYG{o}{=} \PYG{n}{TXLWizard}\PYG{o}{.}\PYG{n}{TXLWriter}\PYG{o}{.}\PYG{n}{TXLWriter}\PYG{p}{(}\PYG{p}{)}

\PYG{c+c1}{\PYGZsh{}\PYGZsh{}\PYGZsh{}\PYGZsh{}\PYGZsh{}\PYGZsh{}\PYGZsh{}\PYGZsh{}\PYGZsh{}\PYGZsh{}\PYGZsh{}\PYGZsh{}\PYGZsh{}\PYGZsh{}\PYGZsh{}\PYGZsh{}\PYGZsh{}\PYGZsh{}\PYGZsh{}\PYGZsh{}\PYGZsh{}}
\PYG{c+c1}{\PYGZsh{} Define Structures \PYGZsh{}}
\PYG{c+c1}{\PYGZsh{}\PYGZsh{}\PYGZsh{}\PYGZsh{}\PYGZsh{}\PYGZsh{}\PYGZsh{}\PYGZsh{}\PYGZsh{}\PYGZsh{}\PYGZsh{}\PYGZsh{}\PYGZsh{}\PYGZsh{}\PYGZsh{}\PYGZsh{}\PYGZsh{}\PYGZsh{}\PYGZsh{}\PYGZsh{}\PYGZsh{}}

\PYG{c+c1}{\PYGZsh{}\PYGZsh{} Sample Label \PYGZsh{}\PYGZsh{}}

\PYG{c+c1}{\PYGZsh{} Give the sample a nice label}
\PYG{n}{SampleLabelObject} \PYG{o}{=} \PYG{n}{TXLWizard}\PYG{o}{.}\PYG{n}{ShapeLibrary}\PYG{o}{.}\PYG{n}{Label}\PYG{o}{.}\PYG{n}{GetLabel}\PYG{p}{(}
    \PYG{n}{TXLWriter}\PYG{p}{,}
    \PYG{n}{Text}\PYG{o}{=}\PYG{l+s+s1}{\PYGZsq{}}\PYG{l+s+s1}{This is my text}\PYG{l+s+s1}{\PYGZsq{}}\PYG{p}{,}
    \PYG{n}{OriginPoint}\PYG{o}{=}\PYG{p}{[}\PYG{o}{\PYGZhy{}}\PYG{l+m+mi}{310}\PYG{p}{,} \PYG{l+m+mi}{240}\PYG{p}{]}\PYG{p}{,}
    \PYG{n}{FontSize}\PYG{o}{=}\PYG{l+m+mi}{50}\PYG{p}{,}
    \PYG{n}{StrokeWidth}\PYG{o}{=}\PYG{l+m+mi}{5}\PYG{p}{,}
    \PYG{n}{RoundCaps}\PYG{o}{=}\PYG{k+kc}{True}\PYG{p}{,}  \PYG{c+c1}{\PYGZsh{} Set to False to improve e\PYGZhy{}Beam performance}
    \PYG{n}{Layer}\PYG{o}{=}\PYG{l+m+mi}{1}
\PYG{p}{)}

\PYG{c+c1}{\PYGZsh{}\PYGZsh{} User Structure: Circle \PYGZsh{}\PYGZsh{}}

\PYG{c+c1}{\PYGZsh{} Create Content Structure for Circle}
\PYG{n}{CircleStructure} \PYG{o}{=} \PYG{n}{TXLWriter}\PYG{o}{.}\PYG{n}{AddContentStructure}\PYG{p}{(}\PYG{l+s+s1}{\PYGZsq{}}\PYG{l+s+s1}{Circle}\PYG{l+s+s1}{\PYGZsq{}}\PYG{p}{)}

\PYG{c+c1}{\PYGZsh{} Add a {}`Pattern{}` of type {}`Circle{}`}
\PYG{n}{CircleStructure}\PYG{o}{.}\PYG{n}{AddPattern}\PYG{p}{(}
    \PYG{l+s+s1}{\PYGZsq{}}\PYG{l+s+s1}{Circle}\PYG{l+s+s1}{\PYGZsq{}}\PYG{p}{,}
    \PYG{n}{Center}\PYG{o}{=}\PYG{p}{[}\PYG{l+m+mi}{0}\PYG{p}{,} \PYG{l+m+mi}{0}\PYG{p}{]}\PYG{p}{,}
    \PYG{n}{Radius}\PYG{o}{=}\PYG{l+m+mi}{150}\PYG{p}{,}
    \PYG{n}{Layer}\PYG{o}{=}\PYG{l+m+mi}{2}
\PYG{p}{)}

\PYG{c+c1}{\PYGZsh{}\PYGZsh{}\PYGZsh{}\PYGZsh{}\PYGZsh{}\PYGZsh{}\PYGZsh{}\PYGZsh{}\PYGZsh{}\PYGZsh{}\PYGZsh{}\PYGZsh{}\PYGZsh{}\PYGZsh{}\PYGZsh{}\PYGZsh{}\PYGZsh{}\PYGZsh{}\PYGZsh{}\PYGZsh{}\PYGZsh{}\PYGZsh{}\PYGZsh{}\PYGZsh{}\PYGZsh{}}
\PYG{c+c1}{\PYGZsh{} Generate Output Files \PYGZsh{}}
\PYG{c+c1}{\PYGZsh{}\PYGZsh{}\PYGZsh{}\PYGZsh{}\PYGZsh{}\PYGZsh{}\PYGZsh{}\PYGZsh{}\PYGZsh{}\PYGZsh{}\PYGZsh{}\PYGZsh{}\PYGZsh{}\PYGZsh{}\PYGZsh{}\PYGZsh{}\PYGZsh{}\PYGZsh{}\PYGZsh{}\PYGZsh{}\PYGZsh{}\PYGZsh{}\PYGZsh{}\PYGZsh{}\PYGZsh{}}

\PYG{c+c1}{\PYGZsh{} Note: The suffix (.txl, .html, .svg) will be appended automatically}
\PYG{n}{TXLWriter}\PYG{o}{.}\PYG{n}{GenerateFiles}\PYG{p}{(}\PYG{l+s+s1}{\PYGZsq{}}\PYG{l+s+s1}{Masks/Example\PYGZus{}Introduction}\PYG{l+s+s1}{\PYGZsq{}}\PYG{p}{)}
\end{Verbatim}


\subsubsection{Generated SVG Image}
\label{Chapters/20_Examples:generated-svg-image}\label{Chapters/20_Examples:exampleintroductionsvg}\begin{figure}[htbp]
\centering
\capstart

\includegraphics{{Example_Introduction}.png}
\caption{Generated SVG Image for \titleref{Content/Example\_Introduction.py}}\label{Chapters/20_Examples:id7}\end{figure}


\subsection{Simple Example}
\label{Chapters/20_Examples:txlwizardexamplesimple}\label{Chapters/20_Examples:simple-example}

\subsubsection{Introduction}
\label{Chapters/20_Examples:id1}
The following code demonstrates a simple example usage of the \titleref{TXLWizard} for
generating TXL files with python code.

The code can be found in the file \code{Content/Example\_Simple.py}.
The resulting SVG image is shown in Figure {\hyperref[Chapters/20_Examples:examplesimplesvg]{\crossref{\DUrole{std,std-ref}{Generated SVG Image}}}}.

A more advanced example is shown in Section {\hyperref[Chapters/20_Examples:txlwizardexampleadvanced]{\crossref{\DUrole{std,std-ref}{Advanced Example}}}}


\subsubsection{Code}
\label{Chapters/20_Examples:id2}
\begin{Verbatim}[commandchars=\\\{\},numbers=left,firstnumber=1,stepnumber=1]
\PYG{c+c1}{\PYGZsh{}\PYGZsh{}\PYGZsh{}\PYGZsh{}\PYGZsh{}\PYGZsh{}\PYGZsh{}\PYGZsh{}\PYGZsh{}\PYGZsh{}\PYGZsh{}\PYGZsh{}\PYGZsh{}\PYGZsh{}\PYGZsh{}\PYGZsh{}\PYGZsh{}\PYGZsh{}\PYGZsh{}\PYGZsh{}}
\PYG{c+c1}{\PYGZsh{} Import Libraries \PYGZsh{}}
\PYG{c+c1}{\PYGZsh{}\PYGZsh{}\PYGZsh{}\PYGZsh{}\PYGZsh{}\PYGZsh{}\PYGZsh{}\PYGZsh{}\PYGZsh{}\PYGZsh{}\PYGZsh{}\PYGZsh{}\PYGZsh{}\PYGZsh{}\PYGZsh{}\PYGZsh{}\PYGZsh{}\PYGZsh{}\PYGZsh{}\PYGZsh{}}

\PYG{c+c1}{\PYGZsh{} Import TXLWriter, the main class for generating TXL Output}
\PYG{k+kn}{import} \PYG{n+nn}{TXLWizard}\PYG{n+nn}{.}\PYG{n+nn}{TXLWriter}

\PYG{c+c1}{\PYGZsh{} Import Pre\PYGZhy{}Defined Shapes / Structures wrapped in functions}
\PYG{k+kn}{import} \PYG{n+nn}{TXLWizard}\PYG{n+nn}{.}\PYG{n+nn}{ShapeLibrary}\PYG{n+nn}{.}\PYG{n+nn}{EndpointDetectionWindows}
\PYG{k+kn}{import} \PYG{n+nn}{TXLWizard}\PYG{n+nn}{.}\PYG{n+nn}{ShapeLibrary}\PYG{n+nn}{.}\PYG{n+nn}{Label}

\PYG{c+c1}{\PYGZsh{} Import math module for calculations}
\PYG{k+kn}{import} \PYG{n+nn}{math}


\PYG{c+c1}{\PYGZsh{}\PYGZsh{}\PYGZsh{}\PYGZsh{}\PYGZsh{}\PYGZsh{}\PYGZsh{}\PYGZsh{}\PYGZsh{}\PYGZsh{}\PYGZsh{}\PYGZsh{}\PYGZsh{}\PYGZsh{}\PYGZsh{}\PYGZsh{}\PYGZsh{}\PYGZsh{}\PYGZsh{}\PYGZsh{}\PYGZsh{}\PYGZsh{}\PYGZsh{}\PYGZsh{}\PYGZsh{}\PYGZsh{}\PYGZsh{}\PYGZsh{}\PYGZsh{}\PYGZsh{}\PYGZsh{}\PYGZsh{}\PYGZsh{}}
\PYG{c+c1}{\PYGZsh{} Sample / Structure Parameters \PYGZsh{}}
\PYG{c+c1}{\PYGZsh{}\PYGZsh{}\PYGZsh{}\PYGZsh{}\PYGZsh{}\PYGZsh{}\PYGZsh{}\PYGZsh{}\PYGZsh{}\PYGZsh{}\PYGZsh{}\PYGZsh{}\PYGZsh{}\PYGZsh{}\PYGZsh{}\PYGZsh{}\PYGZsh{}\PYGZsh{}\PYGZsh{}\PYGZsh{}\PYGZsh{}\PYGZsh{}\PYGZsh{}\PYGZsh{}\PYGZsh{}\PYGZsh{}\PYGZsh{}\PYGZsh{}\PYGZsh{}\PYGZsh{}\PYGZsh{}\PYGZsh{}\PYGZsh{}}

\PYG{c+c1}{\PYGZsh{} Define all sample parameters}
\PYG{n}{SampleParameters} \PYG{o}{=} \PYG{p}{\PYGZob{}}
    \PYG{l+s+s1}{\PYGZsq{}}\PYG{l+s+s1}{Width}\PYG{l+s+s1}{\PYGZsq{}}\PYG{p}{:} \PYG{l+m+mi}{8}\PYG{n}{e3}\PYG{p}{,}
    \PYG{l+s+s1}{\PYGZsq{}}\PYG{l+s+s1}{Height}\PYG{l+s+s1}{\PYGZsq{}}\PYG{p}{:} \PYG{l+m+mi}{8}\PYG{n}{e3}\PYG{p}{,}
    \PYG{l+s+s1}{\PYGZsq{}}\PYG{l+s+s1}{Label}\PYG{l+s+s1}{\PYGZsq{}}\PYG{p}{:} \PYG{l+s+s1}{\PYGZsq{}}\PYG{l+s+s1}{Simple Demo}\PYG{l+s+s1}{\PYGZsq{}}\PYG{p}{,}
\PYG{p}{\PYGZcb{}}

\PYG{c+c1}{\PYGZsh{} Define all structure parameters}
\PYG{n}{StructureParameters} \PYG{o}{=} \PYG{p}{\PYGZob{}}
    \PYG{l+s+s1}{\PYGZsq{}}\PYG{l+s+s1}{Circle}\PYG{l+s+s1}{\PYGZsq{}}\PYG{p}{:} \PYG{p}{\PYGZob{}}
        \PYG{l+s+s1}{\PYGZsq{}}\PYG{l+s+s1}{Radius}\PYG{l+s+s1}{\PYGZsq{}}\PYG{p}{:} \PYG{l+m+mi}{50}\PYG{p}{,}
        \PYG{l+s+s1}{\PYGZsq{}}\PYG{l+s+s1}{Layer}\PYG{l+s+s1}{\PYGZsq{}}\PYG{p}{:} \PYG{l+m+mi}{3}
    \PYG{p}{\PYGZcb{}}\PYG{p}{,}
    \PYG{l+s+s1}{\PYGZsq{}}\PYG{l+s+s1}{CircleArray}\PYG{l+s+s1}{\PYGZsq{}}\PYG{p}{:} \PYG{p}{\PYGZob{}}
        \PYG{l+s+s1}{\PYGZsq{}}\PYG{l+s+s1}{Columns}\PYG{l+s+s1}{\PYGZsq{}}\PYG{p}{:} \PYG{l+m+mi}{6}\PYG{p}{,}
        \PYG{l+s+s1}{\PYGZsq{}}\PYG{l+s+s1}{Rows}\PYG{l+s+s1}{\PYGZsq{}}\PYG{p}{:} \PYG{l+m+mi}{5}\PYG{p}{,}
        \PYG{l+s+s1}{\PYGZsq{}}\PYG{l+s+s1}{ArrayXOffset}\PYG{l+s+s1}{\PYGZsq{}}\PYG{p}{:} \PYG{l+m+mi}{500}\PYG{p}{,}
        \PYG{l+s+s1}{\PYGZsq{}}\PYG{l+s+s1}{ArrayYOffset}\PYG{l+s+s1}{\PYGZsq{}}\PYG{p}{:} \PYG{o}{\PYGZhy{}}\PYG{l+m+mi}{500}\PYG{p}{,}
        \PYG{l+s+s1}{\PYGZsq{}}\PYG{l+s+s1}{ArrayOrigin}\PYG{l+s+s1}{\PYGZsq{}}\PYG{p}{:} \PYG{p}{[}\PYG{l+m+mf}{0.75e3}\PYG{p}{,} \PYG{l+m+mi}{3}\PYG{n}{e3}\PYG{p}{]}\PYG{p}{,}
        \PYG{l+s+s1}{\PYGZsq{}}\PYG{l+s+s1}{Label}\PYG{l+s+s1}{\PYGZsq{}}\PYG{p}{:} \PYG{l+s+s1}{\PYGZsq{}}\PYG{l+s+s1}{R}\PYG{l+s+si}{\PYGZob{}:d\PYGZcb{}}\PYG{l+s+s1}{C}\PYG{l+s+si}{\PYGZob{}:d\PYGZcb{}}\PYG{l+s+s1}{\PYGZsq{}}\PYG{p}{,}
    \PYG{p}{\PYGZcb{}}
\PYG{p}{\PYGZcb{}}


\PYG{c+c1}{\PYGZsh{}\PYGZsh{}\PYGZsh{}\PYGZsh{}\PYGZsh{}\PYGZsh{}\PYGZsh{}\PYGZsh{}\PYGZsh{}\PYGZsh{}\PYGZsh{}\PYGZsh{}\PYGZsh{}\PYGZsh{}\PYGZsh{}\PYGZsh{}\PYGZsh{}\PYGZsh{}\PYGZsh{}\PYGZsh{}\PYGZsh{}\PYGZsh{}\PYGZsh{}\PYGZsh{}}
\PYG{c+c1}{\PYGZsh{} Initialize TXLWriter \PYGZsh{}}
\PYG{c+c1}{\PYGZsh{}\PYGZsh{}\PYGZsh{}\PYGZsh{}\PYGZsh{}\PYGZsh{}\PYGZsh{}\PYGZsh{}\PYGZsh{}\PYGZsh{}\PYGZsh{}\PYGZsh{}\PYGZsh{}\PYGZsh{}\PYGZsh{}\PYGZsh{}\PYGZsh{}\PYGZsh{}\PYGZsh{}\PYGZsh{}\PYGZsh{}\PYGZsh{}\PYGZsh{}\PYGZsh{}}
\PYG{n}{TXLWriter} \PYG{o}{=} \PYG{n}{TXLWizard}\PYG{o}{.}\PYG{n}{TXLWriter}\PYG{o}{.}\PYG{n}{TXLWriter}\PYG{p}{(}
    \PYG{n}{GridWidth}\PYG{o}{=}\PYG{n}{SampleParameters}\PYG{p}{[}\PYG{l+s+s1}{\PYGZsq{}}\PYG{l+s+s1}{Width}\PYG{l+s+s1}{\PYGZsq{}}\PYG{p}{]}\PYG{p}{,}
    \PYG{n}{GridHeight}\PYG{o}{=}\PYG{n}{SampleParameters}\PYG{p}{[}\PYG{l+s+s1}{\PYGZsq{}}\PYG{l+s+s1}{Height}\PYG{l+s+s1}{\PYGZsq{}}\PYG{p}{]}
\PYG{p}{)}

\PYG{c+c1}{\PYGZsh{}\PYGZsh{}\PYGZsh{}\PYGZsh{}\PYGZsh{}\PYGZsh{}\PYGZsh{}\PYGZsh{}\PYGZsh{}\PYGZsh{}\PYGZsh{}\PYGZsh{}\PYGZsh{}\PYGZsh{}\PYGZsh{}\PYGZsh{}\PYGZsh{}\PYGZsh{}\PYGZsh{}\PYGZsh{}\PYGZsh{}}
\PYG{c+c1}{\PYGZsh{} Define Structures \PYGZsh{}}
\PYG{c+c1}{\PYGZsh{}\PYGZsh{}\PYGZsh{}\PYGZsh{}\PYGZsh{}\PYGZsh{}\PYGZsh{}\PYGZsh{}\PYGZsh{}\PYGZsh{}\PYGZsh{}\PYGZsh{}\PYGZsh{}\PYGZsh{}\PYGZsh{}\PYGZsh{}\PYGZsh{}\PYGZsh{}\PYGZsh{}\PYGZsh{}\PYGZsh{}}

\PYG{c+c1}{\PYGZsh{}\PYGZsh{} Sample Label \PYGZsh{}\PYGZsh{}}

\PYG{c+c1}{\PYGZsh{} Give the sample a nice label}
\PYG{n}{SampleLabelObject} \PYG{o}{=} \PYG{n}{TXLWizard}\PYG{o}{.}\PYG{n}{ShapeLibrary}\PYG{o}{.}\PYG{n}{Label}\PYG{o}{.}\PYG{n}{GetLabel}\PYG{p}{(}
    \PYG{n}{TXLWriter}\PYG{p}{,}
    \PYG{n}{Text}\PYG{o}{=}\PYG{n}{SampleParameters}\PYG{p}{[}\PYG{l+s+s1}{\PYGZsq{}}\PYG{l+s+s1}{Label}\PYG{l+s+s1}{\PYGZsq{}}\PYG{p}{]}\PYG{p}{,}
    \PYG{n}{OriginPoint}\PYG{o}{=}\PYG{p}{[}
        \PYG{l+m+mf}{0.5e3}\PYG{p}{,} \PYG{l+m+mf}{1.} \PYG{o}{*} \PYG{n}{SampleParameters}\PYG{p}{[}\PYG{l+s+s1}{\PYGZsq{}}\PYG{l+s+s1}{Height}\PYG{l+s+s1}{\PYGZsq{}}\PYG{p}{]} \PYG{o}{/} \PYG{l+m+mf}{2.} \PYG{o}{\PYGZhy{}} \PYG{l+m+mi}{500}
    \PYG{p}{]}\PYG{p}{,}
    \PYG{n}{FontSize}\PYG{o}{=}\PYG{l+m+mi}{150}\PYG{p}{,}
    \PYG{n}{StrokeWidth}\PYG{o}{=}\PYG{l+m+mi}{20}\PYG{p}{,}
    \PYG{n}{RoundCaps}\PYG{o}{=}\PYG{k+kc}{True}\PYG{p}{,} \PYG{c+c1}{\PYGZsh{} Set to False to improve e\PYGZhy{}Beam performance}
    \PYG{n}{Layer}\PYG{o}{=}\PYG{l+m+mi}{1}
\PYG{p}{)}


\PYG{c+c1}{\PYGZsh{}\PYGZsh{} Endpoint Detection \PYGZsh{}\PYGZsh{}}

\PYG{c+c1}{\PYGZsh{} Use Pre\PYGZhy{}Defined Endpoint Detection Windows}
\PYG{n}{TXLWizard}\PYG{o}{.}\PYG{n}{ShapeLibrary}\PYG{o}{.}\PYG{n}{EndpointDetectionWindows}\PYG{o}{.}\PYG{n}{GetEndpointDetectionWindows}\PYG{p}{(}
    \PYG{n}{TXLWriter}\PYG{p}{,} \PYG{n}{Layer}\PYG{o}{=}\PYG{l+m+mi}{1}\PYG{p}{)}

\PYG{c+c1}{\PYGZsh{}\PYGZsh{} User Structure: Circle \PYGZsh{}\PYGZsh{}}

\PYG{c+c1}{\PYGZsh{} Create Definition Structure for Circle that will be reused}
\PYG{n}{CircleStructure} \PYG{o}{=} \PYG{n}{TXLWriter}\PYG{o}{.}\PYG{n}{AddDefinitionStructure}\PYG{p}{(}\PYG{l+s+s1}{\PYGZsq{}}\PYG{l+s+s1}{MyCircleID}\PYG{l+s+s1}{\PYGZsq{}}\PYG{p}{)}
\PYG{n}{CircleStructure}\PYG{o}{.}\PYG{n}{AddPattern}\PYG{p}{(}
    \PYG{l+s+s1}{\PYGZsq{}}\PYG{l+s+s1}{Circle}\PYG{l+s+s1}{\PYGZsq{}}\PYG{p}{,}
    \PYG{n}{Center}\PYG{o}{=}\PYG{p}{[}\PYG{l+m+mi}{0}\PYG{p}{,} \PYG{l+m+mi}{0}\PYG{p}{]}\PYG{p}{,}
    \PYG{n}{Radius}\PYG{o}{=}\PYG{n}{StructureParameters}\PYG{p}{[}\PYG{l+s+s1}{\PYGZsq{}}\PYG{l+s+s1}{Circle}\PYG{l+s+s1}{\PYGZsq{}}\PYG{p}{]}\PYG{p}{[}\PYG{l+s+s1}{\PYGZsq{}}\PYG{l+s+s1}{Radius}\PYG{l+s+s1}{\PYGZsq{}}\PYG{p}{]}\PYG{p}{,}
    \PYG{n}{Layer}\PYG{o}{=}\PYG{n}{StructureParameters}\PYG{p}{[}\PYG{l+s+s1}{\PYGZsq{}}\PYG{l+s+s1}{Circle}\PYG{l+s+s1}{\PYGZsq{}}\PYG{p}{]}\PYG{p}{[}\PYG{l+s+s1}{\PYGZsq{}}\PYG{l+s+s1}{Layer}\PYG{l+s+s1}{\PYGZsq{}}\PYG{p}{]}

\PYG{p}{)}

\PYG{c+c1}{\PYGZsh{} Create array of the definition structure above}
\PYG{n}{CircleArray} \PYG{o}{=} \PYG{n}{TXLWriter}\PYG{o}{.}\PYG{n}{AddContentStructure}\PYG{p}{(}\PYG{l+s+s1}{\PYGZsq{}}\PYG{l+s+s1}{MyCircleArray}\PYG{l+s+s1}{\PYGZsq{}}\PYG{p}{)}
\PYG{n}{CircleArray}\PYG{o}{.}\PYG{n}{AddPattern}\PYG{p}{(}
    \PYG{l+s+s1}{\PYGZsq{}}\PYG{l+s+s1}{Array}\PYG{l+s+s1}{\PYGZsq{}}\PYG{p}{,}
    \PYG{n}{ReferencedStructureID}\PYG{o}{=}\PYG{n}{CircleStructure}\PYG{o}{.}\PYG{n}{ID}\PYG{p}{,}
    \PYG{n}{OriginPoint}\PYG{o}{=}\PYG{n}{StructureParameters}\PYG{p}{[}\PYG{l+s+s1}{\PYGZsq{}}\PYG{l+s+s1}{CircleArray}\PYG{l+s+s1}{\PYGZsq{}}\PYG{p}{]}\PYG{p}{[}\PYG{l+s+s1}{\PYGZsq{}}\PYG{l+s+s1}{ArrayOrigin}\PYG{l+s+s1}{\PYGZsq{}}\PYG{p}{]}\PYG{p}{,}
    \PYG{n}{PositionDelta1}\PYG{o}{=}\PYG{p}{[}
        \PYG{n}{StructureParameters}\PYG{p}{[}\PYG{l+s+s1}{\PYGZsq{}}\PYG{l+s+s1}{CircleArray}\PYG{l+s+s1}{\PYGZsq{}}\PYG{p}{]}\PYG{p}{[}\PYG{l+s+s1}{\PYGZsq{}}\PYG{l+s+s1}{ArrayXOffset}\PYG{l+s+s1}{\PYGZsq{}}\PYG{p}{]}\PYG{p}{,} \PYG{l+m+mi}{0}
    \PYG{p}{]}\PYG{p}{,}
    \PYG{n}{PositionDelta2}\PYG{o}{=}\PYG{p}{[}
        \PYG{l+m+mi}{0}\PYG{p}{,} \PYG{n}{StructureParameters}\PYG{p}{[}\PYG{l+s+s1}{\PYGZsq{}}\PYG{l+s+s1}{CircleArray}\PYG{l+s+s1}{\PYGZsq{}}\PYG{p}{]}\PYG{p}{[}\PYG{l+s+s1}{\PYGZsq{}}\PYG{l+s+s1}{ArrayYOffset}\PYG{l+s+s1}{\PYGZsq{}}\PYG{p}{]}
    \PYG{p}{]}\PYG{p}{,}
    \PYG{n}{Repetitions1}\PYG{o}{=}\PYG{n}{StructureParameters}\PYG{p}{[}\PYG{l+s+s1}{\PYGZsq{}}\PYG{l+s+s1}{CircleArray}\PYG{l+s+s1}{\PYGZsq{}}\PYG{p}{]}\PYG{p}{[}\PYG{l+s+s1}{\PYGZsq{}}\PYG{l+s+s1}{Columns}\PYG{l+s+s1}{\PYGZsq{}}\PYG{p}{]}\PYG{p}{,}
    \PYG{n}{Repetitions2}\PYG{o}{=}\PYG{n}{StructureParameters}\PYG{p}{[}\PYG{l+s+s1}{\PYGZsq{}}\PYG{l+s+s1}{CircleArray}\PYG{l+s+s1}{\PYGZsq{}}\PYG{p}{]}\PYG{p}{[}\PYG{l+s+s1}{\PYGZsq{}}\PYG{l+s+s1}{Rows}\PYG{l+s+s1}{\PYGZsq{}}\PYG{p}{]}
\PYG{p}{)}



\PYG{c+c1}{\PYGZsh{}\PYGZsh{}\PYGZsh{}\PYGZsh{}\PYGZsh{}\PYGZsh{}\PYGZsh{}\PYGZsh{}\PYGZsh{}\PYGZsh{}\PYGZsh{}\PYGZsh{}\PYGZsh{}\PYGZsh{}\PYGZsh{}\PYGZsh{}\PYGZsh{}\PYGZsh{}\PYGZsh{}\PYGZsh{}\PYGZsh{}\PYGZsh{}\PYGZsh{}\PYGZsh{}\PYGZsh{}}
\PYG{c+c1}{\PYGZsh{} Generate Output Files \PYGZsh{}}
\PYG{c+c1}{\PYGZsh{}\PYGZsh{}\PYGZsh{}\PYGZsh{}\PYGZsh{}\PYGZsh{}\PYGZsh{}\PYGZsh{}\PYGZsh{}\PYGZsh{}\PYGZsh{}\PYGZsh{}\PYGZsh{}\PYGZsh{}\PYGZsh{}\PYGZsh{}\PYGZsh{}\PYGZsh{}\PYGZsh{}\PYGZsh{}\PYGZsh{}\PYGZsh{}\PYGZsh{}\PYGZsh{}\PYGZsh{}}

\PYG{c+c1}{\PYGZsh{} Note: The suffix (.txl, .html, .svg) will be appended automatically}
\PYG{n}{TXLWriter}\PYG{o}{.}\PYG{n}{GenerateFiles}\PYG{p}{(}\PYG{l+s+s1}{\PYGZsq{}}\PYG{l+s+s1}{Masks/Example\PYGZus{}Simple}\PYG{l+s+s1}{\PYGZsq{}}\PYG{p}{)}

\end{Verbatim}


\subsubsection{Generated SVG Image}
\label{Chapters/20_Examples:id3}\label{Chapters/20_Examples:examplesimplesvg}\begin{figure}[htbp]
\centering
\capstart

\includegraphics{{Example_Simple}.png}
\caption{Generated SVG Image for \titleref{Content/Example\_Simple.py}}\label{Chapters/20_Examples:id8}\end{figure}


\subsection{Advanced Example}
\label{Chapters/20_Examples:advanced-example}\label{Chapters/20_Examples:txlwizardexampleadvanced}

\subsubsection{Introduction}
\label{Chapters/20_Examples:id4}
The following code demonstrates an advanced example usage of the \titleref{TXLWizard} for
generating TXL files with python code.

The code can be found in the file \code{Content/Example\_Advanced.py}.
The resulting SVG image is shown in Figure {\hyperref[Chapters/20_Examples:exampleadvancedsvg]{\crossref{\DUrole{std,std-ref}{Generated SVG Image}}}}.


\subsubsection{Code}
\label{Chapters/20_Examples:id5}
\begin{Verbatim}[commandchars=\\\{\},numbers=left,firstnumber=1,stepnumber=1]
\PYG{c+c1}{\PYGZsh{}\PYGZsh{}\PYGZsh{}\PYGZsh{}\PYGZsh{}\PYGZsh{}\PYGZsh{}\PYGZsh{}\PYGZsh{}\PYGZsh{}\PYGZsh{}\PYGZsh{}\PYGZsh{}\PYGZsh{}\PYGZsh{}\PYGZsh{}\PYGZsh{}\PYGZsh{}\PYGZsh{}\PYGZsh{}}
\PYG{c+c1}{\PYGZsh{} Import Libraries \PYGZsh{}}
\PYG{c+c1}{\PYGZsh{}\PYGZsh{}\PYGZsh{}\PYGZsh{}\PYGZsh{}\PYGZsh{}\PYGZsh{}\PYGZsh{}\PYGZsh{}\PYGZsh{}\PYGZsh{}\PYGZsh{}\PYGZsh{}\PYGZsh{}\PYGZsh{}\PYGZsh{}\PYGZsh{}\PYGZsh{}\PYGZsh{}\PYGZsh{}}

\PYG{c+c1}{\PYGZsh{} Import TXLWriter, the main class for generating TXL Output}
\PYG{k+kn}{import} \PYG{n+nn}{TXLWizard}\PYG{n+nn}{.}\PYG{n+nn}{TXLWriter}

\PYG{c+c1}{\PYGZsh{} Import Pre\PYGZhy{}Defined Shapes / Structures wrapped in functions}
\PYG{k+kn}{import} \PYG{n+nn}{TXLWizard}\PYG{n+nn}{.}\PYG{n+nn}{ShapeLibrary}\PYG{n+nn}{.}\PYG{n+nn}{EndpointDetectionWindows}
\PYG{k+kn}{import} \PYG{n+nn}{TXLWizard}\PYG{n+nn}{.}\PYG{n+nn}{ShapeLibrary}\PYG{n+nn}{.}\PYG{n+nn}{Markers}
\PYG{k+kn}{import} \PYG{n+nn}{TXLWizard}\PYG{n+nn}{.}\PYG{n+nn}{ShapeLibrary}\PYG{n+nn}{.}\PYG{n+nn}{Label}
\PYG{k+kn}{import} \PYG{n+nn}{TXLWizard}\PYG{n+nn}{.}\PYG{n+nn}{ShapeLibrary}\PYG{n+nn}{.}\PYG{n+nn}{CornerCube}

\PYG{c+c1}{\PYGZsh{} Import math module for calculations}
\PYG{k+kn}{import} \PYG{n+nn}{math}


\PYG{c+c1}{\PYGZsh{}\PYGZsh{}\PYGZsh{}\PYGZsh{}\PYGZsh{}\PYGZsh{}\PYGZsh{}\PYGZsh{}\PYGZsh{}\PYGZsh{}\PYGZsh{}\PYGZsh{}\PYGZsh{}\PYGZsh{}\PYGZsh{}\PYGZsh{}\PYGZsh{}\PYGZsh{}\PYGZsh{}\PYGZsh{}\PYGZsh{}\PYGZsh{}\PYGZsh{}\PYGZsh{}\PYGZsh{}\PYGZsh{}\PYGZsh{}\PYGZsh{}\PYGZsh{}\PYGZsh{}\PYGZsh{}\PYGZsh{}\PYGZsh{}}
\PYG{c+c1}{\PYGZsh{} Sample / Structure Parameters \PYGZsh{}}
\PYG{c+c1}{\PYGZsh{}\PYGZsh{}\PYGZsh{}\PYGZsh{}\PYGZsh{}\PYGZsh{}\PYGZsh{}\PYGZsh{}\PYGZsh{}\PYGZsh{}\PYGZsh{}\PYGZsh{}\PYGZsh{}\PYGZsh{}\PYGZsh{}\PYGZsh{}\PYGZsh{}\PYGZsh{}\PYGZsh{}\PYGZsh{}\PYGZsh{}\PYGZsh{}\PYGZsh{}\PYGZsh{}\PYGZsh{}\PYGZsh{}\PYGZsh{}\PYGZsh{}\PYGZsh{}\PYGZsh{}\PYGZsh{}\PYGZsh{}\PYGZsh{}}

\PYG{c+c1}{\PYGZsh{} Define all sample parameters}
\PYG{n}{SampleParameters} \PYG{o}{=} \PYG{p}{\PYGZob{}}
    \PYG{l+s+s1}{\PYGZsq{}}\PYG{l+s+s1}{Width}\PYG{l+s+s1}{\PYGZsq{}}\PYG{p}{:} \PYG{l+m+mi}{8}\PYG{n}{e3}\PYG{p}{,}
    \PYG{l+s+s1}{\PYGZsq{}}\PYG{l+s+s1}{Height}\PYG{l+s+s1}{\PYGZsq{}}\PYG{p}{:} \PYG{l+m+mi}{8}\PYG{n}{e3}\PYG{p}{,}
    \PYG{l+s+s1}{\PYGZsq{}}\PYG{l+s+s1}{Label}\PYG{l+s+s1}{\PYGZsq{}}\PYG{p}{:} \PYG{l+s+s1}{\PYGZsq{}}\PYG{l+s+s1}{GOI Demo CornerCube}\PYG{l+s+s1}{\PYGZsq{}}\PYG{p}{,}
\PYG{p}{\PYGZcb{}}

\PYG{c+c1}{\PYGZsh{} Define all structure parameters}
\PYG{n}{StructureParameters} \PYG{o}{=} \PYG{p}{\PYGZob{}}
    \PYG{l+s+s1}{\PYGZsq{}}\PYG{l+s+s1}{CornerCube}\PYG{l+s+s1}{\PYGZsq{}}\PYG{p}{:} \PYG{p}{\PYGZob{}}
        \PYG{l+s+s1}{\PYGZsq{}}\PYG{l+s+s1}{BridgeLength}\PYG{l+s+s1}{\PYGZsq{}}\PYG{p}{:}\PYG{l+m+mi}{8}\PYG{p}{,}
        \PYG{l+s+s1}{\PYGZsq{}}\PYG{l+s+s1}{ParabolaFocus}\PYG{l+s+s1}{\PYGZsq{}}\PYG{p}{:} \PYG{l+m+mi}{9}\PYG{p}{,}
        \PYG{l+s+s1}{\PYGZsq{}}\PYG{l+s+s1}{XCutoff}\PYG{l+s+s1}{\PYGZsq{}}\PYG{p}{:} \PYG{l+m+mi}{9}\PYG{p}{,}
        \PYG{l+s+s1}{\PYGZsq{}}\PYG{l+s+s1}{AirGapX}\PYG{l+s+s1}{\PYGZsq{}}\PYG{p}{:} \PYG{l+m+mi}{3}\PYG{p}{,}
        \PYG{l+s+s1}{\PYGZsq{}}\PYG{l+s+s1}{AirGapY}\PYG{l+s+s1}{\PYGZsq{}}\PYG{p}{:} \PYG{l+m+mi}{1}\PYG{p}{,}
        \PYG{l+s+s1}{\PYGZsq{}}\PYG{l+s+s1}{LabelXOffset}\PYG{l+s+s1}{\PYGZsq{}}\PYG{p}{:} \PYG{l+m+mi}{0}\PYG{p}{,}
        \PYG{l+s+s1}{\PYGZsq{}}\PYG{l+s+s1}{LabelYOffset}\PYG{l+s+s1}{\PYGZsq{}}\PYG{p}{:} \PYG{l+m+mi}{50}\PYG{p}{,}
        \PYG{l+s+s1}{\PYGZsq{}}\PYG{l+s+s1}{Label}\PYG{l+s+s1}{\PYGZsq{}}\PYG{p}{:} \PYG{l+s+s1}{\PYGZsq{}}\PYG{l+s+s1}{R}\PYG{l+s+si}{\PYGZob{}:d\PYGZcb{}}\PYG{l+s+s1}{C}\PYG{l+s+si}{\PYGZob{}:d\PYGZcb{}}\PYG{l+s+s1}{\PYGZsq{}}\PYG{p}{,} \PYG{c+c1}{\PYGZsh{} \PYGZob{}:d\PYGZcb{} will be replaced}
                               \PYG{c+c1}{\PYGZsh{} by str.format() with the corresponding row / column}
        \PYG{l+s+s1}{\PYGZsq{}}\PYG{l+s+s1}{Layer}\PYG{l+s+s1}{\PYGZsq{}}\PYG{p}{:} \PYG{l+m+mi}{2}
    \PYG{p}{\PYGZcb{}}\PYG{p}{,}
    \PYG{l+s+s1}{\PYGZsq{}}\PYG{l+s+s1}{Circle}\PYG{l+s+s1}{\PYGZsq{}}\PYG{p}{:} \PYG{p}{\PYGZob{}}
        \PYG{l+s+s1}{\PYGZsq{}}\PYG{l+s+s1}{Radius}\PYG{l+s+s1}{\PYGZsq{}}\PYG{p}{:} \PYG{l+m+mi}{5}\PYG{p}{,}
        \PYG{l+s+s1}{\PYGZsq{}}\PYG{l+s+s1}{Layer}\PYG{l+s+s1}{\PYGZsq{}}\PYG{p}{:} \PYG{l+m+mi}{3}
    \PYG{p}{\PYGZcb{}}\PYG{p}{,}
    \PYG{l+s+s1}{\PYGZsq{}}\PYG{l+s+s1}{CornerCubeArray}\PYG{l+s+s1}{\PYGZsq{}}\PYG{p}{:} \PYG{p}{\PYGZob{}}
        \PYG{l+s+s1}{\PYGZsq{}}\PYG{l+s+s1}{Columns}\PYG{l+s+s1}{\PYGZsq{}}\PYG{p}{:} \PYG{l+m+mi}{6}\PYG{p}{,}
        \PYG{l+s+s1}{\PYGZsq{}}\PYG{l+s+s1}{Rows}\PYG{l+s+s1}{\PYGZsq{}}\PYG{p}{:} \PYG{l+m+mi}{5}\PYG{p}{,}
        \PYG{l+s+s1}{\PYGZsq{}}\PYG{l+s+s1}{ArrayXOffset}\PYG{l+s+s1}{\PYGZsq{}}\PYG{p}{:} \PYG{l+m+mi}{500}\PYG{p}{,}
        \PYG{l+s+s1}{\PYGZsq{}}\PYG{l+s+s1}{ArrayYOffset}\PYG{l+s+s1}{\PYGZsq{}}\PYG{p}{:} \PYG{o}{\PYGZhy{}}\PYG{l+m+mi}{500}\PYG{p}{,}
        \PYG{l+s+s1}{\PYGZsq{}}\PYG{l+s+s1}{ArrayOrigin}\PYG{l+s+s1}{\PYGZsq{}}\PYG{p}{:} \PYG{p}{[}\PYG{l+m+mf}{0.75e3}\PYG{p}{,} \PYG{l+m+mi}{3}\PYG{n}{e3}\PYG{p}{]}
    \PYG{p}{\PYGZcb{}}
\PYG{p}{\PYGZcb{}}


\PYG{c+c1}{\PYGZsh{}\PYGZsh{}\PYGZsh{}\PYGZsh{}\PYGZsh{}\PYGZsh{}\PYGZsh{}\PYGZsh{}\PYGZsh{}\PYGZsh{}\PYGZsh{}\PYGZsh{}\PYGZsh{}\PYGZsh{}\PYGZsh{}\PYGZsh{}\PYGZsh{}\PYGZsh{}\PYGZsh{}\PYGZsh{}\PYGZsh{}\PYGZsh{}\PYGZsh{}\PYGZsh{}}
\PYG{c+c1}{\PYGZsh{} Initialize TXLWriter \PYGZsh{}}
\PYG{c+c1}{\PYGZsh{}\PYGZsh{}\PYGZsh{}\PYGZsh{}\PYGZsh{}\PYGZsh{}\PYGZsh{}\PYGZsh{}\PYGZsh{}\PYGZsh{}\PYGZsh{}\PYGZsh{}\PYGZsh{}\PYGZsh{}\PYGZsh{}\PYGZsh{}\PYGZsh{}\PYGZsh{}\PYGZsh{}\PYGZsh{}\PYGZsh{}\PYGZsh{}\PYGZsh{}\PYGZsh{}}
\PYG{n}{TXLWriter} \PYG{o}{=} \PYG{n}{TXLWizard}\PYG{o}{.}\PYG{n}{TXLWriter}\PYG{o}{.}\PYG{n}{TXLWriter}\PYG{p}{(}
    \PYG{n}{GridWidth}\PYG{o}{=}\PYG{n}{SampleParameters}\PYG{p}{[}\PYG{l+s+s1}{\PYGZsq{}}\PYG{l+s+s1}{Width}\PYG{l+s+s1}{\PYGZsq{}}\PYG{p}{]}\PYG{p}{,}
    \PYG{n}{GridHeight}\PYG{o}{=}\PYG{n}{SampleParameters}\PYG{p}{[}\PYG{l+s+s1}{\PYGZsq{}}\PYG{l+s+s1}{Height}\PYG{l+s+s1}{\PYGZsq{}}\PYG{p}{]}
\PYG{p}{)}

\PYG{c+c1}{\PYGZsh{}\PYGZsh{}\PYGZsh{}\PYGZsh{}\PYGZsh{}\PYGZsh{}\PYGZsh{}\PYGZsh{}\PYGZsh{}\PYGZsh{}\PYGZsh{}\PYGZsh{}\PYGZsh{}\PYGZsh{}\PYGZsh{}\PYGZsh{}\PYGZsh{}\PYGZsh{}\PYGZsh{}\PYGZsh{}\PYGZsh{}}
\PYG{c+c1}{\PYGZsh{} Define Structures \PYGZsh{}}
\PYG{c+c1}{\PYGZsh{}\PYGZsh{}\PYGZsh{}\PYGZsh{}\PYGZsh{}\PYGZsh{}\PYGZsh{}\PYGZsh{}\PYGZsh{}\PYGZsh{}\PYGZsh{}\PYGZsh{}\PYGZsh{}\PYGZsh{}\PYGZsh{}\PYGZsh{}\PYGZsh{}\PYGZsh{}\PYGZsh{}\PYGZsh{}\PYGZsh{}}

\PYG{c+c1}{\PYGZsh{}\PYGZsh{} Sample Label \PYGZsh{}\PYGZsh{}}

\PYG{c+c1}{\PYGZsh{} Give the sample a nice label...}
\PYG{n}{SampleLabelObject} \PYG{o}{=} \PYG{n}{TXLWizard}\PYG{o}{.}\PYG{n}{ShapeLibrary}\PYG{o}{.}\PYG{n}{Label}\PYG{o}{.}\PYG{n}{GetLabel}\PYG{p}{(}
    \PYG{n}{TXLWriter}\PYG{p}{,}
    \PYG{n}{Text}\PYG{o}{=}\PYG{n}{SampleParameters}\PYG{p}{[}\PYG{l+s+s1}{\PYGZsq{}}\PYG{l+s+s1}{Label}\PYG{l+s+s1}{\PYGZsq{}}\PYG{p}{]}\PYG{p}{,}
    \PYG{n}{OriginPoint}\PYG{o}{=}\PYG{p}{[}
        \PYG{l+m+mf}{0.5e3}\PYG{p}{,} \PYG{l+m+mf}{1.} \PYG{o}{*} \PYG{n}{SampleParameters}\PYG{p}{[}\PYG{l+s+s1}{\PYGZsq{}}\PYG{l+s+s1}{Height}\PYG{l+s+s1}{\PYGZsq{}}\PYG{p}{]} \PYG{o}{/} \PYG{l+m+mf}{2.} \PYG{o}{\PYGZhy{}} \PYG{l+m+mi}{500}
    \PYG{p}{]}\PYG{p}{,}
    \PYG{n}{FontSize}\PYG{o}{=}\PYG{l+m+mi}{150}\PYG{p}{,}
    \PYG{n}{StrokeWidth}\PYG{o}{=}\PYG{l+m+mi}{20}\PYG{p}{,}
    \PYG{n}{RoundCaps}\PYG{o}{=}\PYG{k+kc}{True}\PYG{p}{,}\PYG{c+c1}{\PYGZsh{} Set to False to improve e\PYGZhy{}Beam performance}
    \PYG{n}{Layer}\PYG{o}{=}\PYG{l+m+mi}{1}
\PYG{p}{)}
\PYG{c+c1}{\PYGZsh{} ...and some other information}
\PYG{n}{Alphabet} \PYG{o}{=} \PYG{n}{TXLWizard}\PYG{o}{.}\PYG{n}{ShapeLibrary}\PYG{o}{.}\PYG{n}{Label}\PYG{o}{.}\PYG{n}{GetLabel}\PYG{p}{(}
    \PYG{n}{TXLWriter}\PYG{p}{,}
    \PYG{n}{Text}\PYG{o}{=}\PYG{l+s+s1}{\PYGZsq{}}\PYG{l+s+s1}{abcdefghijklmnopqrstuvwxyz0123456789 megamega ggg ah extraaaa rischaaaar}\PYG{l+s+s1}{\PYGZsq{}}\PYG{p}{,}
    \PYG{n}{OriginPoint}\PYG{o}{=}\PYG{p}{[}
        \PYG{l+m+mf}{0.5e3}\PYG{p}{,} \PYG{l+m+mf}{1.} \PYG{o}{*} \PYG{n}{SampleParameters}\PYG{p}{[}\PYG{l+s+s1}{\PYGZsq{}}\PYG{l+s+s1}{Height}\PYG{l+s+s1}{\PYGZsq{}}\PYG{p}{]} \PYG{o}{/} \PYG{l+m+mf}{2.} \PYG{o}{\PYGZhy{}} \PYG{l+m+mi}{600}
    \PYG{p}{]}\PYG{p}{,}
    \PYG{n}{FontSize}\PYG{o}{=}\PYG{l+m+mi}{50}\PYG{p}{,}
    \PYG{n}{StrokeWidth}\PYG{o}{=}\PYG{l+m+mi}{3}\PYG{p}{,}
    \PYG{n}{RoundCaps}\PYG{o}{=}\PYG{k+kc}{True}\PYG{p}{,} \PYG{c+c1}{\PYGZsh{} Set to False to improve e\PYGZhy{}Beam performance}
    \PYG{n}{Layer}\PYG{o}{=}\PYG{l+m+mi}{1}
\PYG{p}{)}

\PYG{c+c1}{\PYGZsh{}\PYGZsh{} Endpoint Detection \PYGZsh{}\PYGZsh{}}

\PYG{c+c1}{\PYGZsh{} Use Pre\PYGZhy{}Defined Endpoint Detection Windows}
\PYG{n}{TXLWizard}\PYG{o}{.}\PYG{n}{ShapeLibrary}\PYG{o}{.}\PYG{n}{EndpointDetectionWindows}\PYG{o}{.}\PYG{n}{GetEndpointDetectionWindows}\PYG{p}{(}
    \PYG{n}{TXLWriter}\PYG{p}{,} \PYG{n}{Layer}\PYG{o}{=}\PYG{l+m+mi}{1}
\PYG{p}{)}

\PYG{c+c1}{\PYGZsh{}\PYGZsh{} Alignment Markers \PYGZsh{}\PYGZsh{}}

\PYG{c+c1}{\PYGZsh{} Use Pre\PYGZhy{}Defined Alignment Markers}
\PYG{n}{TXLWizard}\PYG{o}{.}\PYG{n}{ShapeLibrary}\PYG{o}{.}\PYG{n}{AlignmentMarkers}\PYG{o}{.}\PYG{n}{GetAlignmentMarkers}\PYG{p}{(}
    \PYG{n}{TXLWriter}\PYG{p}{,} \PYG{n}{Layer}\PYG{o}{=}\PYG{l+m+mi}{1}
\PYG{p}{)}


\PYG{c+c1}{\PYGZsh{}\PYGZsh{} User Structure: Corner Cube \PYGZsh{}\PYGZsh{}}

\PYG{c+c1}{\PYGZsh{} Create Definition Structure for Corner Cube that will be reused}
\PYG{n}{CornerCubeDefinition} \PYG{o}{=} \PYG{n}{TXLWizard}\PYG{o}{.}\PYG{n}{ShapeLibrary}\PYG{o}{.}\PYG{n}{CornerCube}\PYG{o}{.}\PYG{n}{GetCornerCube}\PYG{p}{(}
    \PYG{n}{TXLWriter}\PYG{p}{,}
    \PYG{n}{ParabolaFocus}\PYG{o}{=}\PYG{n}{StructureParameters}\PYG{p}{[}\PYG{l+s+s1}{\PYGZsq{}}\PYG{l+s+s1}{CornerCube}\PYG{l+s+s1}{\PYGZsq{}}\PYG{p}{]}\PYG{p}{[}\PYG{l+s+s1}{\PYGZsq{}}\PYG{l+s+s1}{ParabolaFocus}\PYG{l+s+s1}{\PYGZsq{}}\PYG{p}{]}\PYG{p}{,}
    \PYG{n}{XCutoff}\PYG{o}{=}\PYG{n}{StructureParameters}\PYG{p}{[}\PYG{l+s+s1}{\PYGZsq{}}\PYG{l+s+s1}{CornerCube}\PYG{l+s+s1}{\PYGZsq{}}\PYG{p}{]}\PYG{p}{[}\PYG{l+s+s1}{\PYGZsq{}}\PYG{l+s+s1}{XCutoff}\PYG{l+s+s1}{\PYGZsq{}}\PYG{p}{]}\PYG{p}{,}
    \PYG{n}{AirGapX}\PYG{o}{=}\PYG{n}{StructureParameters}\PYG{p}{[}\PYG{l+s+s1}{\PYGZsq{}}\PYG{l+s+s1}{CornerCube}\PYG{l+s+s1}{\PYGZsq{}}\PYG{p}{]}\PYG{p}{[}\PYG{l+s+s1}{\PYGZsq{}}\PYG{l+s+s1}{AirGapX}\PYG{l+s+s1}{\PYGZsq{}}\PYG{p}{]}\PYG{p}{,}
    \PYG{n}{AirGapY}\PYG{o}{=}\PYG{n}{StructureParameters}\PYG{p}{[}\PYG{l+s+s1}{\PYGZsq{}}\PYG{l+s+s1}{CornerCube}\PYG{l+s+s1}{\PYGZsq{}}\PYG{p}{]}\PYG{p}{[}\PYG{l+s+s1}{\PYGZsq{}}\PYG{l+s+s1}{AirGapY}\PYG{l+s+s1}{\PYGZsq{}}\PYG{p}{]}\PYG{p}{,}
    \PYG{n}{Layer}\PYG{o}{=}\PYG{n}{StructureParameters}\PYG{p}{[}\PYG{l+s+s1}{\PYGZsq{}}\PYG{l+s+s1}{CornerCube}\PYG{l+s+s1}{\PYGZsq{}}\PYG{p}{]}\PYG{p}{[}\PYG{l+s+s1}{\PYGZsq{}}\PYG{l+s+s1}{Layer}\PYG{l+s+s1}{\PYGZsq{}}\PYG{p}{]}
\PYG{p}{)}

\PYG{c+c1}{\PYGZsh{} Create Definition Structure for combination of cornercube and additional circle}
\PYG{n}{FullCornerCubeNoRotation} \PYG{o}{=} \PYG{n}{TXLWriter}\PYG{o}{.}\PYG{n}{AddDefinitionStructure}\PYG{p}{(}\PYG{l+s+s1}{\PYGZsq{}}\PYG{l+s+s1}{FullCornerCubeNoRotation}\PYG{l+s+s1}{\PYGZsq{}}\PYG{p}{)}
\PYG{n}{FullCornerCubeNoRotation}\PYG{o}{.}\PYG{n}{AddPattern}\PYG{p}{(}
    \PYG{l+s+s1}{\PYGZsq{}}\PYG{l+s+s1}{Reference}\PYG{l+s+s1}{\PYGZsq{}}\PYG{p}{,}
    \PYG{n}{ReferencedStructureID}\PYG{o}{=}\PYG{n}{CornerCubeDefinition}\PYG{o}{.}\PYG{n}{ID}\PYG{p}{,}
    \PYG{n}{OriginPoint}\PYG{o}{=}\PYG{p}{[}\PYG{l+m+mf}{1.} \PYG{o}{*} \PYG{n}{StructureParameters}\PYG{p}{[}\PYG{l+s+s1}{\PYGZsq{}}\PYG{l+s+s1}{CornerCube}\PYG{l+s+s1}{\PYGZsq{}}\PYG{p}{]}\PYG{p}{[}\PYG{l+s+s1}{\PYGZsq{}}\PYG{l+s+s1}{BridgeLength}\PYG{l+s+s1}{\PYGZsq{}}\PYG{p}{]} \PYG{o}{/} \PYG{l+m+mf}{2.}\PYG{p}{,} \PYG{l+m+mi}{0}\PYG{p}{]}
\PYG{p}{)}
\PYG{n}{FullCornerCubeNoRotation}\PYG{o}{.}\PYG{n}{AddPattern}\PYG{p}{(}
    \PYG{l+s+s1}{\PYGZsq{}}\PYG{l+s+s1}{Circle}\PYG{l+s+s1}{\PYGZsq{}}\PYG{p}{,}
    \PYG{n}{Center}\PYG{o}{=}\PYG{p}{[}\PYG{l+m+mi}{0}\PYG{p}{,} \PYG{l+m+mi}{0}\PYG{p}{]}\PYG{p}{,}
    \PYG{n}{Radius}\PYG{o}{=}\PYG{n}{StructureParameters}\PYG{p}{[}\PYG{l+s+s1}{\PYGZsq{}}\PYG{l+s+s1}{Circle}\PYG{l+s+s1}{\PYGZsq{}}\PYG{p}{]}\PYG{p}{[}\PYG{l+s+s1}{\PYGZsq{}}\PYG{l+s+s1}{Radius}\PYG{l+s+s1}{\PYGZsq{}}\PYG{p}{]}\PYG{p}{,}
    \PYG{n}{Layer}\PYG{o}{=}\PYG{n}{StructureParameters}\PYG{p}{[}\PYG{l+s+s1}{\PYGZsq{}}\PYG{l+s+s1}{Circle}\PYG{l+s+s1}{\PYGZsq{}}\PYG{p}{]}\PYG{p}{[}\PYG{l+s+s1}{\PYGZsq{}}\PYG{l+s+s1}{Layer}\PYG{l+s+s1}{\PYGZsq{}}\PYG{p}{]}
\PYG{p}{)}

\PYG{c+c1}{\PYGZsh{} Create definition structure with rotation of entire referenced structure}
\PYG{n}{FullCornerCube} \PYG{o}{=} \PYG{n}{TXLWriter}\PYG{o}{.}\PYG{n}{AddDefinitionStructure}\PYG{p}{(}\PYG{l+s+s1}{\PYGZsq{}}\PYG{l+s+s1}{FullCornerCube}\PYG{l+s+s1}{\PYGZsq{}}\PYG{p}{,}
                                                  \PYG{n}{RotationAngle}\PYG{o}{=}\PYG{l+m+mi}{45}\PYG{p}{)}
\PYG{n}{FullCornerCube}\PYG{o}{.}\PYG{n}{AddPattern}\PYG{p}{(}
    \PYG{l+s+s1}{\PYGZsq{}}\PYG{l+s+s1}{Reference}\PYG{l+s+s1}{\PYGZsq{}}\PYG{p}{,}
    \PYG{n}{ReferencedStructureID}\PYG{o}{=}\PYG{n}{FullCornerCubeNoRotation}\PYG{o}{.}\PYG{n}{ID}\PYG{p}{,}
    \PYG{n}{OriginPoint}\PYG{o}{=}\PYG{p}{[}\PYG{l+m+mi}{0}\PYG{p}{,} \PYG{l+m+mi}{0}\PYG{p}{]}
\PYG{p}{)}

\PYG{c+c1}{\PYGZsh{} Create array of the definition structure above}
\PYG{n}{CornerCubeArrayFine} \PYG{o}{=} \PYG{n}{TXLWriter}\PYG{o}{.}\PYG{n}{AddContentStructure}\PYG{p}{(}\PYG{l+s+s1}{\PYGZsq{}}\PYG{l+s+s1}{CornerCubeArrayFine}\PYG{l+s+s1}{\PYGZsq{}}\PYG{p}{)}
\PYG{n}{CornerCubeArrayFine}\PYG{o}{.}\PYG{n}{AddPattern}\PYG{p}{(}
    \PYG{l+s+s1}{\PYGZsq{}}\PYG{l+s+s1}{Array}\PYG{l+s+s1}{\PYGZsq{}}\PYG{p}{,}
    \PYG{n}{ReferencedStructureID}\PYG{o}{=}\PYG{n}{FullCornerCube}\PYG{o}{.}\PYG{n}{ID}\PYG{p}{,}
    \PYG{n}{OriginPoint}\PYG{o}{=}\PYG{n}{StructureParameters}\PYG{p}{[}\PYG{l+s+s1}{\PYGZsq{}}\PYG{l+s+s1}{CornerCubeArray}\PYG{l+s+s1}{\PYGZsq{}}\PYG{p}{]}\PYG{p}{[}\PYG{l+s+s1}{\PYGZsq{}}\PYG{l+s+s1}{ArrayOrigin}\PYG{l+s+s1}{\PYGZsq{}}\PYG{p}{]}\PYG{p}{,}
    \PYG{n}{PositionDelta1}\PYG{o}{=}\PYG{p}{[}
        \PYG{n}{StructureParameters}\PYG{p}{[}\PYG{l+s+s1}{\PYGZsq{}}\PYG{l+s+s1}{CornerCubeArray}\PYG{l+s+s1}{\PYGZsq{}}\PYG{p}{]}\PYG{p}{[}\PYG{l+s+s1}{\PYGZsq{}}\PYG{l+s+s1}{ArrayXOffset}\PYG{l+s+s1}{\PYGZsq{}}\PYG{p}{]}\PYG{p}{,} \PYG{l+m+mi}{0}
    \PYG{p}{]}\PYG{p}{,}
    \PYG{n}{PositionDelta2}\PYG{o}{=}\PYG{p}{[}
        \PYG{l+m+mi}{0}\PYG{p}{,} \PYG{n}{StructureParameters}\PYG{p}{[}\PYG{l+s+s1}{\PYGZsq{}}\PYG{l+s+s1}{CornerCubeArray}\PYG{l+s+s1}{\PYGZsq{}}\PYG{p}{]}\PYG{p}{[}\PYG{l+s+s1}{\PYGZsq{}}\PYG{l+s+s1}{ArrayYOffset}\PYG{l+s+s1}{\PYGZsq{}}\PYG{p}{]}
    \PYG{p}{]}\PYG{p}{,}
    \PYG{n}{Repetitions1}\PYG{o}{=}\PYG{n}{StructureParameters}\PYG{p}{[}\PYG{l+s+s1}{\PYGZsq{}}\PYG{l+s+s1}{CornerCubeArray}\PYG{l+s+s1}{\PYGZsq{}}\PYG{p}{]}\PYG{p}{[}\PYG{l+s+s1}{\PYGZsq{}}\PYG{l+s+s1}{Columns}\PYG{l+s+s1}{\PYGZsq{}}\PYG{p}{]}\PYG{p}{,}
    \PYG{n}{Repetitions2}\PYG{o}{=}\PYG{n}{StructureParameters}\PYG{p}{[}\PYG{l+s+s1}{\PYGZsq{}}\PYG{l+s+s1}{CornerCubeArray}\PYG{l+s+s1}{\PYGZsq{}}\PYG{p}{]}\PYG{p}{[}\PYG{l+s+s1}{\PYGZsq{}}\PYG{l+s+s1}{Rows}\PYG{l+s+s1}{\PYGZsq{}}\PYG{p}{]}
\PYG{p}{)}


\PYG{c+c1}{\PYGZsh{} Add Labels to each array element}
\PYG{k}{for} \PYG{n}{Row} \PYG{o+ow}{in} \PYG{n+nb}{range}\PYG{p}{(}\PYG{l+m+mi}{1}\PYG{p}{,} \PYG{n}{StructureParameters}\PYG{p}{[}\PYG{l+s+s1}{\PYGZsq{}}\PYG{l+s+s1}{CornerCubeArray}\PYG{l+s+s1}{\PYGZsq{}}\PYG{p}{]}\PYG{p}{[}\PYG{l+s+s1}{\PYGZsq{}}\PYG{l+s+s1}{Rows}\PYG{l+s+s1}{\PYGZsq{}}\PYG{p}{]} \PYG{o}{+} \PYG{l+m+mi}{1}\PYG{p}{)}\PYG{p}{:}
    \PYG{k}{for} \PYG{n}{Column} \PYG{o+ow}{in} \PYG{n+nb}{range}\PYG{p}{(}\PYG{l+m+mi}{1}\PYG{p}{,} \PYG{n}{StructureParameters}\PYG{p}{[}\PYG{l+s+s1}{\PYGZsq{}}\PYG{l+s+s1}{CornerCubeArray}\PYG{l+s+s1}{\PYGZsq{}}\PYG{p}{]}\PYG{p}{[}\PYG{l+s+s1}{\PYGZsq{}}\PYG{l+s+s1}{Columns}\PYG{l+s+s1}{\PYGZsq{}}\PYG{p}{]} \PYG{o}{+} \PYG{l+m+mi}{1}\PYG{p}{)}\PYG{p}{:}
        \PYG{n}{RowColumnCountLabel} \PYG{o}{=} \PYG{n}{TXLWizard}\PYG{o}{.}\PYG{n}{ShapeLibrary}\PYG{o}{.}\PYG{n}{Label}\PYG{o}{.}\PYG{n}{GetLabel}\PYG{p}{(}
            \PYG{n}{TXLWriter}\PYG{p}{,}
            \PYG{n}{StructureParameters}\PYG{p}{[}\PYG{l+s+s1}{\PYGZsq{}}\PYG{l+s+s1}{CornerCube}\PYG{l+s+s1}{\PYGZsq{}}\PYG{p}{]}\PYG{p}{[}\PYG{l+s+s1}{\PYGZsq{}}\PYG{l+s+s1}{Label}\PYG{l+s+s1}{\PYGZsq{}}\PYG{p}{]}\PYG{o}{.}\PYG{n}{format}\PYG{p}{(}\PYG{n}{Row}\PYG{p}{,} \PYG{n}{Column}\PYG{p}{)}\PYG{p}{,}
            \PYG{n}{OriginPoint}\PYG{o}{=}\PYG{p}{[}
                \PYG{n}{StructureParameters}\PYG{p}{[}\PYG{l+s+s1}{\PYGZsq{}}\PYG{l+s+s1}{CornerCubeArray}\PYG{l+s+s1}{\PYGZsq{}}\PYG{p}{]}\PYG{p}{[}\PYG{l+s+s1}{\PYGZsq{}}\PYG{l+s+s1}{ArrayOrigin}\PYG{l+s+s1}{\PYGZsq{}}\PYG{p}{]}\PYG{p}{[}\PYG{l+m+mi}{0}\PYG{p}{]}
                \PYG{o}{+} \PYG{n}{StructureParameters}\PYG{p}{[}\PYG{l+s+s1}{\PYGZsq{}}\PYG{l+s+s1}{CornerCubeArray}\PYG{l+s+s1}{\PYGZsq{}}\PYG{p}{]}\PYG{p}{[}\PYG{l+s+s1}{\PYGZsq{}}\PYG{l+s+s1}{ArrayXOffset}\PYG{l+s+s1}{\PYGZsq{}}\PYG{p}{]} \PYG{o}{*} \PYG{p}{(}\PYG{n}{Column} \PYG{o}{\PYGZhy{}} \PYG{l+m+mi}{1}\PYG{p}{)}
                \PYG{o}{+} \PYG{n}{StructureParameters}\PYG{p}{[}\PYG{l+s+s1}{\PYGZsq{}}\PYG{l+s+s1}{CornerCube}\PYG{l+s+s1}{\PYGZsq{}}\PYG{p}{]}\PYG{p}{[}\PYG{l+s+s1}{\PYGZsq{}}\PYG{l+s+s1}{LabelXOffset}\PYG{l+s+s1}{\PYGZsq{}}\PYG{p}{]}\PYG{p}{,}
                \PYG{n}{StructureParameters}\PYG{p}{[}\PYG{l+s+s1}{\PYGZsq{}}\PYG{l+s+s1}{CornerCubeArray}\PYG{l+s+s1}{\PYGZsq{}}\PYG{p}{]}\PYG{p}{[}\PYG{l+s+s1}{\PYGZsq{}}\PYG{l+s+s1}{ArrayOrigin}\PYG{l+s+s1}{\PYGZsq{}}\PYG{p}{]}\PYG{p}{[}\PYG{l+m+mi}{1}\PYG{p}{]}
                \PYG{o}{+} \PYG{n}{StructureParameters}\PYG{p}{[}\PYG{l+s+s1}{\PYGZsq{}}\PYG{l+s+s1}{CornerCubeArray}\PYG{l+s+s1}{\PYGZsq{}}\PYG{p}{]}\PYG{p}{[}\PYG{l+s+s1}{\PYGZsq{}}\PYG{l+s+s1}{ArrayYOffset}\PYG{l+s+s1}{\PYGZsq{}}\PYG{p}{]} \PYG{o}{*} \PYG{p}{(}\PYG{n}{Row} \PYG{o}{\PYGZhy{}} \PYG{l+m+mi}{1}\PYG{p}{)}
                \PYG{o}{+} \PYG{n}{StructureParameters}\PYG{p}{[}\PYG{l+s+s1}{\PYGZsq{}}\PYG{l+s+s1}{CornerCube}\PYG{l+s+s1}{\PYGZsq{}}\PYG{p}{]}\PYG{p}{[}\PYG{l+s+s1}{\PYGZsq{}}\PYG{l+s+s1}{LabelYOffset}\PYG{l+s+s1}{\PYGZsq{}}\PYG{p}{]}\PYG{p}{]}\PYG{p}{,}
            \PYG{n}{FontSize}\PYG{o}{=}\PYG{l+m+mi}{16}\PYG{p}{,}
            \PYG{n}{StrokeWidth}\PYG{o}{=}\PYG{l+m+mi}{3}\PYG{p}{,}
            \PYG{n}{RoundCaps}\PYG{o}{=}\PYG{k+kc}{True}\PYG{p}{,}\PYG{c+c1}{\PYGZsh{} Set to False to improve e\PYGZhy{}Beam performance}
            \PYG{n}{Layer}\PYG{o}{=}\PYG{l+m+mi}{1}\PYG{p}{,}
            \PYG{n}{RotationAngle}\PYG{o}{=}\PYG{l+m+mi}{45}
        \PYG{p}{)}


\PYG{c+c1}{\PYGZsh{}\PYGZsh{}\PYGZsh{}\PYGZsh{}\PYGZsh{}\PYGZsh{}\PYGZsh{}\PYGZsh{}\PYGZsh{}\PYGZsh{}\PYGZsh{}\PYGZsh{}\PYGZsh{}\PYGZsh{}\PYGZsh{}\PYGZsh{}\PYGZsh{}\PYGZsh{}\PYGZsh{}\PYGZsh{}\PYGZsh{}\PYGZsh{}\PYGZsh{}\PYGZsh{}\PYGZsh{}}
\PYG{c+c1}{\PYGZsh{} Generate Output Files \PYGZsh{}}
\PYG{c+c1}{\PYGZsh{}\PYGZsh{}\PYGZsh{}\PYGZsh{}\PYGZsh{}\PYGZsh{}\PYGZsh{}\PYGZsh{}\PYGZsh{}\PYGZsh{}\PYGZsh{}\PYGZsh{}\PYGZsh{}\PYGZsh{}\PYGZsh{}\PYGZsh{}\PYGZsh{}\PYGZsh{}\PYGZsh{}\PYGZsh{}\PYGZsh{}\PYGZsh{}\PYGZsh{}\PYGZsh{}\PYGZsh{}}

\PYG{c+c1}{\PYGZsh{} Note: The suffix (.txl, .html, .svg) will be appended automatically}
\PYG{n}{TXLWriter}\PYG{o}{.}\PYG{n}{GenerateFiles}\PYG{p}{(}\PYG{l+s+s1}{\PYGZsq{}}\PYG{l+s+s1}{Masks/Example\PYGZus{}Advanced}\PYG{l+s+s1}{\PYGZsq{}}\PYG{p}{)}

\end{Verbatim}


\subsubsection{Generated SVG Image}
\label{Chapters/20_Examples:id6}\label{Chapters/20_Examples:exampleadvancedsvg}\begin{figure}[htbp]
\centering
\capstart

\includegraphics{{Example_Advanced}.png}
\caption{Generated SVG Image for \titleref{Content/Example\_Advanced.py}}\label{Chapters/20_Examples:id9}\end{figure}


\section{TXLConverter}
\label{Chapters/30_TXLConverter:txlconverter}\label{Chapters/30_TXLConverter:sec-txlconverter}\label{Chapters/30_TXLConverter::doc}
blub


\section{Python Module Reference}
\label{Chapters/40_PythonModuleReference:python-module-reference}\label{Chapters/40_PythonModuleReference:pythonmodulereference}\label{Chapters/40_PythonModuleReference::doc}

\subsection{TXLWriter}
\label{Chapters/40_PythonModuleReference:txlwriter}
\begin{longtable}{ll}
\hline
\endfirsthead

\multicolumn{2}{c}%
{{\tablecontinued{\tablename\ \thetable{} -- continued from previous page}}} \\
\hline
\endhead

\hline \multicolumn{2}{|r|}{{\tablecontinued{Continued on next page}}} \\ \hline
\endfoot

\endlastfoot


{\hyperref[Chapters/PythonModuleReference/TXLWriter/TXLWizard.TXLWriter:module\string-TXLWizard.TXLWriter]{\crossref{\code{TXLWizard.TXLWriter}}}}
 & 
Controller class for generating TXL / SVG / HTML output.
\\
\hline\end{longtable}



\subsubsection{TXLWizard.TXLWriter}
\label{Chapters/PythonModuleReference/TXLWriter/TXLWizard.TXLWriter:txlwizard-txlwriter}\label{Chapters/PythonModuleReference/TXLWriter/TXLWizard.TXLWriter:module-TXLWizard.TXLWriter}\label{Chapters/PythonModuleReference/TXLWriter/TXLWizard.TXLWriter::doc}\index{TXLWizard.TXLWriter (module)}
Controller class for generating TXL / SVG / HTML output.

Here we can add structures (definitions and content) which will be rendered in the output.


\paragraph{Classes}
\label{Chapters/PythonModuleReference/TXLWriter/TXLWizard.TXLWriter:classes}
\begin{longtable}{ll}
\hline
\endfirsthead

\multicolumn{2}{c}%
{{\tablecontinued{\tablename\ \thetable{} -- continued from previous page}}} \\
\hline
\endhead

\hline \multicolumn{2}{|r|}{{\tablecontinued{Continued on next page}}} \\ \hline
\endfoot

\endlastfoot


{\hyperref[Chapters/PythonModuleReference/TXLWriter/TXLWizard.TXLWriter:TXLWizard.TXLWriter.TXLWriter]{\crossref{\code{TXLWriter}}}}(**kwargs)
 & 
Controller class for generating TXL / SVG / HTML output.
\\
\hline\end{longtable}

\index{TXLWriter (class in TXLWizard.TXLWriter)}

\begin{fulllineitems}
\phantomsection\label{Chapters/PythonModuleReference/TXLWriter/TXLWizard.TXLWriter:TXLWizard.TXLWriter.TXLWriter}\pysiglinewithargsret{\strong{class }\code{TXLWizard.TXLWriter.}\bfcode{TXLWriter}}{\emph{**kwargs}}{}
Bases: \code{object}

Controller class for generating TXL / SVG / HTML output.

Here we can add structures (definitions and content) which will be rendered in the output.

Optionally, a coordinate system grid is drawn.
\begin{quote}\begin{description}
\item[{Parameters}] \leavevmode\begin{itemize}
\item {} 
\textbf{\texttt{ShowGrid}} (\emph{\texttt{bool, optional}}) -- 
Show the coordinate system grid or not.

Defaults to True


\item {} 
\textbf{\texttt{GridWidth}} (\emph{\texttt{int, optional}}) -- 
Full width of the coordinate system grid in um.

Defaults to 800


\item {} 
\textbf{\texttt{GridHeight}} (\emph{\texttt{int, optional}}) -- 
Full height of the coordinate system grid in um.

Defaults to 800


\item {} 
\textbf{\texttt{GridSpacing}} (\emph{\texttt{int, optional}}) -- 
Coordinate Sytem Grid Spacing in um.

Defaults to 100


\item {} 
\textbf{\texttt{SubGridSpacing}} (\emph{\texttt{int, optional}}) -- 
Coordinate System Sub-Grid Spacing in um.

Defaults to 10


\end{itemize}

\end{description}\end{quote}
\paragraph{Examples}

Initialize TXLWriter, add a definition structure,

\begin{Verbatim}[commandchars=\\\{\}]
\PYG{g+gp}{\PYGZgt{}\PYGZgt{}\PYGZgt{} }\PYG{n}{TXLWriter} \PYG{o}{=} \PYG{n}{TXLWizard}\PYG{o}{.}\PYG{n}{TXLWriter}\PYG{o}{.}\PYG{n}{TXLWriter}\PYG{p}{(}
\PYG{g+gp}{\PYGZgt{}\PYGZgt{}\PYGZgt{} }   \PYG{n}{ShowGrid}\PYG{o}{=}\PYG{k+kc}{True}\PYG{p}{,} \PYG{n}{GridWidth}\PYG{o}{=}\PYG{l+m+mi}{800}\PYG{p}{,} \PYG{n}{GridHeight}\PYG{o}{=}\PYG{l+m+mi}{800}
\PYG{g+gp}{\PYGZgt{}\PYGZgt{}\PYGZgt{} }\PYG{p}{)}
\end{Verbatim}

Add a definition structure and add a pattern of type \titleref{Circle}

\begin{Verbatim}[commandchars=\\\{\}]
\PYG{g+gp}{\PYGZgt{}\PYGZgt{}\PYGZgt{} }\PYG{n}{MyDefinitionStructure} \PYG{o}{=} \PYG{n}{TXLWriter}\PYG{o}{.}\PYG{n}{AddDefinitionStructure}\PYG{p}{(}\PYG{l+s+s1}{\PYGZsq{}}\PYG{l+s+s1}{MyDefinition}\PYG{l+s+s1}{\PYGZsq{}}\PYG{p}{)}
\PYG{g+gp}{\PYGZgt{}\PYGZgt{}\PYGZgt{} }\PYG{n}{MyDefinitionStructure}\PYG{o}{.}\PYG{n}{AddPattern}\PYG{p}{(}\PYG{l+s+s1}{\PYGZsq{}}\PYG{l+s+s1}{Circle}\PYG{l+s+s1}{\PYGZsq{}}\PYG{p}{,} \PYG{n}{Center}\PYG{o}{=}\PYG{p}{[}\PYG{l+m+mi}{0}\PYG{p}{,}\PYG{l+m+mi}{0}\PYG{p}{]}\PYG{p}{,} \PYG{n}{Radius}\PYG{o}{=}\PYG{l+m+mi}{20}\PYG{p}{)}
\end{Verbatim}

add a content structure with a pattern \titleref{Reference} to reuse the definition structure.

\begin{Verbatim}[commandchars=\\\{\}]
\PYG{g+gp}{\PYGZgt{}\PYGZgt{}\PYGZgt{} }\PYG{n}{MyContentStructure} \PYG{o}{=} \PYG{n}{TXLWriter}\PYG{o}{.}\PYG{n}{AddContentStructure}\PYG{p}{(}\PYG{l+s+s1}{\PYGZsq{}}\PYG{l+s+s1}{MySuperCircle}\PYG{l+s+s1}{\PYGZsq{}}\PYG{p}{)}
\PYG{g+gp}{\PYGZgt{}\PYGZgt{}\PYGZgt{} }\PYG{n}{MyContentStructure}\PYG{o}{.}\PYG{n}{AddPattern}\PYG{p}{(}
\PYG{g+gp}{\PYGZgt{}\PYGZgt{}\PYGZgt{} }   \PYG{l+s+s1}{\PYGZsq{}}\PYG{l+s+s1}{Reference}\PYG{l+s+s1}{\PYGZsq{}}\PYG{p}{,}
\PYG{g+gp}{\PYGZgt{}\PYGZgt{}\PYGZgt{} }   \PYG{n}{ReferencedStructureID}\PYG{o}{=}\PYG{n}{MyDefinitionStructure}\PYG{o}{.}\PYG{n}{ID}\PYG{p}{,}
\PYG{g+gp}{\PYGZgt{}\PYGZgt{}\PYGZgt{} }   \PYG{n}{OriginPoint}\PYG{o}{=}\PYG{p}{[}\PYG{l+m+mi}{20}\PYG{p}{,}\PYG{l+m+mi}{50}\PYG{p}{]}
\PYG{g+gp}{\PYGZgt{}\PYGZgt{}\PYGZgt{} }\PYG{p}{)}
\end{Verbatim}

Generate the Output files with name \titleref{mask.(txl\textbar{}html\textbar{}svg)} to the folder \titleref{myPath}

\begin{Verbatim}[commandchars=\\\{\}]
\PYG{g+gp}{\PYGZgt{}\PYGZgt{}\PYGZgt{} }\PYG{n}{TXLWriter}\PYG{o}{.}\PYG{n}{GenerateFiles}\PYG{p}{(}\PYG{l+s+s1}{\PYGZsq{}}\PYG{l+s+s1}{myPath/mask}\PYG{l+s+s1}{\PYGZsq{}}\PYG{p}{)}
\end{Verbatim}
\index{AddContentStructure() (TXLWizard.TXLWriter.TXLWriter method)}

\begin{fulllineitems}
\phantomsection\label{Chapters/PythonModuleReference/TXLWriter/TXLWizard.TXLWriter:TXLWizard.TXLWriter.TXLWriter.AddContentStructure}\pysiglinewithargsret{\bfcode{AddContentStructure}}{\emph{ID}, \emph{**kwargs}}{}
Add content structure. A content structure can hold patterns that will render in the output.

A structure corresponds to the ``STRUCT'' command in the TXL file format.
\begin{quote}\begin{description}
\item[{Parameters}] \leavevmode\begin{itemize}
\item {} 
\textbf{\texttt{ID}} (\emph{\texttt{str}}) -- Unique identification of the structure. Must be used when referencing to this structure.

\item {} 
\textbf{\texttt{kwargs}} (\emph{\texttt{dict}}) -- keyword arguments passed to the structure constructor

\end{itemize}

\item[{Returns}] \leavevmode


\item[{Return type}] \leavevmode
{\hyperref[Chapters/PythonModuleReference/Patterns/TXLWizard.Patterns.Structure:TXLWizard.Patterns.Structure.Structure]{\crossref{\code{TXLWizard.Patterns.Structure.Structure}}}} structure instance

\end{description}\end{quote}

\end{fulllineitems}

\index{AddDefinitionStructure() (TXLWizard.TXLWriter.TXLWriter method)}

\begin{fulllineitems}
\phantomsection\label{Chapters/PythonModuleReference/TXLWriter/TXLWizard.TXLWriter:TXLWizard.TXLWriter.TXLWriter.AddDefinitionStructure}\pysiglinewithargsret{\bfcode{AddDefinitionStructure}}{\emph{ID}, \emph{**kwargs}}{}
Add definition structure. A definition structure can be referenced by a content structure.

A structure corresponds to the ``STRUCT'' command in the TXL file format.
\begin{quote}\begin{description}
\item[{Parameters}] \leavevmode\begin{itemize}
\item {} 
\textbf{\texttt{ID}} (\emph{\texttt{str}}) -- Unique identification of the structure. Must be used when referencing to this structure.

\item {} 
\textbf{\texttt{kwargs}} (\emph{\texttt{dict}}) -- keyword arguments passed to the structure constructor

\end{itemize}

\item[{Returns}] \leavevmode


\item[{Return type}] \leavevmode
{\hyperref[Chapters/PythonModuleReference/Patterns/TXLWizard.Patterns.Structure:TXLWizard.Patterns.Structure.Structure]{\crossref{\code{TXLWizard.Patterns.Structure.Structure}}}} structure instance

\end{description}\end{quote}

\end{fulllineitems}

\index{GenerateFiles() (TXLWizard.TXLWriter.TXLWriter method)}

\begin{fulllineitems}
\phantomsection\label{Chapters/PythonModuleReference/TXLWriter/TXLWizard.TXLWriter:TXLWizard.TXLWriter.TXLWriter.GenerateFiles}\pysiglinewithargsret{\bfcode{GenerateFiles}}{\emph{Filename}, \emph{TXL=True}, \emph{SVG=True}, \emph{HTML=True}}{}
Generate the output files (.txl, .svg, .html).
\begin{quote}\begin{description}
\item[{Parameters}] \leavevmode\begin{itemize}
\item {} 
\textbf{\texttt{Filename}} (\emph{\texttt{str}}) -- Path / Filename without extension.
The corresponding path will be created if it does not exist

\item {} 
\textbf{\texttt{TXL}} (\emph{\texttt{Optional{[}bool{]}}}) -- Enable TXL Output

\item {} 
\textbf{\texttt{SVG}} (\emph{\texttt{Optional{[}bool{]}}}) -- Enable SVG Output

\item {} 
\textbf{\texttt{HTML}} (\emph{\texttt{Optional{[}bool{]}}}) -- Enable HTML Output

\end{itemize}

\end{description}\end{quote}

\end{fulllineitems}


\end{fulllineitems}



\subsection{Patterns}
\label{Chapters/40_PythonModuleReference:patterns}\label{Chapters/40_PythonModuleReference:pythonmodulereferencepatterns}
\begin{longtable}{ll}
\hline
\endfirsthead

\multicolumn{2}{c}%
{{\tablecontinued{\tablename\ \thetable{} -- continued from previous page}}} \\
\hline
\endhead

\hline \multicolumn{2}{|r|}{{\tablecontinued{Continued on next page}}} \\ \hline
\endfoot

\endlastfoot


{\hyperref[Chapters/PythonModuleReference/Patterns/TXLWizard.Patterns.AbstractPattern:module\string-TXLWizard.Patterns.AbstractPattern]{\crossref{\code{TXLWizard.Patterns.AbstractPattern}}}}
 & 
Provides an abstract class for \titleref{Pattern} objects.
\\
\hline
{\hyperref[Chapters/PythonModuleReference/Patterns/TXLWizard.Patterns.Array:module\string-TXLWizard.Patterns.Array]{\crossref{\code{TXLWizard.Patterns.Array}}}}
 & 
Implements a class for \titleref{Pattern} objects of type \titleref{Array} (\titleref{AREF}).
\\
\hline
{\hyperref[Chapters/PythonModuleReference/Patterns/TXLWizard.Patterns.Circle:module\string-TXLWizard.Patterns.Circle]{\crossref{\code{TXLWizard.Patterns.Circle}}}}
 & 
Implements a class for \titleref{Pattern} objects of type \titleref{Circle} (\titleref{C}).
\\
\hline
{\hyperref[Chapters/PythonModuleReference/Patterns/TXLWizard.Patterns.Ellipse:module\string-TXLWizard.Patterns.Ellipse]{\crossref{\code{TXLWizard.Patterns.Ellipse}}}}
 & 
Implements a class for \titleref{Pattern} objects of type \titleref{Ellipse} (\titleref{ELP}).
\\
\hline
{\hyperref[Chapters/PythonModuleReference/Patterns/TXLWizard.Patterns.Polygon:module\string-TXLWizard.Patterns.Polygon]{\crossref{\code{TXLWizard.Patterns.Polygon}}}}
 & 
Implements a class for \titleref{Pattern} objects of type \titleref{Polygon} (\titleref{B}).
\\
\hline
{\hyperref[Chapters/PythonModuleReference/Patterns/TXLWizard.Patterns.Polyline:module\string-TXLWizard.Patterns.Polyline]{\crossref{\code{TXLWizard.Patterns.Polyline}}}}
 & 
Implements a class for \titleref{Pattern} objects of type \titleref{Polyline} (\titleref{B}).
\\
\hline
{\hyperref[Chapters/PythonModuleReference/Patterns/TXLWizard.Patterns.Reference:module\string-TXLWizard.Patterns.Reference]{\crossref{\code{TXLWizard.Patterns.Reference}}}}
 & 
Implements a class for \titleref{Pattern} objects of type \titleref{Reference} (\titleref{SREF}).
\\
\hline
{\hyperref[Chapters/PythonModuleReference/Patterns/TXLWizard.Patterns.Structure:module\string-TXLWizard.Patterns.Structure]{\crossref{\code{TXLWizard.Patterns.Structure}}}}
 & 
Implements a class for \titleref{Structure} objects (\titleref{STRUCT}).
\\
\hline\end{longtable}



\subsubsection{TXLWizard.Patterns.AbstractPattern}
\label{Chapters/PythonModuleReference/Patterns/TXLWizard.Patterns.AbstractPattern::doc}\label{Chapters/PythonModuleReference/Patterns/TXLWizard.Patterns.AbstractPattern:txlwizard-patterns-abstractpattern}\label{Chapters/PythonModuleReference/Patterns/TXLWizard.Patterns.AbstractPattern:module-TXLWizard.Patterns.AbstractPattern}\index{TXLWizard.Patterns.AbstractPattern (module)}
Provides an abstract class for \titleref{Pattern} objects.


\paragraph{Classes}
\label{Chapters/PythonModuleReference/Patterns/TXLWizard.Patterns.AbstractPattern:classes}
\begin{longtable}{ll}
\hline
\endfirsthead

\multicolumn{2}{c}%
{{\tablecontinued{\tablename\ \thetable{} -- continued from previous page}}} \\
\hline
\endhead

\hline \multicolumn{2}{|r|}{{\tablecontinued{Continued on next page}}} \\ \hline
\endfoot

\endlastfoot


{\hyperref[Chapters/PythonModuleReference/Patterns/TXLWizard.Patterns.AbstractPattern:TXLWizard.Patterns.AbstractPattern.AbstractPattern]{\crossref{\code{AbstractPattern}}}}(**kwargs)
 & 
Provides an abstract class for \titleref{Pattern} objects.
\\
\hline\end{longtable}

\index{AbstractPattern (class in TXLWizard.Patterns.AbstractPattern)}

\begin{fulllineitems}
\phantomsection\label{Chapters/PythonModuleReference/Patterns/TXLWizard.Patterns.AbstractPattern:TXLWizard.Patterns.AbstractPattern.AbstractPattern}\pysiglinewithargsret{\strong{class }\code{TXLWizard.Patterns.AbstractPattern.}\bfcode{AbstractPattern}}{\emph{**kwargs}}{}
Bases: \code{object}

Provides an abstract class for \titleref{Pattern} objects.
\begin{quote}\begin{description}
\item[{Parameters}] \leavevmode\begin{itemize}
\item {} 
\textbf{\texttt{Layer}} (\emph{\texttt{int, optional}}) -- Specifies the \titleref{Layer} attribute of the pattern.
Defaults to None.

\item {} 
\textbf{\texttt{DataType}} (\emph{\texttt{int, optional}}) -- Specifies the \titleref{DataType} attribute of the pattern.
Defaults to None.

\item {} 
\textbf{\texttt{RotationAngle}} (\emph{\texttt{float, optional}}) -- Specifies the \titleref{RotationAngle} attribute of the pattern.
Defaults to None.

\item {} 
\textbf{\texttt{StrokeWidth}} (\emph{\texttt{float, optional}}) -- Specifies the \titleref{StrokeWidth} attribute of the pattern.
Defaults to None.

\item {} 
\textbf{\texttt{ScaleFactor}} (\emph{\texttt{float, optional}}) -- Specifies the \titleref{ScaleFactor} attribute of the pattern.
Defaults to None.

\end{itemize}

\end{description}\end{quote}
\index{Attributes (TXLWizard.Patterns.AbstractPattern.AbstractPattern attribute)}

\begin{fulllineitems}
\phantomsection\label{Chapters/PythonModuleReference/Patterns/TXLWizard.Patterns.AbstractPattern:TXLWizard.Patterns.AbstractPattern.AbstractPattern.Attributes}\pysigline{\bfcode{Attributes}\strong{ = None}}
\emph{dict} -- attribute values of the current pattern. Default values are copied from \titleref{self.DefaultAttributes}

\end{fulllineitems}

\index{DefaultAttributes (TXLWizard.Patterns.AbstractPattern.AbstractPattern attribute)}

\begin{fulllineitems}
\phantomsection\label{Chapters/PythonModuleReference/Patterns/TXLWizard.Patterns.AbstractPattern:TXLWizard.Patterns.AbstractPattern.AbstractPattern.DefaultAttributes}\pysigline{\bfcode{DefaultAttributes}\strong{ = None}}
\emph{dict} -- default attributes that are copied to \titleref{self.Attributes} upon instantiation. Specifies the allowed attributes

\end{fulllineitems}

\index{GetSVGOutput() (TXLWizard.Patterns.AbstractPattern.AbstractPattern method)}

\begin{fulllineitems}
\phantomsection\label{Chapters/PythonModuleReference/Patterns/TXLWizard.Patterns.AbstractPattern:TXLWizard.Patterns.AbstractPattern.AbstractPattern.GetSVGOutput}\pysiglinewithargsret{\bfcode{GetSVGOutput}}{}{}
Generates the SVG output xml for the current pattern.
Needs to be implemented for each pattern type separately in the corresponding inheriting class.
\begin{quote}\begin{description}
\item[{Returns}] \leavevmode
SVG output xml

\item[{Return type}] \leavevmode
str

\end{description}\end{quote}

\end{fulllineitems}

\index{GetTXLOutput() (TXLWizard.Patterns.AbstractPattern.AbstractPattern method)}

\begin{fulllineitems}
\phantomsection\label{Chapters/PythonModuleReference/Patterns/TXLWizard.Patterns.AbstractPattern:TXLWizard.Patterns.AbstractPattern.AbstractPattern.GetTXLOutput}\pysiglinewithargsret{\bfcode{GetTXLOutput}}{}{}
Generates the TXL output commands for the current pattern.
Needs to be implemented for each pattern type separately in the corresponding inheriting class.
\begin{quote}\begin{description}
\item[{Returns}] \leavevmode
TXL output commands

\item[{Return type}] \leavevmode
str

\end{description}\end{quote}

\end{fulllineitems}

\index{ParentStructure (TXLWizard.Patterns.AbstractPattern.AbstractPattern attribute)}

\begin{fulllineitems}
\phantomsection\label{Chapters/PythonModuleReference/Patterns/TXLWizard.Patterns.AbstractPattern:TXLWizard.Patterns.AbstractPattern.AbstractPattern.ParentStructure}\pysigline{\bfcode{ParentStructure}\strong{ = None}}
{\hyperref[Chapters/PythonModuleReference/Patterns/TXLWizard.Patterns.Structure:TXLWizard.Patterns.Structure.Structure]{\crossref{\code{TXLWizard.Patterns.Structure.Structure}}}}, reference to the \titleref{Structure} instance containing the current pattern

\end{fulllineitems}

\index{Type (TXLWizard.Patterns.AbstractPattern.AbstractPattern attribute)}

\begin{fulllineitems}
\phantomsection\label{Chapters/PythonModuleReference/Patterns/TXLWizard.Patterns.AbstractPattern:TXLWizard.Patterns.AbstractPattern.AbstractPattern.Type}\pysigline{\bfcode{Type}\strong{ = None}}
\emph{str} -- specifies the type of the pattern.

\end{fulllineitems}


\end{fulllineitems}



\subsubsection{TXLWizard.Patterns.Array}
\label{Chapters/PythonModuleReference/Patterns/TXLWizard.Patterns.Array:txlwizard-patterns-array}\label{Chapters/PythonModuleReference/Patterns/TXLWizard.Patterns.Array::doc}\label{Chapters/PythonModuleReference/Patterns/TXLWizard.Patterns.Array:module-TXLWizard.Patterns.Array}\index{TXLWizard.Patterns.Array (module)}
Implements a class for \titleref{Pattern} objects of type \titleref{Array} (\titleref{AREF}).

Replicates the referenced structure in two directions.


\paragraph{Classes}
\label{Chapters/PythonModuleReference/Patterns/TXLWizard.Patterns.Array:classes}
\begin{longtable}{ll}
\hline
\endfirsthead

\multicolumn{2}{c}%
{{\tablecontinued{\tablename\ \thetable{} -- continued from previous page}}} \\
\hline
\endhead

\hline \multicolumn{2}{|r|}{{\tablecontinued{Continued on next page}}} \\ \hline
\endfoot

\endlastfoot


{\hyperref[Chapters/PythonModuleReference/Patterns/TXLWizard.Patterns.Array:TXLWizard.Patterns.Array.Array]{\crossref{\code{Array}}}}(ReferencedStructureID, OriginPoint, ...)
 & 
Implements a class for \titleref{Pattern} objects of type \titleref{Array}.
\\
\hline\end{longtable}

\index{Array (class in TXLWizard.Patterns.Array)}

\begin{fulllineitems}
\phantomsection\label{Chapters/PythonModuleReference/Patterns/TXLWizard.Patterns.Array:TXLWizard.Patterns.Array.Array}\pysiglinewithargsret{\strong{class }\code{TXLWizard.Patterns.Array.}\bfcode{Array}}{\emph{ReferencedStructureID}, \emph{OriginPoint}, \emph{PositionDelta1}, \emph{PositionDelta2}, \emph{Repetitions1}, \emph{Repetitions2}, \emph{**kwargs}}{}
Bases: {\hyperref[Chapters/PythonModuleReference/Patterns/TXLWizard.Patterns.AbstractPattern:TXLWizard.Patterns.AbstractPattern.AbstractPattern]{\crossref{\code{TXLWizard.Patterns.AbstractPattern.AbstractPattern}}}}

Implements a class for \titleref{Pattern} objects of type \titleref{Array}.

Corresponds to the TXL command \titleref{AREF}.

Replicates the referenced structure \titleref{ReferencedStructureID} in two directions \titleref{PositionDelta1} and \titleref{PositionDelta2}
for the number of times specified in \titleref{Repetitions1} and \titleref{Repetitions2},
starting at \titleref{OriginPoint}.

The x- and y-coordinates of the replicated objects are calculated as follows:
\titleref{OriginPoint+i*PositionDelta1+j*PositionDelta2}
where \titleref{i} is an integer that ranges from 0 to \titleref{Repetitions1}
and \titleref{j} is an integer that ranges from 0 to \titleref{Repetitions2}
\begin{quote}\begin{description}
\item[{Parameters}] \leavevmode\begin{itemize}
\item {} 
\textbf{\texttt{ReferencedStructureID}} (\emph{\texttt{str}}) -- ID of the structure being referenced to

\item {} 
\textbf{\texttt{OriginPoint}} (\emph{\texttt{list of float}}) -- x- and y- coordinates of the starting point

\item {} 
\textbf{\texttt{PositionDelta1}} (\emph{\texttt{list of float}}) -- x- and y- coordinates of the first replication direction.

\item {} 
\textbf{\texttt{PositionDelta2}} (\emph{\texttt{list of float}}) -- x- and y- coordinates of the second replication direction.

\item {} 
\textbf{\texttt{Repetitions1}} (\emph{\texttt{int}}) -- Number of replications in the first replication direction

\item {} 
\textbf{\texttt{Repetitions2}} (\emph{\texttt{int}}) -- Number of replications in the second replication direction

\item {} 
\textbf{\texttt{**kwargs}} -- keyword arguments passed to the {\hyperref[Chapters/PythonModuleReference/Patterns/TXLWizard.Patterns.AbstractPattern:TXLWizard.Patterns.AbstractPattern.AbstractPattern]{\crossref{\code{TXLWizard.Patterns.AbstractPattern.AbstractPattern}}}} constructor.
Can specify attributes of the current pattern.

\end{itemize}

\end{description}\end{quote}
\index{PositionDelta1 (TXLWizard.Patterns.Array.Array attribute)}

\begin{fulllineitems}
\phantomsection\label{Chapters/PythonModuleReference/Patterns/TXLWizard.Patterns.Array:TXLWizard.Patterns.Array.Array.PositionDelta1}\pysigline{\bfcode{PositionDelta1}\strong{ = None}}
\emph{list of float} -- x- and y- coordinates of the first replication direction.

\end{fulllineitems}

\index{PositionDelta2 (TXLWizard.Patterns.Array.Array attribute)}

\begin{fulllineitems}
\phantomsection\label{Chapters/PythonModuleReference/Patterns/TXLWizard.Patterns.Array:TXLWizard.Patterns.Array.Array.PositionDelta2}\pysigline{\bfcode{PositionDelta2}\strong{ = None}}
\emph{list of float} -- x- and y- coordinates of the second replication direction.

\end{fulllineitems}

\index{ReferencedStructureID (TXLWizard.Patterns.Array.Array attribute)}

\begin{fulllineitems}
\phantomsection\label{Chapters/PythonModuleReference/Patterns/TXLWizard.Patterns.Array:TXLWizard.Patterns.Array.Array.ReferencedStructureID}\pysigline{\bfcode{ReferencedStructureID}\strong{ = None}}
\emph{str} -- ID of the structure being referenced to

\end{fulllineitems}

\index{Repetitions1 (TXLWizard.Patterns.Array.Array attribute)}

\begin{fulllineitems}
\phantomsection\label{Chapters/PythonModuleReference/Patterns/TXLWizard.Patterns.Array:TXLWizard.Patterns.Array.Array.Repetitions1}\pysigline{\bfcode{Repetitions1}\strong{ = None}}
\emph{int} -- Number of replications in the first replication direction

\end{fulllineitems}

\index{Repetitions2 (TXLWizard.Patterns.Array.Array attribute)}

\begin{fulllineitems}
\phantomsection\label{Chapters/PythonModuleReference/Patterns/TXLWizard.Patterns.Array:TXLWizard.Patterns.Array.Array.Repetitions2}\pysigline{\bfcode{Repetitions2}\strong{ = None}}
\emph{int} -- Number of replications in the second replication direction

\end{fulllineitems}

\index{Type (TXLWizard.Patterns.Array.Array attribute)}

\begin{fulllineitems}
\phantomsection\label{Chapters/PythonModuleReference/Patterns/TXLWizard.Patterns.Array:TXLWizard.Patterns.Array.Array.Type}\pysigline{\bfcode{Type}\strong{ = None}}
\emph{str} -- specifies the type of the pattern. Set to `Array'

\end{fulllineitems}


\end{fulllineitems}



\subsubsection{TXLWizard.Patterns.Circle}
\label{Chapters/PythonModuleReference/Patterns/TXLWizard.Patterns.Circle:module-TXLWizard.Patterns.Circle}\label{Chapters/PythonModuleReference/Patterns/TXLWizard.Patterns.Circle::doc}\label{Chapters/PythonModuleReference/Patterns/TXLWizard.Patterns.Circle:txlwizard-patterns-circle}\index{TXLWizard.Patterns.Circle (module)}
Implements a class for \titleref{Pattern} objects of type \titleref{Circle} (\titleref{C}).

Renders a circle.


\paragraph{Classes}
\label{Chapters/PythonModuleReference/Patterns/TXLWizard.Patterns.Circle:classes}
\begin{longtable}{ll}
\hline
\endfirsthead

\multicolumn{2}{c}%
{{\tablecontinued{\tablename\ \thetable{} -- continued from previous page}}} \\
\hline
\endhead

\hline \multicolumn{2}{|r|}{{\tablecontinued{Continued on next page}}} \\ \hline
\endfoot

\endlastfoot


{\hyperref[Chapters/PythonModuleReference/Patterns/TXLWizard.Patterns.Circle:TXLWizard.Patterns.Circle.Circle]{\crossref{\code{Circle}}}}(Center, Radius, **kwargs)
 & 
Implements a class for \titleref{Pattern} objects of type \titleref{Circle}.
\\
\hline\end{longtable}

\index{Circle (class in TXLWizard.Patterns.Circle)}

\begin{fulllineitems}
\phantomsection\label{Chapters/PythonModuleReference/Patterns/TXLWizard.Patterns.Circle:TXLWizard.Patterns.Circle.Circle}\pysiglinewithargsret{\strong{class }\code{TXLWizard.Patterns.Circle.}\bfcode{Circle}}{\emph{Center}, \emph{Radius}, \emph{**kwargs}}{}
Bases: {\hyperref[Chapters/PythonModuleReference/Patterns/TXLWizard.Patterns.AbstractPattern:TXLWizard.Patterns.AbstractPattern.AbstractPattern]{\crossref{\code{TXLWizard.Patterns.AbstractPattern.AbstractPattern}}}}

Implements a class for \titleref{Pattern} objects of type \titleref{Circle}.

Corresponds to the TXL command \titleref{C} (\titleref{CP} if \titleref{PathOnly} is specified, \titleref{CPR} if \titleref{RoundCaps} and \titleref{CPE} if \titleref{Extended}).

Renders a circle.

Optionally, only a sector is shown when specifying \titleref{StartAngle} and \titleref{EndAngle}.

If \titleref{NumberOfPoints} is given, the number of path segments defining the circle can be specified.

If \titleref{PathOnly} is set to True, only the arc of the circle is shown. Optionally, the ends of the path are rounded by
specifying \titleref{RoundCaps} or extended by specifying \titleref{Extended} along with \titleref{PathOnly}.
\begin{quote}\begin{description}
\item[{Parameters}] \leavevmode\begin{itemize}
\item {} 
\textbf{\texttt{Center}} (\emph{\texttt{list of float}}) -- x- and y-coordinates specifying the center of the circle

\item {} 
\textbf{\texttt{Radius}} (\emph{\texttt{float}}) -- Radius of the circle

\item {} 
\textbf{\texttt{StartAngle}} (\emph{\texttt{float, optional}}) -- If given, only a sector is drawn from \titleref{StartAngle} to \titleref{EndAngle}.
Defaults to None.

\item {} 
\textbf{\texttt{EndAngle}} (\emph{\texttt{float, optional}}) -- If given, only a sector is drawn from \titleref{StartAngle} to \titleref{EndAngle}.
Defaults to None.

\item {} 
\textbf{\texttt{NumberOfPoints}} (\emph{\texttt{int, optional}}) -- Number of path segments used for drawing the circle.
Defaults to None.

\item {} 
\textbf{\texttt{PathOnly}} (\emph{\texttt{bool, optional}}) -- If set to True, only the arc of the circle is drawn.
Defaults to False.

\item {} 
\textbf{\texttt{RoundCaps}} (\emph{\texttt{bool, optional}}) -- If set to True along with \titleref{PathOnly}, the end of the path is rounded.
Defaults to False.

\item {} 
\textbf{\texttt{Extended}} (\emph{\texttt{bool, optional}}) -- If set to True along with \titleref{PathOnly}, the end of the path is extended.
Defaults to False.

\item {} 
\textbf{\texttt{**kwargs}} -- keyword arguments passed to the {\hyperref[Chapters/PythonModuleReference/Patterns/TXLWizard.Patterns.AbstractPattern:TXLWizard.Patterns.AbstractPattern.AbstractPattern]{\crossref{\code{TXLWizard.Patterns.AbstractPattern.AbstractPattern}}}} constructor.
Can specify attributes of the current pattern.

\end{itemize}

\end{description}\end{quote}
\index{Center (TXLWizard.Patterns.Circle.Circle attribute)}

\begin{fulllineitems}
\phantomsection\label{Chapters/PythonModuleReference/Patterns/TXLWizard.Patterns.Circle:TXLWizard.Patterns.Circle.Circle.Center}\pysigline{\bfcode{Center}\strong{ = None}}
\emph{list of float} -- x- and y-coordinates specifying the center of the circle

\end{fulllineitems}

\index{EndAngle (TXLWizard.Patterns.Circle.Circle attribute)}

\begin{fulllineitems}
\phantomsection\label{Chapters/PythonModuleReference/Patterns/TXLWizard.Patterns.Circle:TXLWizard.Patterns.Circle.Circle.EndAngle}\pysigline{\bfcode{EndAngle}\strong{ = None}}
\emph{float} -- If set, only a sector is drawn from \titleref{self.StartAngle} to \titleref{self.EndAngle}.

\end{fulllineitems}

\index{EndPoint (TXLWizard.Patterns.Circle.Circle attribute)}

\begin{fulllineitems}
\phantomsection\label{Chapters/PythonModuleReference/Patterns/TXLWizard.Patterns.Circle:TXLWizard.Patterns.Circle.Circle.EndPoint}\pysigline{\bfcode{EndPoint}\strong{ = None}}
\emph{list of float} -- If \titleref{self.StartAngle} and \titleref{self.EndAngle} are set, the ending point of the segment arc is calculated

\end{fulllineitems}

\index{Extended (TXLWizard.Patterns.Circle.Circle attribute)}

\begin{fulllineitems}
\phantomsection\label{Chapters/PythonModuleReference/Patterns/TXLWizard.Patterns.Circle:TXLWizard.Patterns.Circle.Circle.Extended}\pysigline{\bfcode{Extended}\strong{ = None}}
\emph{bool} -- If set to True along with \titleref{PathOnly}, the end of the path is extended

\end{fulllineitems}

\index{NumberOfPoints (TXLWizard.Patterns.Circle.Circle attribute)}

\begin{fulllineitems}
\phantomsection\label{Chapters/PythonModuleReference/Patterns/TXLWizard.Patterns.Circle:TXLWizard.Patterns.Circle.Circle.NumberOfPoints}\pysigline{\bfcode{NumberOfPoints}\strong{ = None}}
\emph{int} -- Number of path segments used for drawing the circle.

\end{fulllineitems}

\index{PathOnly (TXLWizard.Patterns.Circle.Circle attribute)}

\begin{fulllineitems}
\phantomsection\label{Chapters/PythonModuleReference/Patterns/TXLWizard.Patterns.Circle:TXLWizard.Patterns.Circle.Circle.PathOnly}\pysigline{\bfcode{PathOnly}\strong{ = None}}
\emph{bool} -- If set to True, only the arc of the circle is drawn.

\end{fulllineitems}

\index{Radius (TXLWizard.Patterns.Circle.Circle attribute)}

\begin{fulllineitems}
\phantomsection\label{Chapters/PythonModuleReference/Patterns/TXLWizard.Patterns.Circle:TXLWizard.Patterns.Circle.Circle.Radius}\pysigline{\bfcode{Radius}\strong{ = None}}
\emph{float} -- Radius of the circle

\end{fulllineitems}

\index{RoundCaps (TXLWizard.Patterns.Circle.Circle attribute)}

\begin{fulllineitems}
\phantomsection\label{Chapters/PythonModuleReference/Patterns/TXLWizard.Patterns.Circle:TXLWizard.Patterns.Circle.Circle.RoundCaps}\pysigline{\bfcode{RoundCaps}\strong{ = None}}
\emph{bool} -- If set to True along with \titleref{PathOnly}, the end of the path is rounded

\end{fulllineitems}

\index{StartAngle (TXLWizard.Patterns.Circle.Circle attribute)}

\begin{fulllineitems}
\phantomsection\label{Chapters/PythonModuleReference/Patterns/TXLWizard.Patterns.Circle:TXLWizard.Patterns.Circle.Circle.StartAngle}\pysigline{\bfcode{StartAngle}\strong{ = None}}
\emph{float} -- If set, only a sector is drawn from \titleref{self.StartAngle} to \titleref{self.EndAngle}.

\end{fulllineitems}

\index{StartPoint (TXLWizard.Patterns.Circle.Circle attribute)}

\begin{fulllineitems}
\phantomsection\label{Chapters/PythonModuleReference/Patterns/TXLWizard.Patterns.Circle:TXLWizard.Patterns.Circle.Circle.StartPoint}\pysigline{\bfcode{StartPoint}\strong{ = None}}
\emph{list of float} -- If \titleref{self.StartAngle} and \titleref{self.EndAngle} are set, the starting point of the segment arc is calculated

\end{fulllineitems}

\index{Type (TXLWizard.Patterns.Circle.Circle attribute)}

\begin{fulllineitems}
\phantomsection\label{Chapters/PythonModuleReference/Patterns/TXLWizard.Patterns.Circle:TXLWizard.Patterns.Circle.Circle.Type}\pysigline{\bfcode{Type}\strong{ = None}}
\emph{str} -- specifies the type of the pattern. Set to `Circle'

\end{fulllineitems}


\end{fulllineitems}



\subsubsection{TXLWizard.Patterns.Ellipse}
\label{Chapters/PythonModuleReference/Patterns/TXLWizard.Patterns.Ellipse:txlwizard-patterns-ellipse}\label{Chapters/PythonModuleReference/Patterns/TXLWizard.Patterns.Ellipse:module-TXLWizard.Patterns.Ellipse}\label{Chapters/PythonModuleReference/Patterns/TXLWizard.Patterns.Ellipse::doc}\index{TXLWizard.Patterns.Ellipse (module)}
Implements a class for \titleref{Pattern} objects of type \titleref{Ellipse} (\titleref{ELP}).

Renders an ellipse.


\paragraph{Classes}
\label{Chapters/PythonModuleReference/Patterns/TXLWizard.Patterns.Ellipse:classes}
\begin{longtable}{ll}
\hline
\endfirsthead

\multicolumn{2}{c}%
{{\tablecontinued{\tablename\ \thetable{} -- continued from previous page}}} \\
\hline
\endhead

\hline \multicolumn{2}{|r|}{{\tablecontinued{Continued on next page}}} \\ \hline
\endfoot

\endlastfoot


{\hyperref[Chapters/PythonModuleReference/Patterns/TXLWizard.Patterns.Ellipse:TXLWizard.Patterns.Ellipse.Ellipse]{\crossref{\code{Ellipse}}}}(Center, RadiusX, RadiusY, **kwargs)
 & 
Implements a class for \titleref{Pattern} objects of type \titleref{Ellipse}.
\\
\hline\end{longtable}

\index{Ellipse (class in TXLWizard.Patterns.Ellipse)}

\begin{fulllineitems}
\phantomsection\label{Chapters/PythonModuleReference/Patterns/TXLWizard.Patterns.Ellipse:TXLWizard.Patterns.Ellipse.Ellipse}\pysiglinewithargsret{\strong{class }\code{TXLWizard.Patterns.Ellipse.}\bfcode{Ellipse}}{\emph{Center}, \emph{RadiusX}, \emph{RadiusY}, \emph{**kwargs}}{}
Bases: {\hyperref[Chapters/PythonModuleReference/Patterns/TXLWizard.Patterns.AbstractPattern:TXLWizard.Patterns.AbstractPattern.AbstractPattern]{\crossref{\code{TXLWizard.Patterns.AbstractPattern.AbstractPattern}}}}

Implements a class for \titleref{Pattern} objects of type \titleref{Ellipse}.

Corresponds to the TXL command \titleref{ELP}.

Renders an ellipse. Optionally, only a sector is shown when specifying \titleref{StartAngle} and \titleref{EndAngle}.

If \titleref{NumberOfPoints} is given, the number of path segments defining the ellipse can be specified.

If \titleref{PathOnly} is set to True, only the arc of the ellipse is shown.
\begin{quote}\begin{description}
\item[{Parameters}] \leavevmode\begin{itemize}
\item {} 
\textbf{\texttt{Center}} (\emph{\texttt{list of float}}) -- x- and y-coordinates specifying the center of the ellipse

\item {} 
\textbf{\texttt{RadiusX}} (\emph{\texttt{float}}) -- Semi-major axis of the ellipse in x-direction

\item {} 
\textbf{\texttt{RadiusY}} (\emph{\texttt{float}}) -- Semi-minor axis of the ellipse in y-direction

\item {} 
\textbf{\texttt{StartAngle}} (\emph{\texttt{float, optional}}) -- If given, only a sector is drawn from \titleref{StartAngle} to \titleref{EndAngle}.
Defaults to 0

\item {} 
\textbf{\texttt{EndAngle}} (\emph{\texttt{float, optional}}) -- If given, only a sector is drawn from \titleref{StartAngle} to \titleref{EndAngle}.
Defaults to 0

\item {} 
\textbf{\texttt{NumberOfPoints}} (\emph{\texttt{int, optional}}) -- Number of path segments used for drawing the ellipse.
Defaults to None.

\item {} 
\textbf{\texttt{**kwargs}} -- keyword arguments passed to the {\hyperref[Chapters/PythonModuleReference/Patterns/TXLWizard.Patterns.AbstractPattern:TXLWizard.Patterns.AbstractPattern.AbstractPattern]{\crossref{\code{TXLWizard.Patterns.AbstractPattern.AbstractPattern}}}} constructor.
Can specify attributes of the current pattern.

\end{itemize}

\end{description}\end{quote}
\index{Center (TXLWizard.Patterns.Ellipse.Ellipse attribute)}

\begin{fulllineitems}
\phantomsection\label{Chapters/PythonModuleReference/Patterns/TXLWizard.Patterns.Ellipse:TXLWizard.Patterns.Ellipse.Ellipse.Center}\pysigline{\bfcode{Center}\strong{ = None}}
\emph{list of float} -- x- and y-coordinates specifying the center of the ellipse

\end{fulllineitems}

\index{EndAngle (TXLWizard.Patterns.Ellipse.Ellipse attribute)}

\begin{fulllineitems}
\phantomsection\label{Chapters/PythonModuleReference/Patterns/TXLWizard.Patterns.Ellipse:TXLWizard.Patterns.Ellipse.Ellipse.EndAngle}\pysigline{\bfcode{EndAngle}\strong{ = None}}
\emph{float} -- If given, only a sector is drawn from \titleref{StartAngle} to \titleref{EndAngle}.

\end{fulllineitems}

\index{EndPoint (TXLWizard.Patterns.Ellipse.Ellipse attribute)}

\begin{fulllineitems}
\phantomsection\label{Chapters/PythonModuleReference/Patterns/TXLWizard.Patterns.Ellipse:TXLWizard.Patterns.Ellipse.Ellipse.EndPoint}\pysigline{\bfcode{EndPoint}\strong{ = None}}
\emph{list of float} -- If \titleref{self.StartAngle} and \titleref{self.EndAngle} are set, the ending point of the segment arc is calculated

\end{fulllineitems}

\index{NumberOfPoints (TXLWizard.Patterns.Ellipse.Ellipse attribute)}

\begin{fulllineitems}
\phantomsection\label{Chapters/PythonModuleReference/Patterns/TXLWizard.Patterns.Ellipse:TXLWizard.Patterns.Ellipse.Ellipse.NumberOfPoints}\pysigline{\bfcode{NumberOfPoints}\strong{ = None}}
\emph{int} -- Number of path segments used for drawing the ellipse.

\end{fulllineitems}

\index{RadiusX (TXLWizard.Patterns.Ellipse.Ellipse attribute)}

\begin{fulllineitems}
\phantomsection\label{Chapters/PythonModuleReference/Patterns/TXLWizard.Patterns.Ellipse:TXLWizard.Patterns.Ellipse.Ellipse.RadiusX}\pysigline{\bfcode{RadiusX}\strong{ = None}}
\emph{float} -- Semi-major axis of the ellipse in x-direction

\end{fulllineitems}

\index{RadiusY (TXLWizard.Patterns.Ellipse.Ellipse attribute)}

\begin{fulllineitems}
\phantomsection\label{Chapters/PythonModuleReference/Patterns/TXLWizard.Patterns.Ellipse:TXLWizard.Patterns.Ellipse.Ellipse.RadiusY}\pysigline{\bfcode{RadiusY}\strong{ = None}}
\emph{float} -- Semi-minor axis of the ellipse in y-direction

\end{fulllineitems}

\index{StartAngle (TXLWizard.Patterns.Ellipse.Ellipse attribute)}

\begin{fulllineitems}
\phantomsection\label{Chapters/PythonModuleReference/Patterns/TXLWizard.Patterns.Ellipse:TXLWizard.Patterns.Ellipse.Ellipse.StartAngle}\pysigline{\bfcode{StartAngle}\strong{ = None}}
\emph{float} -- If given, only a sector is drawn from \titleref{StartAngle} to \titleref{EndAngle}.

\end{fulllineitems}

\index{StartPoint (TXLWizard.Patterns.Ellipse.Ellipse attribute)}

\begin{fulllineitems}
\phantomsection\label{Chapters/PythonModuleReference/Patterns/TXLWizard.Patterns.Ellipse:TXLWizard.Patterns.Ellipse.Ellipse.StartPoint}\pysigline{\bfcode{StartPoint}\strong{ = None}}
\emph{list of float} -- If \titleref{self.StartAngle} and \titleref{self.EndAngle} are set, the starting point of the segment arc is calculated

\end{fulllineitems}

\index{Type (TXLWizard.Patterns.Ellipse.Ellipse attribute)}

\begin{fulllineitems}
\phantomsection\label{Chapters/PythonModuleReference/Patterns/TXLWizard.Patterns.Ellipse:TXLWizard.Patterns.Ellipse.Ellipse.Type}\pysigline{\bfcode{Type}\strong{ = None}}
\emph{str} -- specifies the type of the pattern. Set to `Ellipse'

\end{fulllineitems}


\end{fulllineitems}



\subsubsection{TXLWizard.Patterns.Polygon}
\label{Chapters/PythonModuleReference/Patterns/TXLWizard.Patterns.Polygon:module-TXLWizard.Patterns.Polygon}\label{Chapters/PythonModuleReference/Patterns/TXLWizard.Patterns.Polygon:txlwizard-patterns-polygon}\label{Chapters/PythonModuleReference/Patterns/TXLWizard.Patterns.Polygon::doc}\index{TXLWizard.Patterns.Polygon (module)}
Implements a class for \titleref{Pattern} objects of type \titleref{Polygon} (\titleref{B}).

Renders an polygon.


\paragraph{Classes}
\label{Chapters/PythonModuleReference/Patterns/TXLWizard.Patterns.Polygon:classes}
\begin{longtable}{ll}
\hline
\endfirsthead

\multicolumn{2}{c}%
{{\tablecontinued{\tablename\ \thetable{} -- continued from previous page}}} \\
\hline
\endhead

\hline \multicolumn{2}{|r|}{{\tablecontinued{Continued on next page}}} \\ \hline
\endfoot

\endlastfoot


{\hyperref[Chapters/PythonModuleReference/Patterns/TXLWizard.Patterns.Polygon:TXLWizard.Patterns.Polygon.Polygon]{\crossref{\code{Polygon}}}}(Points, **kwargs)
 & 
Implements a class for \titleref{Pattern} objects of type \titleref{Polygon}.
\\
\hline\end{longtable}

\index{Polygon (class in TXLWizard.Patterns.Polygon)}

\begin{fulllineitems}
\phantomsection\label{Chapters/PythonModuleReference/Patterns/TXLWizard.Patterns.Polygon:TXLWizard.Patterns.Polygon.Polygon}\pysiglinewithargsret{\strong{class }\code{TXLWizard.Patterns.Polygon.}\bfcode{Polygon}}{\emph{Points}, \emph{**kwargs}}{}
Bases: {\hyperref[Chapters/PythonModuleReference/Patterns/TXLWizard.Patterns.AbstractPattern:TXLWizard.Patterns.AbstractPattern.AbstractPattern]{\crossref{\code{TXLWizard.Patterns.AbstractPattern.AbstractPattern}}}}

Implements a class for \titleref{Pattern} objects of type \titleref{Polygon}.

Corresponds to the TXL command \titleref{B}

Renders an polygon.

The boundary is always closed so the last point connects to the starting point
\begin{quote}\begin{description}
\item[{Parameters}] \leavevmode\begin{itemize}
\item {} 
\textbf{\texttt{Points}} (\emph{\texttt{list of list of float}}) -- List of points (each point is a list of float, specifying the x- and y-coordinate of the point) that define the polygon

\item {} 
\textbf{\texttt{**kwargs}} -- keyword arguments passed to the {\hyperref[Chapters/PythonModuleReference/Patterns/TXLWizard.Patterns.AbstractPattern:TXLWizard.Patterns.AbstractPattern.AbstractPattern]{\crossref{\code{TXLWizard.Patterns.AbstractPattern.AbstractPattern}}}} constructor.
Can specify attributes of the current pattern.

\end{itemize}

\end{description}\end{quote}
\index{Points (TXLWizard.Patterns.Polygon.Polygon attribute)}

\begin{fulllineitems}
\phantomsection\label{Chapters/PythonModuleReference/Patterns/TXLWizard.Patterns.Polygon:TXLWizard.Patterns.Polygon.Polygon.Points}\pysigline{\bfcode{Points}\strong{ = None}}
\emph{list of list of float} -- List of points (each point is a list of float, specifying the x- and y-coordinate of the point) that define the polygon

\end{fulllineitems}

\index{Type (TXLWizard.Patterns.Polygon.Polygon attribute)}

\begin{fulllineitems}
\phantomsection\label{Chapters/PythonModuleReference/Patterns/TXLWizard.Patterns.Polygon:TXLWizard.Patterns.Polygon.Polygon.Type}\pysigline{\bfcode{Type}\strong{ = None}}
\emph{str} -- specifies the type of the pattern. Set to `Polygon'

\end{fulllineitems}


\end{fulllineitems}



\subsubsection{TXLWizard.Patterns.Polyline}
\label{Chapters/PythonModuleReference/Patterns/TXLWizard.Patterns.Polyline:module-TXLWizard.Patterns.Polyline}\label{Chapters/PythonModuleReference/Patterns/TXLWizard.Patterns.Polyline::doc}\label{Chapters/PythonModuleReference/Patterns/TXLWizard.Patterns.Polyline:txlwizard-patterns-polyline}\index{TXLWizard.Patterns.Polyline (module)}
Implements a class for \titleref{Pattern} objects of type \titleref{Polyline} (\titleref{B}).

Renders an path specified by points.


\paragraph{Classes}
\label{Chapters/PythonModuleReference/Patterns/TXLWizard.Patterns.Polyline:classes}
\begin{longtable}{ll}
\hline
\endfirsthead

\multicolumn{2}{c}%
{{\tablecontinued{\tablename\ \thetable{} -- continued from previous page}}} \\
\hline
\endhead

\hline \multicolumn{2}{|r|}{{\tablecontinued{Continued on next page}}} \\ \hline
\endfoot

\endlastfoot


{\hyperref[Chapters/PythonModuleReference/Patterns/TXLWizard.Patterns.Polyline:TXLWizard.Patterns.Polyline.Polyline]{\crossref{\code{Polyline}}}}(Points, **kwargs)
 & 
Implements a class for \titleref{Pattern} objects of type \titleref{Polyline}.
\\
\hline\end{longtable}

\index{Polyline (class in TXLWizard.Patterns.Polyline)}

\begin{fulllineitems}
\phantomsection\label{Chapters/PythonModuleReference/Patterns/TXLWizard.Patterns.Polyline:TXLWizard.Patterns.Polyline.Polyline}\pysiglinewithargsret{\strong{class }\code{TXLWizard.Patterns.Polyline.}\bfcode{Polyline}}{\emph{Points}, \emph{**kwargs}}{}
Bases: {\hyperref[Chapters/PythonModuleReference/Patterns/TXLWizard.Patterns.AbstractPattern:TXLWizard.Patterns.AbstractPattern.AbstractPattern]{\crossref{\code{TXLWizard.Patterns.AbstractPattern.AbstractPattern}}}}

Implements a class for \titleref{Pattern} objects of type \titleref{Polyline}.

Corresponds to the TXL command \titleref{P} (\titleref{PR} if \titleref{RoundCaps} is True).

Renders an path specified by points.

The ends can be rounded by specifying \titleref{RoundCaps}
\begin{quote}\begin{description}
\item[{Parameters}] \leavevmode\begin{itemize}
\item {} 
\textbf{\texttt{Points}} (\emph{\texttt{list of list of float}}) -- List of points (each point is a list of float, specifying the x- and y-coordinate of the point) that define the path

\item {} 
\textbf{\texttt{RoundCaps}} (\emph{\texttt{bool, optional}}) -- If set to True, the end of the path is rounded.
Defaults to False.

\item {} 
\textbf{\texttt{**kwargs}} -- keyword arguments passed to the {\hyperref[Chapters/PythonModuleReference/Patterns/TXLWizard.Patterns.AbstractPattern:TXLWizard.Patterns.AbstractPattern.AbstractPattern]{\crossref{\code{TXLWizard.Patterns.AbstractPattern.AbstractPattern}}}} constructor.
Can specify attributes of the current pattern.

\end{itemize}

\end{description}\end{quote}
\index{Points (TXLWizard.Patterns.Polyline.Polyline attribute)}

\begin{fulllineitems}
\phantomsection\label{Chapters/PythonModuleReference/Patterns/TXLWizard.Patterns.Polyline:TXLWizard.Patterns.Polyline.Polyline.Points}\pysigline{\bfcode{Points}\strong{ = None}}
\emph{list of list of float} -- List of points (each point is a list of float, specifying the x- and y-coordinate of the point) that define the polygon

\end{fulllineitems}

\index{RoundCaps (TXLWizard.Patterns.Polyline.Polyline attribute)}

\begin{fulllineitems}
\phantomsection\label{Chapters/PythonModuleReference/Patterns/TXLWizard.Patterns.Polyline:TXLWizard.Patterns.Polyline.Polyline.RoundCaps}\pysigline{\bfcode{RoundCaps}\strong{ = None}}
\emph{bool} -- If set to True, the end of the path is rounded

\end{fulllineitems}

\index{Type (TXLWizard.Patterns.Polyline.Polyline attribute)}

\begin{fulllineitems}
\phantomsection\label{Chapters/PythonModuleReference/Patterns/TXLWizard.Patterns.Polyline:TXLWizard.Patterns.Polyline.Polyline.Type}\pysigline{\bfcode{Type}\strong{ = None}}
\emph{str} -- specifies the type of the pattern. Set to `Polyline'

\end{fulllineitems}


\end{fulllineitems}



\subsubsection{TXLWizard.Patterns.Reference}
\label{Chapters/PythonModuleReference/Patterns/TXLWizard.Patterns.Reference:txlwizard-patterns-reference}\label{Chapters/PythonModuleReference/Patterns/TXLWizard.Patterns.Reference:module-TXLWizard.Patterns.Reference}\label{Chapters/PythonModuleReference/Patterns/TXLWizard.Patterns.Reference::doc}\index{TXLWizard.Patterns.Reference (module)}
Implements a class for \titleref{Pattern} objects of type \titleref{Reference} (\titleref{SREF}).

Renders a copy of the referenced structure.


\paragraph{Classes}
\label{Chapters/PythonModuleReference/Patterns/TXLWizard.Patterns.Reference:classes}
\begin{longtable}{ll}
\hline
\endfirsthead

\multicolumn{2}{c}%
{{\tablecontinued{\tablename\ \thetable{} -- continued from previous page}}} \\
\hline
\endhead

\hline \multicolumn{2}{|r|}{{\tablecontinued{Continued on next page}}} \\ \hline
\endfoot

\endlastfoot


{\hyperref[Chapters/PythonModuleReference/Patterns/TXLWizard.Patterns.Reference:TXLWizard.Patterns.Reference.Reference]{\crossref{\code{Reference}}}}(ReferencedStructureID, ...)
 & 
Implements a class for \titleref{Pattern} objects of type \titleref{Reference}.
\\
\hline\end{longtable}

\index{Reference (class in TXLWizard.Patterns.Reference)}

\begin{fulllineitems}
\phantomsection\label{Chapters/PythonModuleReference/Patterns/TXLWizard.Patterns.Reference:TXLWizard.Patterns.Reference.Reference}\pysiglinewithargsret{\strong{class }\code{TXLWizard.Patterns.Reference.}\bfcode{Reference}}{\emph{ReferencedStructureID}, \emph{OriginPoint}, \emph{**kwargs}}{}
Bases: {\hyperref[Chapters/PythonModuleReference/Patterns/TXLWizard.Patterns.AbstractPattern:TXLWizard.Patterns.AbstractPattern.AbstractPattern]{\crossref{\code{TXLWizard.Patterns.AbstractPattern.AbstractPattern}}}}

Implements a class for \titleref{Pattern} objects of type \titleref{Reference}.

Corresponds to the TXL command \titleref{SREF}.

Renders a copy of the structure identified by \titleref{ReferencedStructureID} at \titleref{OriginPoint}.
\begin{quote}\begin{description}
\item[{Parameters}] \leavevmode\begin{itemize}
\item {} 
\textbf{\texttt{ReferencedStructureID}} (\emph{\texttt{str}}) -- ID of the structure being referenced to

\item {} 
\textbf{\texttt{OriginPoint}} (\emph{\texttt{list of float}}) -- x- and y-coordinates of the starting point

\item {} 
\textbf{\texttt{**kwargs}} -- keyword arguments passed to the {\hyperref[Chapters/PythonModuleReference/Patterns/TXLWizard.Patterns.AbstractPattern:TXLWizard.Patterns.AbstractPattern.AbstractPattern]{\crossref{\code{TXLWizard.Patterns.AbstractPattern.AbstractPattern}}}} constructor.
Can specify attributes of the current pattern.

\end{itemize}

\end{description}\end{quote}
\index{ReferencedStructureID (TXLWizard.Patterns.Reference.Reference attribute)}

\begin{fulllineitems}
\phantomsection\label{Chapters/PythonModuleReference/Patterns/TXLWizard.Patterns.Reference:TXLWizard.Patterns.Reference.Reference.ReferencedStructureID}\pysigline{\bfcode{ReferencedStructureID}\strong{ = None}}
\emph{str} -- ID of the structure being referenced to

\end{fulllineitems}

\index{Type (TXLWizard.Patterns.Reference.Reference attribute)}

\begin{fulllineitems}
\phantomsection\label{Chapters/PythonModuleReference/Patterns/TXLWizard.Patterns.Reference:TXLWizard.Patterns.Reference.Reference.Type}\pysigline{\bfcode{Type}\strong{ = None}}
\emph{str} -- specifies the type of the pattern. Set to `Reference'

\end{fulllineitems}


\end{fulllineitems}



\subsubsection{TXLWizard.Patterns.Structure}
\label{Chapters/PythonModuleReference/Patterns/TXLWizard.Patterns.Structure:module-TXLWizard.Patterns.Structure}\label{Chapters/PythonModuleReference/Patterns/TXLWizard.Patterns.Structure::doc}\label{Chapters/PythonModuleReference/Patterns/TXLWizard.Patterns.Structure:txlwizard-patterns-structure}\index{TXLWizard.Patterns.Structure (module)}
Implements a class for \titleref{Structure} objects (\titleref{STRUCT}).

A \titleref{Structure} is a container for \titleref{Pattern} objects.


\paragraph{Classes}
\label{Chapters/PythonModuleReference/Patterns/TXLWizard.Patterns.Structure:classes}
\begin{longtable}{ll}
\hline
\endfirsthead

\multicolumn{2}{c}%
{{\tablecontinued{\tablename\ \thetable{} -- continued from previous page}}} \\
\hline
\endhead

\hline \multicolumn{2}{|r|}{{\tablecontinued{Continued on next page}}} \\ \hline
\endfoot

\endlastfoot


{\hyperref[Chapters/PythonModuleReference/Patterns/TXLWizard.Patterns.Structure:TXLWizard.Patterns.Structure.Structure]{\crossref{\code{Structure}}}}(ID, **kwargs)
 & 
Implements a class for \titleref{Structure} objects.
\\
\hline\end{longtable}

\index{Structure (class in TXLWizard.Patterns.Structure)}

\begin{fulllineitems}
\phantomsection\label{Chapters/PythonModuleReference/Patterns/TXLWizard.Patterns.Structure:TXLWizard.Patterns.Structure.Structure}\pysiglinewithargsret{\strong{class }\code{TXLWizard.Patterns.Structure.}\bfcode{Structure}}{\emph{ID}, \emph{**kwargs}}{}
Bases: {\hyperref[Chapters/PythonModuleReference/Patterns/TXLWizard.Patterns.AbstractPattern:TXLWizard.Patterns.AbstractPattern.AbstractPattern]{\crossref{\code{TXLWizard.Patterns.AbstractPattern.AbstractPattern}}}}

Implements a class for \titleref{Structure} objects.

Corresponds to the TXL command \titleref{STRUCT}.

A \titleref{Structure} is a container for \titleref{Pattern} objects.
\begin{quote}\begin{description}
\item[{Parameters}] \leavevmode\begin{itemize}
\item {} 
\textbf{\texttt{ID}} (\emph{\texttt{str}}) -- Unique identification of the structure. Also used when referencing to this structure.

\item {} 
\textbf{\texttt{TXLOutput}} (\emph{\texttt{bool, optional}}) -- If set to False, the TXL Output is suppressed.
Defaults to True

\item {} 
\textbf{\texttt{**kwargs}} -- keyword arguments passed to the {\hyperref[Chapters/PythonModuleReference/Patterns/TXLWizard.Patterns.AbstractPattern:TXLWizard.Patterns.AbstractPattern.AbstractPattern]{\crossref{\code{TXLWizard.Patterns.AbstractPattern.AbstractPattern}}}} constructor.
Can specify attributes of the current pattern.

\end{itemize}

\end{description}\end{quote}
\index{AddPattern() (TXLWizard.Patterns.Structure.Structure method)}

\begin{fulllineitems}
\phantomsection\label{Chapters/PythonModuleReference/Patterns/TXLWizard.Patterns.Structure:TXLWizard.Patterns.Structure.Structure.AddPattern}\pysiglinewithargsret{\bfcode{AddPattern}}{\emph{PatternType}, \emph{**kwargs}}{}
Adds a \titleref{Pattern} of type \titleref{PatternType} to the structure.
Creates an instance of \titleref{TXLWizard.Patterns.\{PatternType\}.\{PatternType\}}.
The \titleref{kwargs} are passed to the corresponding constructor and allow specifying
pattern parameters as defined in the constructor of the corresponding pattern class
and attributes as defined in {\hyperref[Chapters/PythonModuleReference/Patterns/TXLWizard.Patterns.AbstractPattern:TXLWizard.Patterns.AbstractPattern.AbstractPattern]{\crossref{\code{TXLWizard.Patterns.AbstractPattern.AbstractPattern}}}}.
\begin{quote}\begin{description}
\item[{Parameters}] \leavevmode\begin{itemize}
\item {} 
\textbf{\texttt{PatternType}} (\emph{\texttt{\{'Array', 'Circle', 'Ellipse', 'Polygon', 'Polyline', 'Reference'\}}}) -- Type of the pattern to be added.

\item {} 
\textbf{\texttt{**kwargs}} -- keyword arguments are passed to the corresponding constructor and allow specifying
pattern parameters as defined in the constructor of the corresponding pattern class
and attributes as defined in {\hyperref[Chapters/PythonModuleReference/Patterns/TXLWizard.Patterns.AbstractPattern:TXLWizard.Patterns.AbstractPattern.AbstractPattern]{\crossref{\code{TXLWizard.Patterns.AbstractPattern.AbstractPattern}}}}.

\end{itemize}

\item[{Returns}] \leavevmode
returns the created pattern object

\item[{Return type}] \leavevmode
\code{TXLWizard.Patterns.\{PatternType\}.\{PatternType\}}

\end{description}\end{quote}

\end{fulllineitems}

\index{CurrentAttributes (TXLWizard.Patterns.Structure.Structure attribute)}

\begin{fulllineitems}
\phantomsection\label{Chapters/PythonModuleReference/Patterns/TXLWizard.Patterns.Structure:TXLWizard.Patterns.Structure.Structure.CurrentAttributes}\pysigline{\bfcode{CurrentAttributes}\strong{ = None}}
\emph{dict} -- attribute values of the next pattern to be added. Default values are copied from \titleref{self.DefaultAttributes}

\end{fulllineitems}

\index{ID (TXLWizard.Patterns.Structure.Structure attribute)}

\begin{fulllineitems}
\phantomsection\label{Chapters/PythonModuleReference/Patterns/TXLWizard.Patterns.Structure:TXLWizard.Patterns.Structure.Structure.ID}\pysigline{\bfcode{ID}\strong{ = None}}
\emph{str} -- Unique identification of the structure. Also used when referencing to this structure.

\end{fulllineitems}

\index{Patterns (TXLWizard.Patterns.Structure.Structure attribute)}

\begin{fulllineitems}
\phantomsection\label{Chapters/PythonModuleReference/Patterns/TXLWizard.Patterns.Structure:TXLWizard.Patterns.Structure.Structure.Patterns}\pysigline{\bfcode{Patterns}\strong{ = None}}
list of {\hyperref[Chapters/PythonModuleReference/Patterns/TXLWizard.Patterns.AbstractPattern:TXLWizard.Patterns.AbstractPattern.AbstractPattern]{\crossref{\code{TXLWizard.Patterns.AbstractPattern.AbstractPattern}}}} -- Patterns that are contained in this structure

\end{fulllineitems}

\index{TXLOutput (TXLWizard.Patterns.Structure.Structure attribute)}

\begin{fulllineitems}
\phantomsection\label{Chapters/PythonModuleReference/Patterns/TXLWizard.Patterns.Structure:TXLWizard.Patterns.Structure.Structure.TXLOutput}\pysigline{\bfcode{TXLOutput}\strong{ = None}}
\emph{bool} -- If set to False, the TXL Output is suppressed.

\end{fulllineitems}

\index{Type (TXLWizard.Patterns.Structure.Structure attribute)}

\begin{fulllineitems}
\phantomsection\label{Chapters/PythonModuleReference/Patterns/TXLWizard.Patterns.Structure:TXLWizard.Patterns.Structure.Structure.Type}\pysigline{\bfcode{Type}\strong{ = None}}
\emph{str} -- specifies the type of the pattern. Set to `Structure'

\end{fulllineitems}


\end{fulllineitems}



\subsection{Shape Library}
\label{Chapters/40_PythonModuleReference:shape-library}\label{Chapters/40_PythonModuleReference:pythonmodulereferenceshapelibrary}
\begin{longtable}{ll}
\hline
\endfirsthead

\multicolumn{2}{c}%
{{\tablecontinued{\tablename\ \thetable{} -- continued from previous page}}} \\
\hline
\endhead

\hline \multicolumn{2}{|r|}{{\tablecontinued{Continued on next page}}} \\ \hline
\endfoot

\endlastfoot


{\hyperref[Chapters/PythonModuleReference/ShapeLibrary/TXLWizard.ShapeLibrary.Label:module\string-TXLWizard.ShapeLibrary.Label]{\crossref{\code{TXLWizard.ShapeLibrary.Label}}}}
 & 
Renders arbitrary text in \titleref{TXLWriter}.
\\
\hline
{\hyperref[Chapters/PythonModuleReference/ShapeLibrary/TXLWizard.ShapeLibrary.EndpointDetectionWindows:module\string-TXLWizard.ShapeLibrary.EndpointDetectionWindows]{\crossref{\code{TXLWizard.ShapeLibrary.EndpointDetectionWindows}}}}
 & 
Add five squares to \titleref{TXLWriter} that can be used as endpoint detection windows.
\\
\hline
{\hyperref[Chapters/PythonModuleReference/ShapeLibrary/TXLWizard.ShapeLibrary.AlignmentMarkers:module\string-TXLWizard.ShapeLibrary.AlignmentMarkers]{\crossref{\code{TXLWizard.ShapeLibrary.AlignmentMarkers}}}}
 & 
Add squares to \titleref{TXLWriter} that can be used as alignment markers.
\\
\hline\end{longtable}



\subsubsection{TXLWizard.ShapeLibrary.Label}
\label{Chapters/PythonModuleReference/ShapeLibrary/TXLWizard.ShapeLibrary.Label:txlwizard-shapelibrary-label}\label{Chapters/PythonModuleReference/ShapeLibrary/TXLWizard.ShapeLibrary.Label::doc}\label{Chapters/PythonModuleReference/ShapeLibrary/TXLWizard.ShapeLibrary.Label:module-TXLWizard.ShapeLibrary.Label}\index{TXLWizard.ShapeLibrary.Label (module)}
Renders arbitrary text in \titleref{TXLWriter}.


\paragraph{Functions}
\label{Chapters/PythonModuleReference/ShapeLibrary/TXLWizard.ShapeLibrary.Label:functions}
\begin{longtable}{ll}
\hline
\endfirsthead

\multicolumn{2}{c}%
{{\tablecontinued{\tablename\ \thetable{} -- continued from previous page}}} \\
\hline
\endhead

\hline \multicolumn{2}{|r|}{{\tablecontinued{Continued on next page}}} \\ \hline
\endfoot

\endlastfoot


{\hyperref[Chapters/PythonModuleReference/ShapeLibrary/TXLWizard.ShapeLibrary.Label:TXLWizard.ShapeLibrary.Label.GetLabel]{\crossref{\code{GetLabel}}}}(TXLWriter, Text{[}, OriginPoint, ...{]})
 & 
Renders arbitrary text.
\\
\hline\end{longtable}

\index{GetLabel() (in module TXLWizard.ShapeLibrary.Label)}

\begin{fulllineitems}
\phantomsection\label{Chapters/PythonModuleReference/ShapeLibrary/TXLWizard.ShapeLibrary.Label:TXLWizard.ShapeLibrary.Label.GetLabel}\pysiglinewithargsret{\code{TXLWizard.ShapeLibrary.Label.}\bfcode{GetLabel}}{\emph{TXLWriter, Text, OriginPoint={[}0, 0{]}, FontSize=100, StrokeWidth=10, RotationAngle=0, FillCharacters=True, RoundCaps=False, Layer=1, **kwargs}}{}
Renders arbitrary text.
Will have an automatically generated ID.
\begin{quote}\begin{description}
\item[{Parameters}] \leavevmode\begin{itemize}
\item {} 
\textbf{\texttt{TXLWriter}} ({\hyperref[Chapters/PythonModuleReference/TXLWriter/TXLWizard.TXLWriter:TXLWizard.TXLWriter.TXLWriter]{\crossref{\code{TXLWizard.TXLWriter.TXLWriter}}}}) -- Current Instance of {\hyperref[Chapters/PythonModuleReference/TXLWriter/TXLWizard.TXLWriter:TXLWizard.TXLWriter.TXLWriter]{\crossref{\code{TXLWizard.TXLWriter.TXLWriter}}}}

\item {} 
\textbf{\texttt{Text}} (\emph{\texttt{str}}) -- Text to be displayed

\item {} 
\textbf{\texttt{OriginPoint}} (\emph{\texttt{list of float, optional}}) -- x- and y-coordinates of the origin point of the label.
Defaults to {[}0,0{]}

\item {} 
\textbf{\texttt{FontSize}} (\emph{\texttt{float, optional}}) -- Font size. Character height = font size.
Defaults to 100

\item {} 
\textbf{\texttt{StrokeWidth}} (\emph{\texttt{float}}) -- line thickness of the letters.
Defaults to 10

\item {} 
\textbf{\texttt{RotationAngle}} (\emph{\texttt{float}}) -- Angle by which the text is rotated.
Defaults to 0

\item {} 
\textbf{\texttt{FillCharacters}} (\emph{\texttt{bool, optional}}) -- If set to True, closed boundaries will be filled.
Can be useful if there should be no free-standing parts.
Defaults to True

\item {} 
\textbf{\texttt{RoundCaps}} (\emph{\texttt{bool, optional}}) -- If set to True, the paths will habe rounded ends. Should be set to False for better e-Beam Performance
Defaults to False.

\item {} 
\textbf{\texttt{Layer}} (\emph{\texttt{int, optional}}) -- Layer the text should be rendered in.
Defaults to 1

\item {} 
\textbf{\texttt{**kwargs}} -- keyword arguments

\end{itemize}

\item[{Returns}] \leavevmode
\titleref{Structure} object containing the patterns representing the text

\item[{Return type}] \leavevmode
{\hyperref[Chapters/PythonModuleReference/Patterns/TXLWizard.Patterns.Structure:TXLWizard.Patterns.Structure.Structure]{\crossref{\code{TXLWizard.Patterns.Structure.Structure}}}}

\end{description}\end{quote}

\end{fulllineitems}



\subsubsection{TXLWizard.ShapeLibrary.EndpointDetectionWindows}
\label{Chapters/PythonModuleReference/ShapeLibrary/TXLWizard.ShapeLibrary.EndpointDetectionWindows:module-TXLWizard.ShapeLibrary.EndpointDetectionWindows}\label{Chapters/PythonModuleReference/ShapeLibrary/TXLWizard.ShapeLibrary.EndpointDetectionWindows::doc}\label{Chapters/PythonModuleReference/ShapeLibrary/TXLWizard.ShapeLibrary.EndpointDetectionWindows:txlwizard-shapelibrary-endpointdetectionwindows}\index{TXLWizard.ShapeLibrary.EndpointDetectionWindows (module)}
Add five squares to \titleref{TXLWriter} that can be used as endpoint detection windows.


\paragraph{Functions}
\label{Chapters/PythonModuleReference/ShapeLibrary/TXLWizard.ShapeLibrary.EndpointDetectionWindows:functions}
\begin{longtable}{ll}
\hline
\endfirsthead

\multicolumn{2}{c}%
{{\tablecontinued{\tablename\ \thetable{} -- continued from previous page}}} \\
\hline
\endhead

\hline \multicolumn{2}{|r|}{{\tablecontinued{Continued on next page}}} \\ \hline
\endfoot

\endlastfoot


{\hyperref[Chapters/PythonModuleReference/ShapeLibrary/TXLWizard.ShapeLibrary.EndpointDetectionWindows:TXLWizard.ShapeLibrary.EndpointDetectionWindows.GetEndpointDetectionWindows]{\crossref{\code{GetEndpointDetectionWindows}}}}(TXLWriter{[}, ...{]})
 & 
Add five squares that can be used as endpoint detection windows.
\\
\hline\end{longtable}

\index{GetEndpointDetectionWindows() (in module TXLWizard.ShapeLibrary.EndpointDetectionWindows)}

\begin{fulllineitems}
\phantomsection\label{Chapters/PythonModuleReference/ShapeLibrary/TXLWizard.ShapeLibrary.EndpointDetectionWindows:TXLWizard.ShapeLibrary.EndpointDetectionWindows.GetEndpointDetectionWindows}\pysiglinewithargsret{\code{TXLWizard.ShapeLibrary.EndpointDetectionWindows.}\bfcode{GetEndpointDetectionWindows}}{\emph{TXLWriter}, \emph{SizeLarge=1000}, \emph{SizeSmall=750}, \emph{Offset=1500}, \emph{Layer=1}}{}
Add five squares that can be used as endpoint detection windows.
The first square of size \titleref{SizeLarge} will be placed in the center.
The second to fifth square of size \titleref{SizeSmall} will be placed at x / y = +-\titleref{Offset} / +-\titleref{Offset}
\begin{quote}\begin{description}
\item[{Parameters}] \leavevmode\begin{itemize}
\item {} 
\textbf{\texttt{TXLWriter}} ({\hyperref[Chapters/PythonModuleReference/TXLWriter/TXLWizard.TXLWriter:TXLWizard.TXLWriter.TXLWriter]{\crossref{\code{TXLWizard.TXLWriter.TXLWriter}}}}) -- Current Instance of {\hyperref[Chapters/PythonModuleReference/TXLWriter/TXLWizard.TXLWriter:TXLWizard.TXLWriter.TXLWriter]{\crossref{\code{TXLWizard.TXLWriter.TXLWriter}}}}

\item {} 
\textbf{\texttt{SizeLarge}} (\emph{\texttt{float, optional}}) -- Size of the center square.
Defaults to 1000

\item {} 
\textbf{\texttt{SizeSmall}} (\emph{\texttt{float, optional}}) -- Size of the four peripheral square.
Defaults to 750

\item {} 
\textbf{\texttt{Offset}} (\emph{\texttt{float, optional}}) -- Offset of the peripheral squares to the center.
Defaults to 1500

\item {} 
\textbf{\texttt{Layer}} (\emph{\texttt{int, optional}}) -- Layer the pattern should be rendered in.
Defaults to 1

\end{itemize}

\item[{Returns}] \leavevmode
\titleref{Structure} object containing the patterns representing the endpoint detection windows

\item[{Return type}] \leavevmode
{\hyperref[Chapters/PythonModuleReference/Patterns/TXLWizard.Patterns.Structure:TXLWizard.Patterns.Structure.Structure]{\crossref{\code{TXLWizard.Patterns.Structure.Structure}}}}

\end{description}\end{quote}

\end{fulllineitems}



\subsubsection{TXLWizard.ShapeLibrary.AlignmentMarkers}
\label{Chapters/PythonModuleReference/ShapeLibrary/TXLWizard.ShapeLibrary.AlignmentMarkers:txlwizard-shapelibrary-alignmentmarkers}\label{Chapters/PythonModuleReference/ShapeLibrary/TXLWizard.ShapeLibrary.AlignmentMarkers:module-TXLWizard.ShapeLibrary.AlignmentMarkers}\label{Chapters/PythonModuleReference/ShapeLibrary/TXLWizard.ShapeLibrary.AlignmentMarkers::doc}\index{TXLWizard.ShapeLibrary.AlignmentMarkers (module)}
Add squares to \titleref{TXLWriter} that can be used as alignment markers.


\paragraph{Functions}
\label{Chapters/PythonModuleReference/ShapeLibrary/TXLWizard.ShapeLibrary.AlignmentMarkers:functions}
\begin{longtable}{ll}
\hline
\endfirsthead

\multicolumn{2}{c}%
{{\tablecontinued{\tablename\ \thetable{} -- continued from previous page}}} \\
\hline
\endhead

\hline \multicolumn{2}{|r|}{{\tablecontinued{Continued on next page}}} \\ \hline
\endfoot

\endlastfoot


{\hyperref[Chapters/PythonModuleReference/ShapeLibrary/TXLWizard.ShapeLibrary.AlignmentMarkers:TXLWizard.ShapeLibrary.AlignmentMarkers.GetAlignmentMarkers]{\crossref{\code{GetAlignmentMarkers}}}}(TXLWriter{[}, Size, ...{]})
 & 
Add squares that can be used as alignment markers
\\
\hline\end{longtable}

\index{GetAlignmentMarkers() (in module TXLWizard.ShapeLibrary.AlignmentMarkers)}

\begin{fulllineitems}
\phantomsection\label{Chapters/PythonModuleReference/ShapeLibrary/TXLWizard.ShapeLibrary.AlignmentMarkers:TXLWizard.ShapeLibrary.AlignmentMarkers.GetAlignmentMarkers}\pysiglinewithargsret{\code{TXLWizard.ShapeLibrary.AlignmentMarkers.}\bfcode{GetAlignmentMarkers}}{\emph{TXLWriter}, \emph{Size=10}, \emph{OffsetSmall=750}, \emph{OffsetLarge=1500}, \emph{Layer=1}}{}
Add squares that can be used as alignment markers
\begin{quote}\begin{description}
\item[{Parameters}] \leavevmode\begin{itemize}
\item {} 
\textbf{\texttt{TXLWriter}} ({\hyperref[Chapters/PythonModuleReference/TXLWriter/TXLWizard.TXLWriter:TXLWizard.TXLWriter.TXLWriter]{\crossref{\code{TXLWizard.TXLWriter.TXLWriter}}}}) -- Current Instance of {\hyperref[Chapters/PythonModuleReference/TXLWriter/TXLWizard.TXLWriter:TXLWizard.TXLWriter.TXLWriter]{\crossref{\code{TXLWizard.TXLWriter.TXLWriter}}}}

\item {} 
\textbf{\texttt{Size}} (\emph{\texttt{float, optional}}) -- Size of the markers.
Defaults to 10

\item {} 
\textbf{\texttt{OffsetSmall}} (\emph{\texttt{float, optional}}) -- first offset from center.
Defaults to 750

\item {} 
\textbf{\texttt{OffsetLarge}} (\emph{\texttt{float, optional}}) -- second offset from center.
Defaults to 1500

\item {} 
\textbf{\texttt{Layer}} (\emph{\texttt{int, optional}}) -- Layer the pattern should be rendered in.
Defaults to 1

\end{itemize}

\item[{Returns}] \leavevmode
\titleref{Structure} object containing the patterns representing the alignment markers

\item[{Return type}] \leavevmode
{\hyperref[Chapters/PythonModuleReference/Patterns/TXLWizard.Patterns.Structure:TXLWizard.Patterns.Structure.Structure]{\crossref{\code{TXLWizard.Patterns.Structure.Structure}}}}

\end{description}\end{quote}

\end{fulllineitems}



\subsection{TXLConverter}
\label{Chapters/40_PythonModuleReference:txlconverter}
\begin{longtable}{ll}
\hline
\endfirsthead

\multicolumn{2}{c}%
{{\tablecontinued{\tablename\ \thetable{} -- continued from previous page}}} \\
\hline
\endhead

\hline \multicolumn{2}{|r|}{{\tablecontinued{Continued on next page}}} \\ \hline
\endfoot

\endlastfoot


{\hyperref[Chapters/PythonModuleReference/TXLConverter/TXLWizard.TXLConverter:module\string-TXLWizard.TXLConverter]{\crossref{\code{TXLWizard.TXLConverter}}}}
 & 
Class for parsing TXL files and converting them to html / svg using {\hyperref[Chapters/PythonModuleReference/TXLWriter/TXLWizard.TXLWriter:module\string-TXLWizard.TXLWriter]{\crossref{\code{TXLWizard.TXLWriter}}}}
\\
\hline
{\hyperref[Chapters/PythonModuleReference/TXLConverter/TXLWizard.TXLConverterCLI:module\string-TXLWizard.TXLConverterCLI]{\crossref{\code{TXLWizard.TXLConverterCLI}}}}
 & 
Provides a command line interface for the {\hyperref[Chapters/PythonModuleReference/TXLConverter/TXLWizard.TXLConverter:TXLWizard.TXLConverter.TXLConverter]{\crossref{\code{TXLWizard.TXLConverter.TXLConverter}}}} class.
\\
\hline\end{longtable}



\subsubsection{TXLWizard.TXLConverter}
\label{Chapters/PythonModuleReference/TXLConverter/TXLWizard.TXLConverter:txlwizard-txlconverter}\label{Chapters/PythonModuleReference/TXLConverter/TXLWizard.TXLConverter::doc}\label{Chapters/PythonModuleReference/TXLConverter/TXLWizard.TXLConverter:module-TXLWizard.TXLConverter}\index{TXLWizard.TXLConverter (module)}
Class for parsing TXL files and converting them to html / svg using {\hyperref[Chapters/PythonModuleReference/TXLWriter/TXLWizard.TXLWriter:module\string-TXLWizard.TXLWriter]{\crossref{\code{TXLWizard.TXLWriter}}}}


\paragraph{Classes}
\label{Chapters/PythonModuleReference/TXLConverter/TXLWizard.TXLConverter:classes}
\begin{longtable}{ll}
\hline
\endfirsthead

\multicolumn{2}{c}%
{{\tablecontinued{\tablename\ \thetable{} -- continued from previous page}}} \\
\hline
\endhead

\hline \multicolumn{2}{|r|}{{\tablecontinued{Continued on next page}}} \\ \hline
\endfoot

\endlastfoot


{\hyperref[Chapters/PythonModuleReference/TXLConverter/TXLWizard.TXLConverter:TXLWizard.TXLConverter.TXLConverter]{\crossref{\code{TXLConverter}}}}(Filename, **kwargs)
 & 
Class for parsing TXL files and converting them to
\\
\hline\end{longtable}

\index{TXLConverter (class in TXLWizard.TXLConverter)}

\begin{fulllineitems}
\phantomsection\label{Chapters/PythonModuleReference/TXLConverter/TXLWizard.TXLConverter:TXLWizard.TXLConverter.TXLConverter}\pysiglinewithargsret{\strong{class }\code{TXLWizard.TXLConverter.}\bfcode{TXLConverter}}{\emph{Filename}, \emph{**kwargs}}{}
Bases: \code{object}

Class for parsing TXL files and converting them to
html / svg using {\hyperref[Chapters/PythonModuleReference/TXLWriter/TXLWizard.TXLWriter:module\string-TXLWizard.TXLWriter]{\crossref{\code{TXLWizard.TXLWriter}}}}
\begin{quote}\begin{description}
\item[{Parameters}] \leavevmode\begin{itemize}
\item {} 
\textbf{\texttt{Filename}} (\emph{\texttt{str}}) -- Path / Filename of the .txl file

\item {} 
\textbf{\texttt{LayersToProcess}} (\emph{\texttt{list of int, optional}}) -- if given, only layers in this list are processed / shown

\end{itemize}

\end{description}\end{quote}

\end{fulllineitems}



\subsubsection{TXLWizard.TXLConverterCLI}
\label{Chapters/PythonModuleReference/TXLConverter/TXLWizard.TXLConverterCLI::doc}\label{Chapters/PythonModuleReference/TXLConverter/TXLWizard.TXLConverterCLI:txlwizard-txlconvertercli}\label{Chapters/PythonModuleReference/TXLConverter/TXLWizard.TXLConverterCLI:module-TXLWizard.TXLConverterCLI}\index{TXLWizard.TXLConverterCLI (module)}
Provides a command line interface for the {\hyperref[Chapters/PythonModuleReference/TXLConverter/TXLWizard.TXLConverter:TXLWizard.TXLConverter.TXLConverter]{\crossref{\code{TXLWizard.TXLConverter.TXLConverter}}}} class.


\paragraph{Classes}
\label{Chapters/PythonModuleReference/TXLConverter/TXLWizard.TXLConverterCLI:classes}
\begin{longtable}{ll}
\hline
\endfirsthead

\multicolumn{2}{c}%
{{\tablecontinued{\tablename\ \thetable{} -- continued from previous page}}} \\
\hline
\endhead

\hline \multicolumn{2}{|r|}{{\tablecontinued{Continued on next page}}} \\ \hline
\endfoot

\endlastfoot


{\hyperref[Chapters/PythonModuleReference/TXLConverter/TXLWizard.TXLConverterCLI:TXLWizard.TXLConverterCLI.TXLConverterCLI]{\crossref{\code{TXLConverterCLI}}}}({[}JSONConfigurationFile, ...{]})
 & 
Provides a command line interface for the {\hyperref[Chapters/PythonModuleReference/TXLConverter/TXLWizard.TXLConverter:TXLWizard.TXLConverter.TXLConverter]{\crossref{\code{TXLWizard.TXLConverter.TXLConverter}}}} class.
\\
\hline\end{longtable}

\index{TXLConverterCLI (class in TXLWizard.TXLConverterCLI)}

\begin{fulllineitems}
\phantomsection\label{Chapters/PythonModuleReference/TXLConverter/TXLWizard.TXLConverterCLI:TXLWizard.TXLConverterCLI.TXLConverterCLI}\pysiglinewithargsret{\strong{class }\code{TXLWizard.TXLConverterCLI.}\bfcode{TXLConverterCLI}}{\emph{JSONConfigurationFile='TXLConverterConfiguration.json'}, \emph{UpdateConfigurationFile=True}, \emph{OverrideConfiguration=\{\}}}{}
Bases: \code{object}

Provides a command line interface for the {\hyperref[Chapters/PythonModuleReference/TXLConverter/TXLWizard.TXLConverter:TXLWizard.TXLConverter.TXLConverter]{\crossref{\code{TXLWizard.TXLConverter.TXLConverter}}}} class.

The configuration is read and stored in the JSON format in the file specified  in \titleref{JSONConfigurationFile}.
\begin{quote}\begin{description}
\item[{Parameters}] \leavevmode\begin{itemize}
\item {} 
\textbf{\texttt{JSONConfigurationFile}} (\emph{\texttt{str, optional}}) -- Path / Filename of the file where the configuration is read and stored in the JSON format.
Defaults to `TXLConverterConfiguration.json'

\item {} 
\textbf{\texttt{UpdateConfigurationFile}} (\emph{\texttt{bool, optional}}) -- Flag whether to update the configuration file.
Defaults to True.

\item {} 
\textbf{\texttt{OverrideConfiguration}} (\emph{\texttt{dict, optional}}) -- Dictionary with configuration options overriding the default / stored configuration.
Defaults to \{\}

\end{itemize}

\end{description}\end{quote}

\end{fulllineitems}



\chapter{Indices and tables}
\label{index:indices-and-tables}\begin{itemize}
\item {} 
\DUrole{xref,std,std-ref}{genindex}

\item {} 
\DUrole{xref,std,std-ref}{modindex}

\item {} 
\DUrole{xref,std,std-ref}{search}

\end{itemize}


\renewcommand{\indexname}{Python Module Index}
\begin{theindex}
\def\bigletter#1{{\Large\sffamily#1}\nopagebreak\vspace{1mm}}
\bigletter{t}
\item {\texttt{TXLWizard.Patterns.AbstractPattern}}, \pageref{Chapters/PythonModuleReference/Patterns/TXLWizard.Patterns.AbstractPattern:module-TXLWizard.Patterns.AbstractPattern}
\item {\texttt{TXLWizard.Patterns.Array}}, \pageref{Chapters/PythonModuleReference/Patterns/TXLWizard.Patterns.Array:module-TXLWizard.Patterns.Array}
\item {\texttt{TXLWizard.Patterns.Circle}}, \pageref{Chapters/PythonModuleReference/Patterns/TXLWizard.Patterns.Circle:module-TXLWizard.Patterns.Circle}
\item {\texttt{TXLWizard.Patterns.Ellipse}}, \pageref{Chapters/PythonModuleReference/Patterns/TXLWizard.Patterns.Ellipse:module-TXLWizard.Patterns.Ellipse}
\item {\texttt{TXLWizard.Patterns.Polygon}}, \pageref{Chapters/PythonModuleReference/Patterns/TXLWizard.Patterns.Polygon:module-TXLWizard.Patterns.Polygon}
\item {\texttt{TXLWizard.Patterns.Polyline}}, \pageref{Chapters/PythonModuleReference/Patterns/TXLWizard.Patterns.Polyline:module-TXLWizard.Patterns.Polyline}
\item {\texttt{TXLWizard.Patterns.Reference}}, \pageref{Chapters/PythonModuleReference/Patterns/TXLWizard.Patterns.Reference:module-TXLWizard.Patterns.Reference}
\item {\texttt{TXLWizard.Patterns.Structure}}, \pageref{Chapters/PythonModuleReference/Patterns/TXLWizard.Patterns.Structure:module-TXLWizard.Patterns.Structure}
\item {\texttt{TXLWizard.ShapeLibrary.AlignmentMarkers}}, \pageref{Chapters/PythonModuleReference/ShapeLibrary/TXLWizard.ShapeLibrary.AlignmentMarkers:module-TXLWizard.ShapeLibrary.AlignmentMarkers}
\item {\texttt{TXLWizard.ShapeLibrary.EndpointDetectionWindows}}, \pageref{Chapters/PythonModuleReference/ShapeLibrary/TXLWizard.ShapeLibrary.EndpointDetectionWindows:module-TXLWizard.ShapeLibrary.EndpointDetectionWindows}
\item {\texttt{TXLWizard.ShapeLibrary.Label}}, \pageref{Chapters/PythonModuleReference/ShapeLibrary/TXLWizard.ShapeLibrary.Label:module-TXLWizard.ShapeLibrary.Label}
\item {\texttt{TXLWizard.TXLConverter}}, \pageref{Chapters/PythonModuleReference/TXLConverter/TXLWizard.TXLConverter:module-TXLWizard.TXLConverter}
\item {\texttt{TXLWizard.TXLConverterCLI}}, \pageref{Chapters/PythonModuleReference/TXLConverter/TXLWizard.TXLConverterCLI:module-TXLWizard.TXLConverterCLI}
\item {\texttt{TXLWizard.TXLWriter}}, \pageref{Chapters/PythonModuleReference/TXLWriter/TXLWizard.TXLWriter:module-TXLWizard.TXLWriter}
\end{theindex}

\renewcommand{\indexname}{Index}
\printindex
\end{document}
