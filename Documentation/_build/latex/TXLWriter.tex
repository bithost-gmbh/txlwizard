% Generated by Sphinx.
\def\sphinxdocclass{report}
\documentclass[letterpaper,10pt,english]{sphinxmanual}

\usepackage[utf8]{inputenc}
\ifdefined\DeclareUnicodeCharacter
  \DeclareUnicodeCharacter{00A0}{\nobreakspace}
\else\fi
\usepackage{cmap}
\usepackage[T1]{fontenc}
\usepackage{amsmath,amssymb}
\usepackage{babel}
\usepackage{times}
\usepackage[Bjarne]{fncychap}
\usepackage{longtable}
\usepackage{sphinx}
\usepackage{multirow}
\usepackage{eqparbox}


\addto\captionsenglish{\renewcommand{\figurename}{Fig. }}
\addto\captionsenglish{\renewcommand{\tablename}{Table }}
\SetupFloatingEnvironment{literal-block}{name=Listing }

\addto\extrasenglish{\def\pageautorefname{page}}

\setcounter{tocdepth}{2}


\title{TXLWriter Documentation}
\date{May 17, 2016}
\release{1.0.0}
\author{Esteban Marin}
\newcommand{\sphinxlogo}{}
\renewcommand{\releasename}{Release}
\makeindex

\makeatletter
\def\PYG@reset{\let\PYG@it=\relax \let\PYG@bf=\relax%
    \let\PYG@ul=\relax \let\PYG@tc=\relax%
    \let\PYG@bc=\relax \let\PYG@ff=\relax}
\def\PYG@tok#1{\csname PYG@tok@#1\endcsname}
\def\PYG@toks#1+{\ifx\relax#1\empty\else%
    \PYG@tok{#1}\expandafter\PYG@toks\fi}
\def\PYG@do#1{\PYG@bc{\PYG@tc{\PYG@ul{%
    \PYG@it{\PYG@bf{\PYG@ff{#1}}}}}}}
\def\PYG#1#2{\PYG@reset\PYG@toks#1+\relax+\PYG@do{#2}}

\expandafter\def\csname PYG@tok@gd\endcsname{\def\PYG@tc##1{\textcolor[rgb]{0.63,0.00,0.00}{##1}}}
\expandafter\def\csname PYG@tok@gu\endcsname{\let\PYG@bf=\textbf\def\PYG@tc##1{\textcolor[rgb]{0.50,0.00,0.50}{##1}}}
\expandafter\def\csname PYG@tok@gt\endcsname{\def\PYG@tc##1{\textcolor[rgb]{0.00,0.27,0.87}{##1}}}
\expandafter\def\csname PYG@tok@gs\endcsname{\let\PYG@bf=\textbf}
\expandafter\def\csname PYG@tok@gr\endcsname{\def\PYG@tc##1{\textcolor[rgb]{1.00,0.00,0.00}{##1}}}
\expandafter\def\csname PYG@tok@cm\endcsname{\let\PYG@it=\textit\def\PYG@tc##1{\textcolor[rgb]{0.25,0.50,0.56}{##1}}}
\expandafter\def\csname PYG@tok@vg\endcsname{\def\PYG@tc##1{\textcolor[rgb]{0.73,0.38,0.84}{##1}}}
\expandafter\def\csname PYG@tok@vi\endcsname{\def\PYG@tc##1{\textcolor[rgb]{0.73,0.38,0.84}{##1}}}
\expandafter\def\csname PYG@tok@mh\endcsname{\def\PYG@tc##1{\textcolor[rgb]{0.13,0.50,0.31}{##1}}}
\expandafter\def\csname PYG@tok@cs\endcsname{\def\PYG@tc##1{\textcolor[rgb]{0.25,0.50,0.56}{##1}}\def\PYG@bc##1{\setlength{\fboxsep}{0pt}\colorbox[rgb]{1.00,0.94,0.94}{\strut ##1}}}
\expandafter\def\csname PYG@tok@ge\endcsname{\let\PYG@it=\textit}
\expandafter\def\csname PYG@tok@vc\endcsname{\def\PYG@tc##1{\textcolor[rgb]{0.73,0.38,0.84}{##1}}}
\expandafter\def\csname PYG@tok@il\endcsname{\def\PYG@tc##1{\textcolor[rgb]{0.13,0.50,0.31}{##1}}}
\expandafter\def\csname PYG@tok@go\endcsname{\def\PYG@tc##1{\textcolor[rgb]{0.20,0.20,0.20}{##1}}}
\expandafter\def\csname PYG@tok@cp\endcsname{\def\PYG@tc##1{\textcolor[rgb]{0.00,0.44,0.13}{##1}}}
\expandafter\def\csname PYG@tok@gi\endcsname{\def\PYG@tc##1{\textcolor[rgb]{0.00,0.63,0.00}{##1}}}
\expandafter\def\csname PYG@tok@gh\endcsname{\let\PYG@bf=\textbf\def\PYG@tc##1{\textcolor[rgb]{0.00,0.00,0.50}{##1}}}
\expandafter\def\csname PYG@tok@ni\endcsname{\let\PYG@bf=\textbf\def\PYG@tc##1{\textcolor[rgb]{0.84,0.33,0.22}{##1}}}
\expandafter\def\csname PYG@tok@nl\endcsname{\let\PYG@bf=\textbf\def\PYG@tc##1{\textcolor[rgb]{0.00,0.13,0.44}{##1}}}
\expandafter\def\csname PYG@tok@nn\endcsname{\let\PYG@bf=\textbf\def\PYG@tc##1{\textcolor[rgb]{0.05,0.52,0.71}{##1}}}
\expandafter\def\csname PYG@tok@no\endcsname{\def\PYG@tc##1{\textcolor[rgb]{0.38,0.68,0.84}{##1}}}
\expandafter\def\csname PYG@tok@na\endcsname{\def\PYG@tc##1{\textcolor[rgb]{0.25,0.44,0.63}{##1}}}
\expandafter\def\csname PYG@tok@nb\endcsname{\def\PYG@tc##1{\textcolor[rgb]{0.00,0.44,0.13}{##1}}}
\expandafter\def\csname PYG@tok@nc\endcsname{\let\PYG@bf=\textbf\def\PYG@tc##1{\textcolor[rgb]{0.05,0.52,0.71}{##1}}}
\expandafter\def\csname PYG@tok@nd\endcsname{\let\PYG@bf=\textbf\def\PYG@tc##1{\textcolor[rgb]{0.33,0.33,0.33}{##1}}}
\expandafter\def\csname PYG@tok@ne\endcsname{\def\PYG@tc##1{\textcolor[rgb]{0.00,0.44,0.13}{##1}}}
\expandafter\def\csname PYG@tok@nf\endcsname{\def\PYG@tc##1{\textcolor[rgb]{0.02,0.16,0.49}{##1}}}
\expandafter\def\csname PYG@tok@si\endcsname{\let\PYG@it=\textit\def\PYG@tc##1{\textcolor[rgb]{0.44,0.63,0.82}{##1}}}
\expandafter\def\csname PYG@tok@s2\endcsname{\def\PYG@tc##1{\textcolor[rgb]{0.25,0.44,0.63}{##1}}}
\expandafter\def\csname PYG@tok@nt\endcsname{\let\PYG@bf=\textbf\def\PYG@tc##1{\textcolor[rgb]{0.02,0.16,0.45}{##1}}}
\expandafter\def\csname PYG@tok@nv\endcsname{\def\PYG@tc##1{\textcolor[rgb]{0.73,0.38,0.84}{##1}}}
\expandafter\def\csname PYG@tok@s1\endcsname{\def\PYG@tc##1{\textcolor[rgb]{0.25,0.44,0.63}{##1}}}
\expandafter\def\csname PYG@tok@ch\endcsname{\let\PYG@it=\textit\def\PYG@tc##1{\textcolor[rgb]{0.25,0.50,0.56}{##1}}}
\expandafter\def\csname PYG@tok@m\endcsname{\def\PYG@tc##1{\textcolor[rgb]{0.13,0.50,0.31}{##1}}}
\expandafter\def\csname PYG@tok@gp\endcsname{\let\PYG@bf=\textbf\def\PYG@tc##1{\textcolor[rgb]{0.78,0.36,0.04}{##1}}}
\expandafter\def\csname PYG@tok@sh\endcsname{\def\PYG@tc##1{\textcolor[rgb]{0.25,0.44,0.63}{##1}}}
\expandafter\def\csname PYG@tok@ow\endcsname{\let\PYG@bf=\textbf\def\PYG@tc##1{\textcolor[rgb]{0.00,0.44,0.13}{##1}}}
\expandafter\def\csname PYG@tok@sx\endcsname{\def\PYG@tc##1{\textcolor[rgb]{0.78,0.36,0.04}{##1}}}
\expandafter\def\csname PYG@tok@bp\endcsname{\def\PYG@tc##1{\textcolor[rgb]{0.00,0.44,0.13}{##1}}}
\expandafter\def\csname PYG@tok@c1\endcsname{\let\PYG@it=\textit\def\PYG@tc##1{\textcolor[rgb]{0.25,0.50,0.56}{##1}}}
\expandafter\def\csname PYG@tok@o\endcsname{\def\PYG@tc##1{\textcolor[rgb]{0.40,0.40,0.40}{##1}}}
\expandafter\def\csname PYG@tok@kc\endcsname{\let\PYG@bf=\textbf\def\PYG@tc##1{\textcolor[rgb]{0.00,0.44,0.13}{##1}}}
\expandafter\def\csname PYG@tok@c\endcsname{\let\PYG@it=\textit\def\PYG@tc##1{\textcolor[rgb]{0.25,0.50,0.56}{##1}}}
\expandafter\def\csname PYG@tok@mf\endcsname{\def\PYG@tc##1{\textcolor[rgb]{0.13,0.50,0.31}{##1}}}
\expandafter\def\csname PYG@tok@err\endcsname{\def\PYG@bc##1{\setlength{\fboxsep}{0pt}\fcolorbox[rgb]{1.00,0.00,0.00}{1,1,1}{\strut ##1}}}
\expandafter\def\csname PYG@tok@mb\endcsname{\def\PYG@tc##1{\textcolor[rgb]{0.13,0.50,0.31}{##1}}}
\expandafter\def\csname PYG@tok@ss\endcsname{\def\PYG@tc##1{\textcolor[rgb]{0.32,0.47,0.09}{##1}}}
\expandafter\def\csname PYG@tok@sr\endcsname{\def\PYG@tc##1{\textcolor[rgb]{0.14,0.33,0.53}{##1}}}
\expandafter\def\csname PYG@tok@mo\endcsname{\def\PYG@tc##1{\textcolor[rgb]{0.13,0.50,0.31}{##1}}}
\expandafter\def\csname PYG@tok@kd\endcsname{\let\PYG@bf=\textbf\def\PYG@tc##1{\textcolor[rgb]{0.00,0.44,0.13}{##1}}}
\expandafter\def\csname PYG@tok@mi\endcsname{\def\PYG@tc##1{\textcolor[rgb]{0.13,0.50,0.31}{##1}}}
\expandafter\def\csname PYG@tok@kn\endcsname{\let\PYG@bf=\textbf\def\PYG@tc##1{\textcolor[rgb]{0.00,0.44,0.13}{##1}}}
\expandafter\def\csname PYG@tok@cpf\endcsname{\let\PYG@it=\textit\def\PYG@tc##1{\textcolor[rgb]{0.25,0.50,0.56}{##1}}}
\expandafter\def\csname PYG@tok@kr\endcsname{\let\PYG@bf=\textbf\def\PYG@tc##1{\textcolor[rgb]{0.00,0.44,0.13}{##1}}}
\expandafter\def\csname PYG@tok@s\endcsname{\def\PYG@tc##1{\textcolor[rgb]{0.25,0.44,0.63}{##1}}}
\expandafter\def\csname PYG@tok@kp\endcsname{\def\PYG@tc##1{\textcolor[rgb]{0.00,0.44,0.13}{##1}}}
\expandafter\def\csname PYG@tok@w\endcsname{\def\PYG@tc##1{\textcolor[rgb]{0.73,0.73,0.73}{##1}}}
\expandafter\def\csname PYG@tok@kt\endcsname{\def\PYG@tc##1{\textcolor[rgb]{0.56,0.13,0.00}{##1}}}
\expandafter\def\csname PYG@tok@sc\endcsname{\def\PYG@tc##1{\textcolor[rgb]{0.25,0.44,0.63}{##1}}}
\expandafter\def\csname PYG@tok@sb\endcsname{\def\PYG@tc##1{\textcolor[rgb]{0.25,0.44,0.63}{##1}}}
\expandafter\def\csname PYG@tok@k\endcsname{\let\PYG@bf=\textbf\def\PYG@tc##1{\textcolor[rgb]{0.00,0.44,0.13}{##1}}}
\expandafter\def\csname PYG@tok@se\endcsname{\let\PYG@bf=\textbf\def\PYG@tc##1{\textcolor[rgb]{0.25,0.44,0.63}{##1}}}
\expandafter\def\csname PYG@tok@sd\endcsname{\let\PYG@it=\textit\def\PYG@tc##1{\textcolor[rgb]{0.25,0.44,0.63}{##1}}}

\def\PYGZbs{\char`\\}
\def\PYGZus{\char`\_}
\def\PYGZob{\char`\{}
\def\PYGZcb{\char`\}}
\def\PYGZca{\char`\^}
\def\PYGZam{\char`\&}
\def\PYGZlt{\char`\<}
\def\PYGZgt{\char`\>}
\def\PYGZsh{\char`\#}
\def\PYGZpc{\char`\%}
\def\PYGZdl{\char`\$}
\def\PYGZhy{\char`\-}
\def\PYGZsq{\char`\'}
\def\PYGZdq{\char`\"}
\def\PYGZti{\char`\~}
% for compatibility with earlier versions
\def\PYGZat{@}
\def\PYGZlb{[}
\def\PYGZrb{]}
\makeatother

\renewcommand\PYGZsq{\textquotesingle}

\begin{document}

\maketitle
\tableofcontents
\phantomsection\label{index::doc}


Contents:


\chapter{Introduction}
\label{Chapters/10_Introduction:introduction}\label{Chapters/10_Introduction::doc}\label{Chapters/10_Introduction:welcome-to-txlwriter-s-documentation}
This document describes the usage and technical reference of the python program \titleref{TXLWizard}
written by Esteban Marin (\href{mailto:estebanmarin@gmx.ch}{estebanmarin@gmx.ch}).


\section{What does it do?}
\label{Chapters/10_Introduction:what-does-it-do}
The \titleref{TXLWizard} provides routines for generating TXL files (.txl) for
the preparation of E-Beam lithography masks using python code. The TXL files can be processed with BEAMER.
See the following links:
\begin{itemize}
\item {} 
\url{http://genisys-gmbh.com/web/products/beamer.html}

\item {} 
\url{http://cad035.psi.ch/LB\_index.html}

\item {} 
\url{http://cad035.psi.ch/LBDoc/BEAMER\_Manual.pdf}

\end{itemize}

The generated TXL files are also converted to HTML / SVG for presentation in any modern browser or
vector graphics application.

Moreover, a command line interface \titleref{TXLConverter} provides conversion of existing TXL files to HTML / SVG
(See Section {\hyperref[Chapters/30_TXLConverter:sec\string-txlconverter]{\crossref{\DUrole{std,std-ref}{TXLConverter}}}}).


\section{Technical Information}
\label{Chapters/10_Introduction:technical-information}
The ``TXLWizard'' is written in python and will run in Python version 2.7+ and 3.1+.
In order to use it, the \titleref{TXLWizard} package must be available as
a python package, i.e. either it must be copied to
\begin{quote}

\code{Path\_to\_my\_python\_installation/site-packages/}
\end{quote}

or to the path where your script is located.
\begin{description}
\item[{Alternatively, you can also prepend the following command to your python script:}] \leavevmode
\code{sys.path.append('path to the folder containing TXLWizard')}

\end{description}


\chapter{TXLWizard Example}
\label{Chapters/20_Example:txlwizard-example}\label{Chapters/20_Example::doc}
\begin{DUlineblock}{0em}
\item[] The following code demonstrates a simple example usage of the \titleref{TXLWizard} for
generating TXL files with python code.
\item[] The code can be found in the file \code{/Content/Example\_Simple.py}.
\item[] The resulting image is shown in Figure \DUrole{xref,std,std-ref}{fig-TXLWizardSimpleExample}.
\item[] A more advanced example is shown in Section \DUrole{xref,std,std-ref}{AppendixTXLWizardExampleAdvanced}
\end{DUlineblock}

\begin{Verbatim}[commandchars=\\\{\},numbers=left,firstnumber=1,stepnumber=1]
\PYG{c+c1}{\PYGZsh{}\PYGZsh{}\PYGZsh{}\PYGZsh{}\PYGZsh{}\PYGZsh{}\PYGZsh{}\PYGZsh{}\PYGZsh{}\PYGZsh{}\PYGZsh{}\PYGZsh{}\PYGZsh{}\PYGZsh{}\PYGZsh{}\PYGZsh{}\PYGZsh{}\PYGZsh{}\PYGZsh{}\PYGZsh{}}
\PYG{c+c1}{\PYGZsh{} Import Libraries \PYGZsh{}}
\PYG{c+c1}{\PYGZsh{}\PYGZsh{}\PYGZsh{}\PYGZsh{}\PYGZsh{}\PYGZsh{}\PYGZsh{}\PYGZsh{}\PYGZsh{}\PYGZsh{}\PYGZsh{}\PYGZsh{}\PYGZsh{}\PYGZsh{}\PYGZsh{}\PYGZsh{}\PYGZsh{}\PYGZsh{}\PYGZsh{}\PYGZsh{}}

\PYG{c+c1}{\PYGZsh{} Import TXLWriter, the main class for generating TXL Output}
\PYG{k+kn}{import} \PYG{n+nn}{TXLWizard}\PYG{n+nn}{.}\PYG{n+nn}{TXLWriter}

\PYG{c+c1}{\PYGZsh{} Import Pre\PYGZhy{}Defined Shapes / Structures wrapped in functions}
\PYG{k+kn}{import} \PYG{n+nn}{TXLWizard}\PYG{n+nn}{.}\PYG{n+nn}{ShapeLibrary}\PYG{n+nn}{.}\PYG{n+nn}{EndpointDetectionWindows}
\PYG{k+kn}{import} \PYG{n+nn}{TXLWizard}\PYG{n+nn}{.}\PYG{n+nn}{ShapeLibrary}\PYG{n+nn}{.}\PYG{n+nn}{Label}

\PYG{c+c1}{\PYGZsh{} Import math module for calculations}
\PYG{k+kn}{import} \PYG{n+nn}{math}


\PYG{c+c1}{\PYGZsh{}\PYGZsh{}\PYGZsh{}\PYGZsh{}\PYGZsh{}\PYGZsh{}\PYGZsh{}\PYGZsh{}\PYGZsh{}\PYGZsh{}\PYGZsh{}\PYGZsh{}\PYGZsh{}\PYGZsh{}\PYGZsh{}\PYGZsh{}\PYGZsh{}\PYGZsh{}\PYGZsh{}\PYGZsh{}\PYGZsh{}\PYGZsh{}\PYGZsh{}\PYGZsh{}\PYGZsh{}\PYGZsh{}\PYGZsh{}\PYGZsh{}\PYGZsh{}\PYGZsh{}\PYGZsh{}\PYGZsh{}\PYGZsh{}}
\PYG{c+c1}{\PYGZsh{} Sample / Structure Parameters \PYGZsh{}}
\PYG{c+c1}{\PYGZsh{}\PYGZsh{}\PYGZsh{}\PYGZsh{}\PYGZsh{}\PYGZsh{}\PYGZsh{}\PYGZsh{}\PYGZsh{}\PYGZsh{}\PYGZsh{}\PYGZsh{}\PYGZsh{}\PYGZsh{}\PYGZsh{}\PYGZsh{}\PYGZsh{}\PYGZsh{}\PYGZsh{}\PYGZsh{}\PYGZsh{}\PYGZsh{}\PYGZsh{}\PYGZsh{}\PYGZsh{}\PYGZsh{}\PYGZsh{}\PYGZsh{}\PYGZsh{}\PYGZsh{}\PYGZsh{}\PYGZsh{}\PYGZsh{}}

\PYG{c+c1}{\PYGZsh{} Define all sample parameters}
\PYG{n}{SampleParameters} \PYG{o}{=} \PYG{p}{\PYGZob{}}
    \PYG{l+s+s1}{\PYGZsq{}}\PYG{l+s+s1}{Width}\PYG{l+s+s1}{\PYGZsq{}}\PYG{p}{:} \PYG{l+m+mi}{8}\PYG{n}{e3}\PYG{p}{,}
    \PYG{l+s+s1}{\PYGZsq{}}\PYG{l+s+s1}{Height}\PYG{l+s+s1}{\PYGZsq{}}\PYG{p}{:} \PYG{l+m+mi}{8}\PYG{n}{e3}\PYG{p}{,}
    \PYG{l+s+s1}{\PYGZsq{}}\PYG{l+s+s1}{Label}\PYG{l+s+s1}{\PYGZsq{}}\PYG{p}{:} \PYG{l+s+s1}{\PYGZsq{}}\PYG{l+s+s1}{Simple Demo}\PYG{l+s+s1}{\PYGZsq{}}\PYG{p}{,}
\PYG{p}{\PYGZcb{}}

\PYG{c+c1}{\PYGZsh{} Define all structure parameters}
\PYG{n}{StructureParameters} \PYG{o}{=} \PYG{p}{\PYGZob{}}
    \PYG{l+s+s1}{\PYGZsq{}}\PYG{l+s+s1}{Circle}\PYG{l+s+s1}{\PYGZsq{}}\PYG{p}{:} \PYG{p}{\PYGZob{}}
        \PYG{l+s+s1}{\PYGZsq{}}\PYG{l+s+s1}{Radius}\PYG{l+s+s1}{\PYGZsq{}}\PYG{p}{:} \PYG{l+m+mi}{50}\PYG{p}{,}
        \PYG{l+s+s1}{\PYGZsq{}}\PYG{l+s+s1}{Layer}\PYG{l+s+s1}{\PYGZsq{}}\PYG{p}{:} \PYG{l+m+mi}{3}
    \PYG{p}{\PYGZcb{}}\PYG{p}{,}
    \PYG{l+s+s1}{\PYGZsq{}}\PYG{l+s+s1}{CircleArray}\PYG{l+s+s1}{\PYGZsq{}}\PYG{p}{:} \PYG{p}{\PYGZob{}}
        \PYG{l+s+s1}{\PYGZsq{}}\PYG{l+s+s1}{Columns}\PYG{l+s+s1}{\PYGZsq{}}\PYG{p}{:} \PYG{l+m+mi}{6}\PYG{p}{,}
        \PYG{l+s+s1}{\PYGZsq{}}\PYG{l+s+s1}{Rows}\PYG{l+s+s1}{\PYGZsq{}}\PYG{p}{:} \PYG{l+m+mi}{5}\PYG{p}{,}
        \PYG{l+s+s1}{\PYGZsq{}}\PYG{l+s+s1}{ArrayXOffset}\PYG{l+s+s1}{\PYGZsq{}}\PYG{p}{:} \PYG{l+m+mi}{500}\PYG{p}{,}
        \PYG{l+s+s1}{\PYGZsq{}}\PYG{l+s+s1}{ArrayYOffset}\PYG{l+s+s1}{\PYGZsq{}}\PYG{p}{:} \PYG{o}{\PYGZhy{}}\PYG{l+m+mi}{500}\PYG{p}{,}
        \PYG{l+s+s1}{\PYGZsq{}}\PYG{l+s+s1}{ArrayOrigin}\PYG{l+s+s1}{\PYGZsq{}}\PYG{p}{:} \PYG{p}{[}\PYG{l+m+mf}{0.75e3}\PYG{p}{,} \PYG{l+m+mi}{3}\PYG{n}{e3}\PYG{p}{]}\PYG{p}{,}
        \PYG{l+s+s1}{\PYGZsq{}}\PYG{l+s+s1}{Label}\PYG{l+s+s1}{\PYGZsq{}}\PYG{p}{:} \PYG{l+s+s1}{\PYGZsq{}}\PYG{l+s+s1}{R}\PYG{l+s+si}{\PYGZob{}:d\PYGZcb{}}\PYG{l+s+s1}{C}\PYG{l+s+si}{\PYGZob{}:d\PYGZcb{}}\PYG{l+s+s1}{\PYGZsq{}}\PYG{p}{,}
    \PYG{p}{\PYGZcb{}}
\PYG{p}{\PYGZcb{}}


\PYG{c+c1}{\PYGZsh{}\PYGZsh{}\PYGZsh{}\PYGZsh{}\PYGZsh{}\PYGZsh{}\PYGZsh{}\PYGZsh{}\PYGZsh{}\PYGZsh{}\PYGZsh{}\PYGZsh{}\PYGZsh{}\PYGZsh{}\PYGZsh{}\PYGZsh{}\PYGZsh{}\PYGZsh{}\PYGZsh{}\PYGZsh{}\PYGZsh{}\PYGZsh{}\PYGZsh{}\PYGZsh{}}
\PYG{c+c1}{\PYGZsh{} Initialize TXLWriter \PYGZsh{}}
\PYG{c+c1}{\PYGZsh{}\PYGZsh{}\PYGZsh{}\PYGZsh{}\PYGZsh{}\PYGZsh{}\PYGZsh{}\PYGZsh{}\PYGZsh{}\PYGZsh{}\PYGZsh{}\PYGZsh{}\PYGZsh{}\PYGZsh{}\PYGZsh{}\PYGZsh{}\PYGZsh{}\PYGZsh{}\PYGZsh{}\PYGZsh{}\PYGZsh{}\PYGZsh{}\PYGZsh{}\PYGZsh{}}
\PYG{n}{TXLWriter} \PYG{o}{=} \PYG{n}{TXLWizard}\PYG{o}{.}\PYG{n}{TXLWriter}\PYG{o}{.}\PYG{n}{TXLWriter}\PYG{p}{(}
    \PYG{n}{GridWidth}\PYG{o}{=}\PYG{n}{SampleParameters}\PYG{p}{[}\PYG{l+s+s1}{\PYGZsq{}}\PYG{l+s+s1}{Width}\PYG{l+s+s1}{\PYGZsq{}}\PYG{p}{]}\PYG{p}{,}
    \PYG{n}{GridHeight}\PYG{o}{=}\PYG{n}{SampleParameters}\PYG{p}{[}\PYG{l+s+s1}{\PYGZsq{}}\PYG{l+s+s1}{Height}\PYG{l+s+s1}{\PYGZsq{}}\PYG{p}{]}
\PYG{p}{)}

\PYG{c+c1}{\PYGZsh{}\PYGZsh{}\PYGZsh{}\PYGZsh{}\PYGZsh{}\PYGZsh{}\PYGZsh{}\PYGZsh{}\PYGZsh{}\PYGZsh{}\PYGZsh{}\PYGZsh{}\PYGZsh{}\PYGZsh{}\PYGZsh{}\PYGZsh{}\PYGZsh{}\PYGZsh{}\PYGZsh{}\PYGZsh{}\PYGZsh{}}
\PYG{c+c1}{\PYGZsh{} Define Structures \PYGZsh{}}
\PYG{c+c1}{\PYGZsh{}\PYGZsh{}\PYGZsh{}\PYGZsh{}\PYGZsh{}\PYGZsh{}\PYGZsh{}\PYGZsh{}\PYGZsh{}\PYGZsh{}\PYGZsh{}\PYGZsh{}\PYGZsh{}\PYGZsh{}\PYGZsh{}\PYGZsh{}\PYGZsh{}\PYGZsh{}\PYGZsh{}\PYGZsh{}\PYGZsh{}}

\PYG{c+c1}{\PYGZsh{}\PYGZsh{} Sample Label \PYGZsh{}\PYGZsh{}}

\PYG{c+c1}{\PYGZsh{} Give the sample a nice label}
\PYG{n}{SampleLabelObject} \PYG{o}{=} \PYG{n}{TXLWizard}\PYG{o}{.}\PYG{n}{ShapeLibrary}\PYG{o}{.}\PYG{n}{Label}\PYG{o}{.}\PYG{n}{GetLabel}\PYG{p}{(}
    \PYG{n}{TXLWriter}\PYG{p}{,}
    \PYG{n}{SampleParameters}\PYG{p}{[}\PYG{l+s+s1}{\PYGZsq{}}\PYG{l+s+s1}{Label}\PYG{l+s+s1}{\PYGZsq{}}\PYG{p}{]}\PYG{p}{,}
    \PYG{n}{OriginPoint}\PYG{o}{=}\PYG{p}{[}
        \PYG{l+m+mf}{0.5e3}\PYG{p}{,} \PYG{l+m+mf}{1.} \PYG{o}{*} \PYG{n}{SampleParameters}\PYG{p}{[}\PYG{l+s+s1}{\PYGZsq{}}\PYG{l+s+s1}{Height}\PYG{l+s+s1}{\PYGZsq{}}\PYG{p}{]} \PYG{o}{/} \PYG{l+m+mf}{2.} \PYG{o}{\PYGZhy{}} \PYG{l+m+mi}{500}
    \PYG{p}{]}\PYG{p}{,}
    \PYG{n}{FontSize}\PYG{o}{=}\PYG{l+m+mi}{150}\PYG{p}{,}
    \PYG{n}{StrokeWidth}\PYG{o}{=}\PYG{l+m+mi}{20}\PYG{p}{,}
    \PYG{n}{RoundCaps}\PYG{o}{=}\PYG{k+kc}{True}\PYG{p}{,} \PYG{c+c1}{\PYGZsh{} Set to False to improve e\PYGZhy{}Beam performance}
    \PYG{n}{Layer}\PYG{o}{=}\PYG{l+m+mi}{1}
\PYG{p}{)}


\PYG{c+c1}{\PYGZsh{}\PYGZsh{} Endpoint Detection \PYGZsh{}\PYGZsh{}}

\PYG{c+c1}{\PYGZsh{} Use Pre\PYGZhy{}Defined Endpoint Detection Windows}
\PYG{n}{TXLWizard}\PYG{o}{.}\PYG{n}{ShapeLibrary}\PYG{o}{.}\PYG{n}{EndpointDetectionWindows}\PYG{o}{.}\PYG{n}{GetEndpointDetectionWindows}\PYG{p}{(}
    \PYG{n}{TXLWriter}\PYG{p}{,} \PYG{n}{Layer}\PYG{o}{=}\PYG{l+m+mi}{1}\PYG{p}{)}

\PYG{c+c1}{\PYGZsh{}\PYGZsh{} User Structure: Circle \PYGZsh{}\PYGZsh{}}

\PYG{c+c1}{\PYGZsh{} Create Definition Structure for Circle that will be reused}
\PYG{n}{CircleStructure} \PYG{o}{=} \PYG{n}{TXLWriter}\PYG{o}{.}\PYG{n}{AddDefinitionStructure}\PYG{p}{(}\PYG{l+s+s1}{\PYGZsq{}}\PYG{l+s+s1}{Circle}\PYG{l+s+s1}{\PYGZsq{}}\PYG{p}{)}
\PYG{n}{CircleStructure}\PYG{o}{.}\PYG{n}{AddPattern}\PYG{p}{(}\PYG{l+s+s1}{\PYGZsq{}}\PYG{l+s+s1}{Circle}\PYG{l+s+s1}{\PYGZsq{}}\PYG{p}{,}
    \PYG{n}{Center}\PYG{o}{=}\PYG{p}{[}\PYG{l+m+mi}{0}\PYG{p}{,} \PYG{l+m+mi}{0}\PYG{p}{]}\PYG{p}{,}
    \PYG{n}{Radius}\PYG{o}{=}\PYG{n}{StructureParameters}\PYG{p}{[}\PYG{l+s+s1}{\PYGZsq{}}\PYG{l+s+s1}{Circle}\PYG{l+s+s1}{\PYGZsq{}}\PYG{p}{]}\PYG{p}{[}\PYG{l+s+s1}{\PYGZsq{}}\PYG{l+s+s1}{Radius}\PYG{l+s+s1}{\PYGZsq{}}\PYG{p}{]}\PYG{p}{,}
    \PYG{n}{Layer}\PYG{o}{=}\PYG{n}{StructureParameters}\PYG{p}{[}\PYG{l+s+s1}{\PYGZsq{}}\PYG{l+s+s1}{Circle}\PYG{l+s+s1}{\PYGZsq{}}\PYG{p}{]}\PYG{p}{[}\PYG{l+s+s1}{\PYGZsq{}}\PYG{l+s+s1}{Layer}\PYG{l+s+s1}{\PYGZsq{}}\PYG{p}{]}

\PYG{p}{)}

\PYG{c+c1}{\PYGZsh{} Create array of the definition structure above}
\PYG{n}{CircleArray} \PYG{o}{=} \PYG{n}{TXLWriter}\PYG{o}{.}\PYG{n}{AddContentStructure}\PYG{p}{(}\PYG{l+s+s1}{\PYGZsq{}}\PYG{l+s+s1}{CircleArray}\PYG{l+s+s1}{\PYGZsq{}}\PYG{p}{)}
\PYG{n}{CircleArray}\PYG{o}{.}\PYG{n}{AddPattern}\PYG{p}{(}\PYG{l+s+s1}{\PYGZsq{}}\PYG{l+s+s1}{Array}\PYG{l+s+s1}{\PYGZsq{}}\PYG{p}{,}
    \PYG{n}{ReferencedStructureID}\PYG{o}{=}\PYG{n}{CircleStructure}\PYG{o}{.}\PYG{n}{ID}\PYG{p}{,}
    \PYG{n}{OriginPoint}\PYG{o}{=}\PYG{n}{StructureParameters}\PYG{p}{[}\PYG{l+s+s1}{\PYGZsq{}}\PYG{l+s+s1}{CircleArray}\PYG{l+s+s1}{\PYGZsq{}}\PYG{p}{]}\PYG{p}{[}\PYG{l+s+s1}{\PYGZsq{}}\PYG{l+s+s1}{ArrayOrigin}\PYG{l+s+s1}{\PYGZsq{}}\PYG{p}{]}\PYG{p}{,}
    \PYG{n}{PositionDelta1}\PYG{o}{=}\PYG{p}{[}
        \PYG{n}{StructureParameters}\PYG{p}{[}\PYG{l+s+s1}{\PYGZsq{}}\PYG{l+s+s1}{CircleArray}\PYG{l+s+s1}{\PYGZsq{}}\PYG{p}{]}\PYG{p}{[}\PYG{l+s+s1}{\PYGZsq{}}\PYG{l+s+s1}{ArrayXOffset}\PYG{l+s+s1}{\PYGZsq{}}\PYG{p}{]}\PYG{p}{,} \PYG{l+m+mi}{0}
    \PYG{p}{]}\PYG{p}{,}
    \PYG{n}{PositionDelta2}\PYG{o}{=}\PYG{p}{[}
        \PYG{l+m+mi}{0}\PYG{p}{,} \PYG{n}{StructureParameters}\PYG{p}{[}\PYG{l+s+s1}{\PYGZsq{}}\PYG{l+s+s1}{CircleArray}\PYG{l+s+s1}{\PYGZsq{}}\PYG{p}{]}\PYG{p}{[}\PYG{l+s+s1}{\PYGZsq{}}\PYG{l+s+s1}{ArrayYOffset}\PYG{l+s+s1}{\PYGZsq{}}\PYG{p}{]}
    \PYG{p}{]}\PYG{p}{,}
    \PYG{n}{Repetitions1}\PYG{o}{=}\PYG{n}{StructureParameters}\PYG{p}{[}\PYG{l+s+s1}{\PYGZsq{}}\PYG{l+s+s1}{CircleArray}\PYG{l+s+s1}{\PYGZsq{}}\PYG{p}{]}\PYG{p}{[}\PYG{l+s+s1}{\PYGZsq{}}\PYG{l+s+s1}{Columns}\PYG{l+s+s1}{\PYGZsq{}}\PYG{p}{]}\PYG{p}{,}
    \PYG{n}{Repetitions2}\PYG{o}{=}\PYG{n}{StructureParameters}\PYG{p}{[}\PYG{l+s+s1}{\PYGZsq{}}\PYG{l+s+s1}{CircleArray}\PYG{l+s+s1}{\PYGZsq{}}\PYG{p}{]}\PYG{p}{[}\PYG{l+s+s1}{\PYGZsq{}}\PYG{l+s+s1}{Rows}\PYG{l+s+s1}{\PYGZsq{}}\PYG{p}{]}
\PYG{p}{)}



\PYG{c+c1}{\PYGZsh{}\PYGZsh{}\PYGZsh{}\PYGZsh{}\PYGZsh{}\PYGZsh{}\PYGZsh{}\PYGZsh{}\PYGZsh{}\PYGZsh{}\PYGZsh{}\PYGZsh{}\PYGZsh{}\PYGZsh{}\PYGZsh{}\PYGZsh{}\PYGZsh{}\PYGZsh{}\PYGZsh{}\PYGZsh{}\PYGZsh{}\PYGZsh{}\PYGZsh{}\PYGZsh{}\PYGZsh{}}
\PYG{c+c1}{\PYGZsh{} Generate Output Files \PYGZsh{}}
\PYG{c+c1}{\PYGZsh{}\PYGZsh{}\PYGZsh{}\PYGZsh{}\PYGZsh{}\PYGZsh{}\PYGZsh{}\PYGZsh{}\PYGZsh{}\PYGZsh{}\PYGZsh{}\PYGZsh{}\PYGZsh{}\PYGZsh{}\PYGZsh{}\PYGZsh{}\PYGZsh{}\PYGZsh{}\PYGZsh{}\PYGZsh{}\PYGZsh{}\PYGZsh{}\PYGZsh{}\PYGZsh{}\PYGZsh{}}

\PYG{c+c1}{\PYGZsh{} Note: The suffix (.txl, .html, .svg) will be appended automatically}
\PYG{n}{TXLWriter}\PYG{o}{.}\PYG{n}{GenerateFiles}\PYG{p}{(}\PYG{l+s+s1}{\PYGZsq{}}\PYG{l+s+s1}{Masks/Example\PYGZus{}Simple}\PYG{l+s+s1}{\PYGZsq{}}\PYG{p}{)}

\end{Verbatim}
\begin{figure}[htbp]
\centering
\capstart

\includegraphics{{Example_Simple}.pdf}
\caption{Simple Example: Generated Mask}\label{Chapters/20_Example:fig-txlwizardsimpleexample}\label{Chapters/20_Example:id2}\end{figure}


\chapter{TXLConverter}
\label{Chapters/30_TXLConverter:txlconverter}\label{Chapters/30_TXLConverter:sec-txlconverter}\label{Chapters/30_TXLConverter::doc}
blub


\chapter{Python Module Reference}
\label{Chapters/40_PythonModuleReference:python-module-reference}\label{Chapters/40_PythonModuleReference:module-TXLWizard.TXLWriter}\label{Chapters/40_PythonModuleReference::doc}\index{TXLWizard.TXLWriter (module)}\index{TXLWriter (class in TXLWizard.TXLWriter)}

\begin{fulllineitems}
\phantomsection\label{Chapters/40_PythonModuleReference:TXLWizard.TXLWriter.TXLWriter}\pysiglinewithargsret{\strong{class }\code{TXLWizard.TXLWriter.}\bfcode{TXLWriter}}{\emph{**kwargs}}{}
Controller class for generating TXL / SVG / HTML output.

Here we can add structures (definitions and content) which will be rendered in the output.
Optionally a coordinate system grid is drawn.
\begin{quote}\begin{description}
\item[{Parameters}] \leavevmode\begin{itemize}
\item {} 
\textbf{\texttt{ShowGrid}} (\emph{\texttt{bool, optional}}) -- 
Show the coordinate system grid or not.

Defaults to True


\item {} 
\textbf{\texttt{GridWidth}} (\emph{\texttt{int, optional}}) -- 
Full width of the coordinate system grid in um.

Defaults to 800


\item {} 
\textbf{\texttt{GridHeight}} (\emph{\texttt{int, optional}}) -- 
Full height of the coordinate system grid in um.

Defaults to 800


\item {} 
\textbf{\texttt{GridSpacing}} (\emph{\texttt{int, optional}}) -- 
Coordinate Sytem Grid Spacing in um.

Defaults to 100


\item {} 
\textbf{\texttt{SubGridSpacing}} (\emph{\texttt{int, optional}}) -- 
Coordinate System Sub-Grid Spacing in um.

Defaults to 10


\end{itemize}

\end{description}\end{quote}
\index{AddContentStructure() (TXLWizard.TXLWriter.TXLWriter method)}

\begin{fulllineitems}
\phantomsection\label{Chapters/40_PythonModuleReference:TXLWizard.TXLWriter.TXLWriter.AddContentStructure}\pysiglinewithargsret{\bfcode{AddContentStructure}}{\emph{Index}, \emph{**kwargs}}{}
Add content structure. A content structure can hold patterns that will render in the output.

A structure corresponds to the ``STRUCT'' command in the TXL file format.
\begin{quote}\begin{description}
\item[{Parameters}] \leavevmode\begin{itemize}
\item {} 
\textbf{\texttt{Index}} (\emph{\texttt{str}}) -- Unique identification of the structure. Must be used when referencing to this structure.

\item {} 
\textbf{\texttt{kwargs}} (\emph{\texttt{dict}}) -- keyword arguments passed to the structure constructor

\end{itemize}

\item[{Returns}] \leavevmode


\item[{Return type}] \leavevmode
\code{Structure} structure instance

\end{description}\end{quote}

\end{fulllineitems}

\index{AddDefinitionStructure() (TXLWizard.TXLWriter.TXLWriter method)}

\begin{fulllineitems}
\phantomsection\label{Chapters/40_PythonModuleReference:TXLWizard.TXLWriter.TXLWriter.AddDefinitionStructure}\pysiglinewithargsret{\bfcode{AddDefinitionStructure}}{\emph{Index}, \emph{**kwargs}}{}
Add definition structure. A definition structure can be referenced by a content structure.

A structure corresponds to the ``STRUCT'' command in the TXL file format.
\begin{quote}\begin{description}
\item[{Parameters}] \leavevmode\begin{itemize}
\item {} 
\textbf{\texttt{Index}} (\emph{\texttt{str}}) -- Unique identification of the structure. Must be used when referencing to this structure.

\item {} 
\textbf{\texttt{kwargs}} (\emph{\texttt{dict}}) -- keyword arguments passed to the structure constructor

\end{itemize}

\item[{Returns}] \leavevmode


\item[{Return type}] \leavevmode
\code{Structure} structure instance

\end{description}\end{quote}

\end{fulllineitems}

\index{AddHelperStructure() (TXLWizard.TXLWriter.TXLWriter method)}

\begin{fulllineitems}
\phantomsection\label{Chapters/40_PythonModuleReference:TXLWizard.TXLWriter.TXLWriter.AddHelperStructure}\pysiglinewithargsret{\bfcode{AddHelperStructure}}{\emph{Index}, \emph{**kwargs}}{}
Add helper structure. Helper structures are only visible in the HTML / SVG Output.

A structure corresponds to the ``STRUCT'' command in the TXL file format.
\begin{quote}\begin{description}
\item[{Parameters}] \leavevmode\begin{itemize}
\item {} 
\textbf{\texttt{Index}} (\emph{\texttt{str}}) -- Unique identification of the structure. Must be used when referencing to this structure.

\item {} 
\textbf{\texttt{kwargs}} (\emph{\texttt{dict}}) -- keyword arguments passed to the structure constructor

\end{itemize}

\item[{Returns}] \leavevmode


\item[{Return type}] \leavevmode
\code{Structure} structure instance

\end{description}\end{quote}

\end{fulllineitems}

\index{GenerateFiles() (TXLWizard.TXLWriter.TXLWriter method)}

\begin{fulllineitems}
\phantomsection\label{Chapters/40_PythonModuleReference:TXLWizard.TXLWriter.TXLWriter.GenerateFiles}\pysiglinewithargsret{\bfcode{GenerateFiles}}{\emph{Filename}, \emph{TXL=True}, \emph{SVG=True}, \emph{HTML=True}}{}
Generate the output files (.txl, .svg, .html).
\begin{quote}\begin{description}
\item[{Parameters}] \leavevmode\begin{itemize}
\item {} 
\textbf{\texttt{Filename}} (\emph{\texttt{str}}) -- Path / Filename without extension.
The corresponding path will be created if it does not exist

\item {} 
\textbf{\texttt{TXL}} (\emph{\texttt{Optional{[}bool{]}}}) -- Enable TXL Output

\item {} 
\textbf{\texttt{SVG}} (\emph{\texttt{Optional{[}bool{]}}}) -- Enable SVG Output

\item {} 
\textbf{\texttt{HTML}} (\emph{\texttt{Optional{[}bool{]}}}) -- Enable HTML Output

\end{itemize}

\end{description}\end{quote}

\end{fulllineitems}


\end{fulllineitems}



\chapter{Indices and tables}
\label{index:indices-and-tables}\begin{itemize}
\item {} 
\DUrole{xref,std,std-ref}{genindex}

\item {} 
\DUrole{xref,std,std-ref}{modindex}

\item {} 
\DUrole{xref,std,std-ref}{search}

\end{itemize}


\renewcommand{\indexname}{Python Module Index}
\begin{theindex}
\def\bigletter#1{{\Large\sffamily#1}\nopagebreak\vspace{1mm}}
\bigletter{t}
\item {\texttt{TXLWizard.TXLWriter}}, \pageref{Chapters/40_PythonModuleReference:module-TXLWizard.TXLWriter}
\end{theindex}

\renewcommand{\indexname}{Index}
\printindex
\end{document}
